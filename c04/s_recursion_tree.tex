\startsection[
  title={The recursion-tree method for solving recurrences},
]

%e4.4-1
\startEXERCISE
畫出下列遞迴式的遞迴樹,
並猜測其漸進上界。
然後用替代法證明之。
\startigBase[a]
% a
\startitem
$T(n)=T(n/2)+n^3$
\stopitem

\startANSWER
$T(n)\le c n^3$,
\startsplitformula\startmathalignment
\NC T(n/2) \NC \le c (n/2)^3 \NR
\NC T(n) \NC \le c (n/2)^3 + n^3 \NR
\stopmathalignment\stopsplitformula
有 $c/8 + 1\le c$,解得 $c\ge 8/7$。

{\externalfigure[output/e4_4_1-1]}
\stopANSWER

% b
\startitem
$T(n)=4T(n/3)+n$
\stopitem

\startANSWER
$T(n)\le c (\lg_3 n)4^{\lg_3 n}$,
即 $T(n)\le c k 4^k$,其中 $k=\lg_3 n$:
\startsplitformula\startmathalignment
\NC T(n/3) \NC \le c(k-1)4^{k-1} \NR
\NC T(n) \NC = 4T(n/3) + n \NR
\NC \NC \le 4c(k-1)4^{k-1} + n \NR
\NC \NC = c(k-1)4^k + n \NR
\NC \NC = ck4^k + (n-c4^k) \NR
\NC \NC \le ck4^k \qquad \text{滿足 $n-c4^k\le 0$}\NR
\stopmathalignment\stopsplitformula
只需滿足 $n\ge c^{\log_{3}\frac{3}{4}}$。

{\externalfigure[output/e4_4_1-2]}
\stopANSWER

% c
\startitem
$T(n)=4T(n/2)+n$
\stopitem

\startANSWER
令 $T(n)\le c n^2 - n$:
\startsplitformula\startmathalignment
\NC T(n/2) \NC \le c(n/2)^2 - n/2 \NR
\NC T(n) \NC = 4T(n/2) + n \NR
\NC \NC \le 4c(n/2)^2 - 2n + n \NR
\NC \NC = cn^2 - n \NR
\stopmathalignment\stopsplitformula

{\externalfigure[output/e4_4_1-3]}
\stopANSWER

% d
\startitem
$T(n)=3T(n-1)+1$
\stopitem

\startANSWER
令 $T(n)\le c 3^n - 1$:
\startsplitformula\startmathalignment
\NC T(n-1) \NC \le c 3^{n-1} - 1 \NR
\NC T(n) \NC = 3T(n-1) + 1 \NR
\NC \NC \le 3c3^{n-1} - 3 + 1 \NR
\NC \NC = c3^n - 2 \NR
\NC \NC < c3^n - 1 \NR
\stopmathalignment\stopsplitformula

{\externalfigure[output/e4_4_1-4]}
\stopANSWER

\stopigBase
\stopEXERCISE

%e4.4-2
\startEXERCISE
用替代法證明遞迴式(4.15)的漸進下届 $L(n)=\Omega(n)$,
並證明 $L(n)=\Theta(n)$。
附遞迴式(4.15):
\startformula
L(n)=\startmathcases
\NC 1 \NC \qquad \text{如果 $n<n_0$,} \NR
\NC L(n/3) + L(2n/3) \NC \qquad \text{如果 $n\ge n_0$。} \NR
\stopmathcases
\stopformula
\stopEXERCISE
\startANSWER
令 $T(n)\ge cn$,其中 $c>0$,則:
\startsplitformula\startmathalignment
\NC T(n/3) \NC \ge cn/3 \NR
\NC T(2n/3) \NC \ge 2cn/3 \NR
\NC T(n) \NC \ge cn/3 + 2cn/3 \NR
\NC \NC = cn \NR
\stopmathalignment\stopsplitformula
將 $\ge$ 換成 $\le$ 一樣成立,
因此 $L(n)=\Theta(N)$。
\stopANSWER

%e4.4-3
\startEXERCISE
用替代法證明遞迴式(4.14)的解為 $T(n)=\Omega(n\lg n)$,
並證明 $T(n)=\Theta(n\lg n)$。
附遞迴式(4.14):
\startformula
T(n)=T(n/3)+T(2n/3)+\Theta(n)
\stopformula
\stopEXERCISE
\startANSWER
令 $c_1 n\lg n \le T(n) \le c_2 n\lg n$,則:
\startsplitformula\startmathalignment[n=3,align={right,middle,left}]
\NC \frac{c_1 n}{3}\lg \frac{n}{3} \le \NC T(n/3) \NC \le \frac{c_2 n}{3}\lg \frac{n}{3} \NR
\NC \frac{2c_1 n}{3}\lg \frac{2n}{3} \le \NC T(2n/3) \NC \le \frac{2c_2 n}{3}\lg \frac{2n}{3} \NR
\NC c_1 n\lg n + (\frac{2}{3}-\lg 3)c_1 n + \Theta(n) \le
	\NC T(n)
	\NC \le c_2 n\lg n + (\frac{2}{3}-\lg 3)c_2 n + \Theta(n) \NR
\stopmathalignment\stopsplitformula
令 $c_3 n\le \Theta(n)\le c_4 n$,則:
\startsplitformula\startmathalignment
\NC (\frac{2}{3}-\lg 3)c_1 + c_3 \NC > 0 \NR
\NC (\frac{2}{3}-\lg 3)c_2 + c_4 \NC > 0 \NR
\stopmathalignment\stopsplitformula
求解得:
\startsplitformula\startmathalignment
\NC c_1 \NC \le \frac{c_3}{\lg 3 - \frac{2}{3}} \NR
\NC c_2 \NC \ge \frac{c_4}{\lg 3 - \frac{2}{3}} \NR
\stopmathalignment\stopsplitformula
\stopANSWER

%e4.4-4
\startEXERCISE[exercise:partition_alpha]
有遞迴式 $T(n) = T(\alpha n) + T((1-\alpha)n) + \Theta n$,
其中 $\alpha$ 是常數且滿足 $0 < \alpha < 1$,
給出此遞迴式的解並利用遞迴樹進行驗證。
\stopEXERCISE
\startANSWER
我們可以假設 \m{\alpha \le 1/2},因爲我們可以讓 \m{\beta = 1 - \alpha} 來代替 \m{\alpha}。

由此,樹的深度爲 \m{\log_{1/\alpha}n},每一層爲 \m{cn}。
葉子節點不明朗,但我們可以假設爲 \m{\Theta(n)}。
\startsplitformula\startmathalignment
\NC T(n) \NC = \sum_{i=0}^{\log_{1/\alpha}n}cn + \Theta(n) \NR
\NC      \NC = cn\log_{1/\alpha}n + \Theta(n) \NR
\NC      \NC = \Theta(n\lg{n}) \NR
\stopmathalignment\stopsplitformula

另一種方式,設 \m{\beta = 1 - \alpha},則:
\startsplitformula\startmathalignment[n=3]
\NC T(n) \NC = \NC T(\alpha n) + T(\beta n) + cn \NR
\NC      \NC = \NC T(\alpha^2 n) + 2T(\alpha \beta n) + T(\beta^2 n) + cn + c \alpha n  + c \beta n \NR
\NC      \NC = \NC T(\alpha^2 n) + 2T(\alpha \beta n) + T(\beta^2 n) + 2cn \NR
\NC      \NC = \NC T(\alpha^3 n) + T(\alpha^2 \beta n) + c\alpha^2 n +
               2T(\alpha^2 \beta n) + 2T(\alpha \beta^2 n) \NR
\NC      \NC   \NC + 2c\alpha\beta n + T(\alpha \beta^2 n) + T(\beta ^ 3 n) + c\beta ^ 2 n + 2cn \NR
\NC      \NC = \NC T(\alpha^3 n) + 3T(\alpha^2 \beta n) + 3T(\alpha \beta^2 n) + T(\beta^3 n) \NR
\NC      \NC   \NC + c \alpha^2 n + 2c \alpha \beta n + c \beta ^ 2 n + 2cn \NR
\NC      \NC = \NC T(\alpha^3 n) + 3T(\alpha^2 \beta n) + 3T(\alpha \beta^2 n) + T(\beta^3 n) + 3cn \NR
\NC      \NC = \NC \ldots \NR
\stopmathalignment\stopsplitformula

一直到 \m{\alpha^k n \le 1},最後得到 \m{T(n) = O(1) + ckn}。
\startsplitformula\startmathalignment
\NC \NC \alpha^k = \frac{1}{n} \NR
\NC \Rightarrow \NC \log{\alpha^k} = \log\frac{1}{n} \NR
\NC \Rightarrow \NC k\log\alpha = - \log{n} \NR
\NC \Rightarrow \NC k = \frac{-\log{n}}{\log\alpha} = \frac{\log{n}}{\log(1/\alpha)} = \log_{1/\alpha}n \NR
\stopmathalignment\stopsplitformula

用代入法進行驗證。猜測 \m{T(n) \le dn\lg{n}}:
\startsplitformula\startmathalignment
\NC T(n) \NC \le d \alpha n \lg(\alpha n) + c \beta n \lg(\beta n) + cn \NR
\NC      \NC \le d \alpha n \lg{n} + d \beta n \lg{n} + d \alpha n \lg\alpha + d \beta n \lg\beta + cn \NR
\NC      \NC \le d n \lg{n} + \big(d (\alpha \lg\alpha + \beta \lg\beta) + c\big)n
                \qquad (d(\alpha\lg\alpha + \beta\lg\beta) + c \le 0)\NR
\NC      \NC \le d n \lg{n} \NR
\stopmathalignment\stopsplitformula

由於 \m{1/2 \le \alpha < 1} 並且 \m{0 < 1 - \alpha \le 1/2},
有 \m{\lg\alpha < 0} 以及 \m{\lg(1-\alpha)<0}。
因此 \m{\alpha\lg\alpha + \beta\lg\beta < 0},因此:
\startformula
d \ge -\frac{c}{\alpha\lg\alpha + (1-\alpha)\lg(1-\alpha)}
\stopformula
不等式右側爲一個正常數,只要 \m{d} 滿足此不等式, \m{T(n) \le dn\lg{n}} 就成立。

同理若要 \m{T(n) \ge dn\lg{n}},只需 \m{d(\alpha\lg\alpha + \beta\lg\beta) + c \ge 0} 即可,
即:
\startformula
0 \le d \le -\frac{c}{\alpha\lg\alpha + (1-\alpha)\lg(1-\alpha)}
\stopformula

因此 \m{T(n) = \Theta(n\lg n)}。
\stopANSWER

\stopsection

\startsection[
  title={Analysis of quicksort}
]

%e7.4-1
\startEXERCISE
證明遞迴式:
\startformula
T(n) = \max\{T(q) + T(n-q-1): 0 \le q \le n-1\} + \Theta(n)
\stopformula
的下界爲 $T(n)=\Omega(n^2)$。
\stopEXERCISE

\startANSWER
猜測 $T(n)\ge cn^2-2n$:
\startsplitformula\startmathalignment
\NC T(n) \NC=   \max_{0 \le q \le n-1} (T(q) + T(n-q-1)) + \Theta(n) \NR
\NC      \NC\ge \max_{0 \le q \le n-1} (cq^2 - 2q + c(n-q-1)^2 - 2n - 2q -1) + \Theta(n) \NR
\NC      \NC\ge c\max_{0 \le q \le n-1} (q^2 + (n-q-1)^2 - (2n + 4q + 1)/c) + \Theta(n) \NR
\NC      \NC\ge cn^2 - c(2n-1) + \Theta(n) \NR
\NC      \NC\ge cn^2 - 2cn + 2c \qquad (c \le 1) \NR
\NC      \NC\ge cn^2 - 2n \NR
\stopmathalignment\stopsplitformula
\stopANSWER

%e7.4-2
\startEXERCISE
證明:在最好情況下,快速排序的運行時間爲 $\Omega(n\lg{n})$。
\stopEXERCISE

\startANSWER
最好情況下運行時間爲:
\startformula
T(n)=2T(n/2)+\Theta(n)
\stopformula
由主定理解得 $T(n)=\Theta(n\lg{n})$。
\stopANSWER

%e7.4-3
\startEXERCISE
證明:在 $q=0, 1, \ldots, n-1$ 區間內,當 $q=0$ 或 $q=n-1$ 時,
 $q^2+(n-q-1)^2$ 取得最大值。
\stopEXERCISE

\startANSWER
\startsplitformula\startmathalignment
\NC f(q)  \NC= q^2 + (n - q - 1)^2 \NR
\NC f'(q) \NC= 2q - 2(n - q - 1) = 4q - 2n + 2 \NR
\NC f''(q)\NC= 4 \NR
\stopmathalignment\stopsplitformula
當 $q=\frac{1}{2}n-\frac{1}{4}$ 時, $f'(q)=0$。
$f'(q)$ 是連續的。
由於 $\forall q : f''(q) > 0$,所以在 $f'(q) =0$ 的左邊 $f'(q)<0$,
在 $f'(q) =0$ 的右邊 $f'(q)>0$。
即 $f(q)$ 在 $f'(q)=0$ 時取得最小值,左邊遞減,右邊遞增。
所以 $q$ 取兩端的值時, $f(q)$ 取得最大值。
\stopANSWER

%e7.4-4
\startEXERCISE
證明: \ALGO{RANDOMIZED-QUICKSORT} 期望運行時間時 $\Omega(n\lg{n})$。
\stopEXERCISE

\startANSWER
與比較次數期望值的推理類似,只是方向不同:
\startsplitformula\startmathalignment
\NC E[X] \NC=   \sum_{i=1}^{n-1} \sum_{j=i+1}^n \frac{2}{j-i+1} \NR
\NC      \NC=   \sum_{i=1}^{n-1} \sum_{k=1}^{n-i} \frac{2}{k + 1} \qquad (k \ge 1) \NR
\NC      \NC\ge \sum_{i=1}^{n-1} \sum_{k=1}^{n-i} \frac{2}{2k} \NR
\NC      \NC\ge \sum_{i=1}^{n-1} \Omega(\lg{n}) \NR
\NC      \NC=   \Omega(n\lg{n}) \NR
\stopmathalignment\stopsplitformula
\stopANSWER

%e7.4-5
\startEXERCISE
當輸入數據已經“幾乎有序”時,插入排序速度很快。
在實際應用中,我們可以利用這一特點提升快速排序的速度。
當子數列規模小於 $k$ 時,改用插入排序,而不是遞迴調用快速排序。
證明:這一排序算法的期望時間復雜度爲 $O(nk+n\lg(n/k))$。
分別從理論和實踐的角度說明應如何選擇 $k$。
\stopEXERCISE

\startANSWER
在所推薦的算法中,遞迴在 $\lg(n/k)$ 層處終止,
期望運行時間爲 $O(n\lg(n/k))$。
但是還有 $n/k$ 個(最大)長度爲 $k$,且沒有排序的子數列。
由插入排序的性質可知,他會將一個個子數列按順序排序。
即所有子數列的時間復雜度是相同的: $\frac{n}{k}O(k^2)=O(nk)$。

理論上,求解 $k$ 時我們可以忽略常數因子:
\startsplitformula\startmathalignment[n=1]
\NC n\lg{n} \ge nk + n\lg{n/k} \NR
\NC \Downarrow \NR
\NC \lg{n} \ge k + \lg{n} - \lg{k} \NR
\NC \Downarrow \NR
\NC \lg{k} \ge k \NR
\stopmathalignment\stopsplitformula
但這是不可能的。

而如果加上常數因子,則:
\startsplitformula\startmathalignment[n=1]
\NC c_qn\lg{n} \ge c_ink + c_qn\lg(n/k) \NR
\NC \Downarrow \NR
\NC c_q\lg{n} \ge c_ik + c_q\lg{n} - c_q\lg{k} \NR
\NC \Downarrow \NR
\NC \lg{k} \ge \frac{c_i}{c_q}k \NR
\stopmathalignment\stopsplitformula
這意味着有解。進一步,可能還要考慮低階項。

在實踐中,應當通過實驗來確定 $k$。
\stopANSWER

%e7.4-6
\startEXERCISE[exercise:random_three_median]\DIFFICULT
考慮修改 \ALGO{PARTITION}:
從子數列 $A[p:r]$ 中隨機選出三個元素,
並用這三個元素的中位數(即這三個元素按大小排在中間的值)對數列進行劃分。
求劃分比例比 $\alpha:(1-\alpha)$ 更差的近似概率,
其中 $0<\alpha<1/2$。
\stopEXERCISE

\startANSWER
簡單起見,假定可以重復選擇同一元素。

將元素按大小分成三部分,最小的 $\alpha n$ 個元素,最大的 $\alpha n$ 個元素,
還有大小處於中間的 $1-2\alpha n$ 個元素。
任選一元素位於最小那部分中的概率爲 $\alpha n/n = \alpha$。

如果恰好有兩個元素在最小的那部分中,其概率爲 $\alpha^2(1-\alpha)$。
如果三個元素均在最小的那部分中,其概率爲 $\alpha^3$。
前者共有三種可能(三個元素中任一元素不在最小那部分中),後者只有一種可能,
總的概率爲 $3\alpha^2(1-\alpha)+\alpha^3 = 3\alpha^2-2\alpha^3$。

最小那部分與最大那部分類似,將前面所列情況中最大、最小互換,其概率不變。
這樣這些情況總的概率爲 $(3\alpha^2-2\alpha^3) \times 2 = 6\alpha^2-4\alpha^3$。
這些情況所得劃分均比 $\alpha:(1-\alpha)$ 要糟糕。
\stopANSWER

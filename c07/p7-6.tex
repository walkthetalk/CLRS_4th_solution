%p7-6
\startPROBLEM
(Fuzzy sorting of intervals)
考慮這樣一種排序問題:我們無法準確知道待排序的數字是什麼,
但是對於每一個數,我們知道他屬於實數數軸上的某個區間。
也就是說,我們得到了 $n$ 個形如 $[a_i,b_i]$ 的閉區間,其中 $a_i\le b_i$。
我們的目標是實現這些區間的\emph{模糊排序},
即對 $j=1,2,\ldots,n$ 生成一個區間的排列 $\langle i_1,i_2,\ldots,i_n\rangle$,
且存在 $c_j \in [a_{i_j}, b_{i_j}]$,
滿足 $c_1\le c_2\le\ldots\le c_n$。
\startigBase[a]\startitem
爲 $n$ 個區間的模糊排序設計一個隨機算法。
你的算法應該具有算法的一般結構,
他可以對左邊端點(即 $a_i$ 的值)進行快速排序,
同時他也能利用區間的重疊性質來改善時間性能。
(當區間重疊越來越多的時候,區間的模糊排序問題變得越來越容易。
你的算法應充分利用這一重疊性質。)
\stopitem\stopigBase

\startANSWER
與\refproblem{quicksort_with_equal} 類似。
在(隨機)選擇主元區間後,檢查與其他區間是否重疊。
具體而言,累計主區間和其他區間的重疊區間。
然後用此區間進行比較,而不是用主元進行比較。
比較時,凡是包含此區間的其他區間都可以認爲是相等的。
在將主區間左邊的區間排列完畢後,可以將相等的區間全部放到主元的右邊。
與\refproblem{quicksort_with_equal} 類似,返回兩個點(區間)用來進行遞迴。

即使劃分時需要掃描三次子數列(最壞情況),整個算法仍然是線性的。
\stopANSWER

\startigBase[a,continue]\startitem
證明:在一般情況下,你的算法期望運行時間爲 $\Theta(n\lg{n})$。
但是,當所有的區間都有重疊的時候,算法的期望運行時間爲 $\Theta(n)$。
也就是說,存在一個值 $x$,對所有的 $i$,都有 $x\in [a_i,b_i]$。
你的算法不必顯式地檢查這種情況,
而是隨着重疊情況的增加,算法的性能自然地提高。
\stopitem\stopigBase

\startANSWER
\CLRSH{FUZZY-PARTITION(A, p, r)}
\startCLRSCODE
x = A[r]
// find the minimus interval for pivot
for j = p to r - 1
	if x.left <= A[j].right and A[j].right < x.right
		x.right = A[j].right
	if x.left < A[j].left and A[j].left <= x.right
		x.left = A[j].left

q = p - 1
t = q
for j = p to r - 1
	if A[j].right <= x.right
		if A[j].right = x.right
			q = q + 1
			\ALGO{SWAP(A[q],A[j])}
		t = t + 1
		\ALGO{SWAP(A[t],A[j])}
q = q + 1
t = t + 1
\ALGO{SWAP(A[t],A[r])}

return q, t
\stopCLRSCODE
\stopANSWER

\stopPROBLEM

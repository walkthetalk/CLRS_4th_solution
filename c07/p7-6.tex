%p7-6
\startPROBLEM
(Median-of-3 partition)
一種改進 \ALGO{RANDOMIZED-QUICKSORT} 的方法就是在劃分時,
從子數列中更細緻地選擇主元(而不是簡單的隨機選擇)。
通常用的是三數取中法:
從子數列中隨機選出三個元素,
取其中位數作爲主元(參見\inexercise[random_three_median])。
對於這個問題的分析,我們不妨假設數列 $A[1..n]$ 的元素是互異的且 $n\ge 3$。
我們用 $z_1,z_2,\ldots,z_n$ 表示已排序的 $A[p..r]$。
用三數取中法選擇主元 $x$,並定義 $p_i=\Pr\{x=z_i\}$。
% a
\startigBase[a]\startitem
對於 $i=2,3,\ldots,n-1$,
請給出 $p_i$ 的準確表達式,用 $n$ 和 $i$ 表示
(注意 $p_1=p_n=0$)。
\stopitem\stopigBase

\startANSWER
共有 $n!/(n-3)!$ 種三元素排列。如果中位數是第 $i$ 小元素,
則需要有一個小於他的元素,一個大於他的元素。
小於他的元素有 $i-1$ 個,大於他的有 $n-i$ 個。
這三個元素共有 $3!$ 種選取順序。因此:
\startformula
p_i = \frac{6(i-1)(n-i)}{n(n-1)(n-2)}
\stopformula
\stopANSWER

% b
\startigBase[a,continue]\startitem
與普通實現相比,在這種實現中,
所選主元滿足 $x=z_{\lfloor (n+1)/2 \rfloor}$
 (即 $A[p..r]$ 的中位數)
的概率增加了多少?
假設 $n\to\infty$,請給出這一概率比值的極限值。
\stopitem\stopigBase

\startANSWER
\startsplitformula\startmathalignment
\NC \NC \lim_{n \to \infty}\frac{6(i-1)(n-i)}{n(n-1)(n-2)}/\frac{1}{n} \NR
\NC = \NC \lim_{n \to \infty}\frac{6n(n/2 - 1)(n/2)}{n(n-1)(n-2)} \NR
\NC = \NC \lim_{n \to \infty}\frac{6(n^2 - 2n)}{4(n^2 - 3n + 2)} \NR
\NC = \NC \frac{6}{4} \NR
\stopmathalignment\stopsplitformula
概率變爲原來的 1.5 倍,變化不大。
\stopANSWER

% c
\startigBase[a,continue]\startitem
如果我們定義一個“好”的劃分意味着選擇 $x=z_i$ 作爲主元,
其中 $n/3\le i\le 2n/3$。
與普通實現相比,這種實現仲得到一個好的劃分的概率增加了多少?
(\hint 用積分來近似累加和)
\stopitem\stopigBase

\startANSWER
由\inexercise[const_part_probability]可知,
普通實現中得到“好”的劃分的概率爲 $1-2(1/3)=1/3$。
而三數取中法可將概率變爲:
\startsplitformula\startmathalignment
\NC   \NC \lim_{n \to \infty}\sum_{i=n/3}^{2n/3}\frac{6(i-1)(n-i)}{n(n-1)(n-2)} \NR
\NC = \NC \lim_{n \to \infty}\frac{6}{n(n-1)(n-2)}\sum_{i=n/3}^{2n/3}(i-1)(n-i) \NR
\NC = \NC \lim_{n \to \infty}\binom{n}{3}\int_{n/3}^{2n/3}(i-1)(n-i)\mathrm{d}i \NR
\NC   \NC  \qquad \left( \int(i-1)(n-i)\mathrm{d}i
                      = \frac{1}{6}(3ni^2 - 6ni - 2i^3 + 3i^2) \right) \NR
\NC = \NC \lim_{n \to \infty}\binom{n}{3}\frac{1}{6}\left[
          \frac{36}{27}n^3 - \frac{16}{27}n^3 + o(n^3) -
          \frac{9}{27}n^3 + \frac{2}{27}n^3 + o(n^3)
        \right] \NR
\NC = \NC \lim_{n \to \infty}\frac{1}{n(n-1)(n-2)} \frac{13}{27}(n^3 + o(n^3)) \NR
\NC = \NC \lim_{n \to \infty}\frac{13}{27}\frac{n^3 + o(n^3)}{n^3 + o(n^3)} \NR
\NC = \NC \frac{13}{27} \NR
\stopmathalignment\stopsplitformula

$n$ 趨近 $\infty$ 時,概率收斂爲 $13/27$,
是普通實現的 $\frac{13}{27} \div \frac{1}{3} = \frac{39}{27} \approx 1.\dot{4}$ 倍。
\stopANSWER

% d
\startigBase[a,continue]\startitem
證明:對快速排序而言,三數取中法只影響其時間複雜度 $\Omega(n\lg{n})$ 的常數項因子。
\stopitem\stopigBase

\startANSWER
如果新的方法總能得到“好”的劃分,則會減少運行時間。然而,他不能。
新方法只能保證無論怎樣劃分,都不會出現空的子數列,但是仍會出現 $1:(n-2)$ 的劃分。
他提升了“好”的劃分的概率,同時也增加了選擇主元的開銷,
但無法保證劃分的質量。
因此算法的時間複雜度仍然爲 $\Omega(n\lg{n})$ 和 $O(n^2)$。
\stopANSWER

\stopPROBLEM

%p7-2
\startPROBLEM[problem:quicksort_with_equal]
(Quicksort with equal element values)
在節 7.4.2 分析隨機化快速排序時,
我們假設輸入元素的值是互異的,
本題中,我們來看看如果這一假設不成立會出現什麼情況。
% a
\startigBase[a]
\item 如果所有輸入元素的值都相同,那麼隨機化快速排序的運行時間會是多少?
\stopigBase

\startANSWER
運行時間是 $\Theta(n^2)$。所有的劃分都將是 $(n-1):1$
 (參見\inexercise[same_partition])。
\stopANSWER

\startigBase[a,continue]
\item \ALGO{PARTITION} 返回索引 $q$,
使得 $A[p:q-1]$ 中的每個元素都小於或等於 $A[q]$,
而 $A[q+1:r]$ 中的每個元素都大於 $A[q]$。
修改 \ALGO{PARTITION} 構造一個新的 \ALGO{PARTITION'(A, p, r)},
他將 $A[p:r]$ 中的元素排序後,
返回值是兩個索引 $q$ 和 $t$,
其中 $p\le q\le t\le r$,且:
\startigBase
\item $A[q:t]$ 中的所有元素都相等;
\item $A[p:q-1]$ 中的所有元素都小於 $A[q]$;
\item $A[t+1:r]$ 中的所有元素都大於 $A[q]$。
\stopigBase
與 \ALGO{PARTITION} 類似, \ALGO{PARTITION'} 的時間復雜度是 $\Theta(r-p)$。
\stopigBase

\startANSWER
\CLRSH{PARTITION'(A, p, r)}
\startCLRSCODE
x = A[r]
q = p - 1
t = q
for j = p to r - 1
	if A[j] <= x
		if A[j] = x
			q = q + 1
			\ALGO{SWAP(A[q],A[j])}
		t = t + 1
		\ALGO{SWAP(A[t],A[j])}
q = q + 1
t = t + 1
\ALGO{SWAP(A[t],A[r])}

return q, t
\stopCLRSCODE
\stopANSWER

\startigBase[a,continue]
\item 修改 \ALGO{RANDOMIZED-QUICKSORT},改成調用 \ALGO{PARTITION'},
並重新命名爲 \ALGO{RANDOMIZED-QUICKSORT'}。
然後修改 \ALGO{QUICKSORT},改成調用 \ALGO{RANDOMIZED-PARTITION'},
並且只有分區內的元素不同時才做遞迴調用,
重新命名爲 \ALGO{QUICKSORT'(A, p, r)}。
\stopigBase

\startANSWER
\CLRSH{RANDOMIZED-QUICKSORT'(A, p, r)}
\startCLRSCODE
if p < r
	q, t = \ALGO{PARTITION'(A, p, r)}
	\ALGO{RANDOMIZED-QUICKSORT'(A, p, q - 1)}
	\ALGO{RANDOMIZED-QUICKSORT'(A, t + 1, r)}
\stopCLRSCODE

\CLRSH{QUICKSORT'(A, p, r)}
\startCLRSCODE
if p < r
	q, t = \ALGO{RANDOMIZED-PARTITION'(A, p, r)}
	\ALGO{QUICKSORT'(A, p, q - 1)}
	\ALGO{QUICKSORT'(A, t + 1, r)}
\stopCLRSCODE
\stopANSWER

\startigBase[a,continue]
\item 在 \ALGO{QUICKSORT'} 中,如何改變節 7.4.2 中的分析方法,
才能避免所有元素互異這一假設?
\stopigBase

\startANSWER
略。
\stopANSWER
\stopPROBLEM

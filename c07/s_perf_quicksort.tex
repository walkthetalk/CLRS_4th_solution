\startsection[
  title=Performance of quicksort
]

%e7.2-1
\startEXERCISE
用代入法證明:遞迴式 $T(n)=T(n-1)+\Theta(n)$ 的解爲 $T(n)=\Theta(n^2)$。
\stopEXERCISE

\startANSWER
令 $\Theta(n)$ 爲 $c_2 n$,猜測 $T(n)\le c_1 n^2$:
\startsplitformula\startmathalignment
\NC T(n) \NC=   T(n-1) + c_2n \NR
\NC      \NC\le c_1(n-1)^2 + c_2n \NR
\NC      \NC=   c_1n^2 - 2c_1n + c_1 + c_2n \qquad (2c_1 > c_2, n \ge c_1/(2c_1 - c_2))\NR
\NC      \NC\le c_1n^2 \NR
\stopmathalignment\stopsplitformula
\stopANSWER

%e7.2-2
\startEXERCISE
當數列 $A$ 中的元素均相同時, \ALGO{QUICKSORT} 的時間復雜度是什麼?
\stopEXERCISE

\startANSWER
時間復雜度爲 $\Theta(n^2)$,
因爲 \ALGO{PARTITION} 時總有一邊爲空(參見\inexercise[same_partition])。
\stopANSWER

%e7.2-3
\startEXERCISE
證明:當數列 $A$ 包含的元素不同,
且按降序排列時, \ALGO{QUICKSORT} 的時間復雜度爲 $\Theta(n^2)$。
\stopEXERCISE

\startANSWER
\ALGO{PARTITION} 總是返回 $p$,即總有一邊是空。

遞迴式仍爲 $T(n)=T(n-1)+\Theta(n)$。
(即使 \emph{if} 塊內執行不到, \emph{for} 循環的執行次數不變,
仍是 $\Theta(n)$。)
\stopANSWER

%e7.2-4
\startEXERCISE
銀行一般會按照交易時間記錄賬戶的交易情況。
但是,很多人希望收到的銀行賬單是按支票號碼排列的。
這是因爲,人們通常都是按照支票號碼的順序開支票,
而商人也通常都是根據支票號碼的順序兌付支票。
這一問題是將按交易時間排序的支票改成按支票號碼排序,
實質上是對幾乎有序的輸入序列進行排序。
請證明:在這個問題上, \ALGO{INSERTION-SORT} 的性能往往要優於 \ALGO{QUICKSORT}。
\stopEXERCISE

\startANSWER
\ALGO{INSERTION-SORT} 的時間復雜度爲 $\Theta(n+d)$,其中 $d$ 爲逆序對的數目。
在本例中, $d$ 的值幾乎爲 $0$,因此插入排序的時間幾乎是線性的。

而在基本有序的情況下,大部分 \ALGO{PARTITION} 都會出現一邊爲空的情況。
從而使得 \ALGO{QUICKSORT} 所用時間復雜度接近 $\Theta(n^2)$。
\stopANSWER

%e7.2-5
\startEXERCISE
假設快速排序的每一層所做的劃分比例都是 $\alpha : \beta$,
其中 $\alpha+\beta=1$,且 $0<\alpha\le \beta < 1$ 且均爲常數。
證明:在相應的遞迴樹中,葉子節點的最小深度大約是 $\log_{1/\alpha}{n}$,
最大深度大約是 $\log_{1/\beta}{n}$(無需考慮整數舍入問題)。
\stopEXERCISE

\startANSWER
令樹的最小深度是 $x$,則 $n\alpha^x\approx 1$,解得 $x\approx\log_{1/\alpha}{n}$。

類似,令最大深度爲 $y$,則 $n\beta^y\approx 1$,解得 $y\approx\log_{1/\beta}{n}$。
\stopANSWER

%e7.2-6
\startEXERCISE[exercise:const_part_probability]
證明:有一隨機輸入數列,其元素各不相同,
對於任何常數 $0<\alpha\le 1/2$,
 \ALGO{PARTITION} 產生比 $1-\alpha : \alpha$ 更平衡的劃分的概率約爲 $1-2\alpha$。
\stopEXERCISE

\startANSWER
將元素劃分成三部分,最小的 $\alpha n$ 個元素、最大的 $\alpha n$ 個元素,
還有 $(1-2\alpha)n$ 個中間值元素。
只有當 \ALGO{PARTITION} 所選的主元處於最後一個集合中時,所的劃分才比 $1-\alpha : \alpha$ 更平衡。
其概率爲 $(1-2\alpha)n / n = (1-2\alpha)$。
\stopANSWER

\stopsection

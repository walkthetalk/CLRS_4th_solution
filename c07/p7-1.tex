%p7-1
\startPROBLEM
(Hoare partition correctness)
本章中的 \ALGO{PARTITION} 並不是其最初版本。
下面給出的是原始的劃分算法,由 C.R.Hoare 設計:

\CLRSH{HOARE-PARTITION(A, p, r)}
\startCLRSCODE
x = A[p]
i = p - 1
j = r + 1
while TRUE
	repeat
		j = j - 1
	until A[j] <= x
	repeat
		i = i + 1
	until A[j] >= x
	if i < j
		\ALGO{SWAP(A[i],A[j])}
	else
		return j
\stopCLRSCODE

% a
\startigBase[a]
\item 試說明在數列 $A = \langle 13, 19, 9, 5, 12, 8, 7, 4, 11, 2, 6, 21 \rangle$ 上
 \ALGO{HOARE-PARTITION} 的操作過程,
並說明在每一次執行第 4~13 行 \emph{while} 循環后
數列元素和索引 $i$、 $j$ 的值。
\stopigBase

\startANSWER
輔助變量的值: $x = 13$, $j = 9$, $i = 10$。

{\externalfigure[output/p7_1_a-1]}
{\externalfigure[output/p7_1_a-2]}
{\externalfigure[output/p7_1_a-3]}
\stopANSWER

% b
\startigBase[a,continue]\startitem
如果 $A[p:r]$ 中所有元素均相同,
節 7.1 中的 \ALGO{PARTITION} 與 \ALGO{HOARE-PARTITION} 會有什麼區別?
\stopitem\stopigBase

\startANSWER
\ALGO{PARTITION} 會執行 $r-p+1$ 次交換,返回的是 $r$。

\ALGO{HOARE-PARTITION} 不會執行交換,返回的是 $p$。
\stopANSWER

後續的三個問題要求讀者仔細論證 \ALGO{HOARE-PARTITION} 的正確性。
假設子數列 $A[p..r]$ 至少包含 2 個元素,證明下列問題:
% c
\startigBase[a,continue]
\item 下標 $i$ 和 $j$ 可以保證只能訪問子數列 $A[p..r]$,
而無法訪問 $A$ 中其他元素。
\stopigBase

\startANSWER
第一次比較時, $i < j$, $i=p$ 且 $j\ge p$ (因爲 $A[p]=x$)。
如果 $i=j$,算法終止,不會訪問任何“無效”元素。
如果 $i<j$,下一次循環中 $i$ 和 $j$ 都仍有效($i\le r$ 且 $j\ge p$)。
如果某一個下標到達數列兩端,則 $i$ 不再小於等於 $j$。
\stopANSWER

% d
\startigBase[a,continue]
\item 當 \ALGO{HOARE-PARTITION} 結束時,他返回的 $j$ 滿足 $p\le j < r$。
\stopigBase

\startANSWER
第一次迭代時,如果 $A[p]$ 是最大元素,則 $i=p$ 且 $j=p<r$;
否則會交換 $A[p]$ 和 $A[j]$,其中 $j\le r$。
算法此時不會終止,下一次迭代時, $j$ 會減小,因此 $p\le j < r$。
\stopANSWER

% e
\startigBase[a,continue]
\item 當 \ALGO{HOARE-PARTITION} 結束時, $A[p..j]$ 中
的所有元素都小於或等於 $A[j+1..r]$ 中的元素。
\stopigBase

\startANSWER
比較 $i$ 和 $j$ 之前, $A[p..i-1]$ 中所有元素都小於等於 $x$;
而 $A[j+1..r]$ 中所有元素都大於等於 $x$。

\emph{初始化:}兩個 \emph{repeat} 塊將建立此條件;

\emph{保持:}通過交換 $A[i]$ 和 $A[j]$,使得 $A[p..i]\le x$ 且 $A[j..r]\ge x$。
增大 $i$ 和減小 $j$ 都將維持不變式;

\emph{終止:}當 $i\ge j$ 時循環終止。不變式仍然得以保持。
\stopANSWER

在\insection[desc_quicksort] 的 \ALGO{PARTITION} 過程中,
主元(原來存儲在 $A[r]$ 中)是不在兩個分區之內的。
與之對應,在 \ALGO{HOARE-PARTITION} 中,
主元(原來在 $A[p]$ 中)是在 $A[p..j]$ 或 $A[j+1..r]$ 中的。
因爲 $p\le j<r$,所以這一劃分總是非平凡的。
% f
\startigBase[a,continue]
\item 利用 \ALGO{HOARE-PARTITION},重寫 \ALGO{QUICKSORT} 算法。
\stopigBase

\startANSWER
\CLRSH{QUICKSORT''(A, p, r)}
\startCLRSCODE
if p < r
	q = \ALGO{HOARE-PARTITION(A, p, r)}
	\ALGO{QUICKSORT(A, p, q)}
	\ALGO{QUICKSORT(A, q + 1, r)}
\stopCLRSCODE
\stopANSWER
\stopPROBLEM

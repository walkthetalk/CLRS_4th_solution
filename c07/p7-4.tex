%p7-4
\startPROBLEM
(Stack depth for quicksort)
\insection[desc_quicksort] 中的 \ALGO{QUICKSORT} 算法包含了兩個對其自身的遞迴調用。
在調用 \ALGO{PARTITION} 後, \ALGO{QUICKSORT} 分別遞迴調用了左邊的子數列和右邊的子數列。
 \ALGO{QUICKSORT} 中的第二個遞迴調用並不是必須的。
我們可以用一個循環控制結構來代替他。
這一技術稱爲\emph{尾遞迴},好的編譯器都提供這一功能。
考慮廈門這個版本的快速排序,他模擬了尾遞迴的情況:

\CLRSH{TAIL-RECURSIVE-QUICKSORT(A, p, r)}
\startCLRSCODE
while p < r
	// Partition and sort left subarray.
	q = \ALGO{PARTITION(A, p, r)}
	\ALGO{TAIL-RECURSIVE-QUICKSORT(A, p, q-1)}
	p = q + 1
\stopCLRSCODE

\startigBase[a]
\item 證明: \ALGO{TAIL-RECURSIVE-QUICKSORT(A, 1, A.length)} 能正確地對數列 $A$ 進行排序。
\stopigBase

\startANSWER
原 \ALGO{QUICKSORT} 將數列劃分成兩部分,並對其分別遞迴調用自身。
而這個版本所做的事情是一樣的,只是方式不同,不是調用 \ALGO{TAIL-RECURSIVE-QUICKSORT},
而是改變 $p$,繼續循環。
\stopANSWER

編譯器通常使用\emph{棧}來存儲遞迴過程中的相關信息,
包括每一次遞迴調用的參數等。
最新調用的信息存在棧的頂部;
當他結束時,其信息則被\emph{彈出}。
因爲我們假設數列參數是用指針來指示的,所以每次過程調用只需要 $O(1)$ 的棧空間。
\emph{棧深度}是在一次計算仲會用到的棧空間的最大值。
\startigBase[a,continue]
\startitem
請描述一個場景,使得 \ALGO{TAIL-RECURSIVE-QUICKSORT} 作用在規模爲 $n$ 的數列上時,
棧深度爲 $\Theta(n)$。
\stopitem

\startANSWER
如果 \ALGO{PARTITION} 返回的一直是 $r$,
則棧深度爲 $\Theta(n)$。
在數列已經排好序時就會出現這種情況。
\stopANSWER

\startitem
修改 \ALGO{TAIL-RECURSIVE-QUICKSORT} 的代碼,
使其最壞請看下棧深度爲 $\Theta(\lg{n})$,
並且能夠保持 $O(n\lg{n})$ 的期望時間複雜度。
\stopitem

\startANSWER
如果每次針對較大的子數列進行尾遞迴,則能滿足要求。

\CLRSH{TAIL-RECURSIVE-QUICKSORT‘(A, p, r)}
\startCLRSCODE
while p < r
	// Partition and sort left subarray.
	q = \ALGO{PARTITION(A, p, r)}
	if q < (p + r) / 2
		\ALGO{TAIL-RECURSIVE-QUICKSORT’(A, p, q-1)}
		p = q + 1
	else
		\ALGO{TAIL-RECURSIVE-QUICKSORT’(A, q+1, r)}
		r = q - 1
\stopCLRSCODE
\stopANSWER
\stopigBase
\stopPROBLEM

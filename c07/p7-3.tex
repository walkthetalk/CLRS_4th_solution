%p7-3
\startPROBLEM[problem:alt_quicksort_analysis]
(Alternative quicksort analysis)
在隨機化快速排序中,還有一種性能分析方法,
這一方法關注於每一次單獨遞迴調用的期望運行時間,而不是比較的次數。
\startigBase[a]
\item 證明:給定一個大小爲 $n$ 的數列,任何特定元素被選爲主元的概率爲 $1/n$。
利用這一點來定義指示器隨機變量 $X_i=I\{\text{第 $i$ 小的元素被選爲主元}\}$,
那麼 $E[X_i]$ 是什麼?
\stopigBase

\startANSWER
$E[X_i] = 1/n$。
\stopANSWER

\startigBase[a,continue]\startitem
令 $T(n)$ 表示快速排序在大小爲 $n$ 的數列上的運行時間,證明:
\setnumber[formula]{5}
\placeformula
\startformula
E[T(n)] = E\left[\sum_{q=1}^nX_q(T(q-1) + T(n-q) + \Theta(n))\right]
\stopformula
\stopitem\stopigBase

\startANSWER
令第 $q$ 小元素爲主元,則有 $n$ 種可能,每種概率爲 $X_q$。
每種都會將數列劃分成 $(q-1):(n-q)$ 兩部分。
\stopANSWER

\startigBase[a,continue]\startitem
證明公式可以重寫爲:
\placeformula[formula:7_6]
\startformula
E[T(n)] = \frac{2}{n}\sum_{q=2}^{n-1}E[T(q)] + \Theta(n)
\stopformula
\stopitem\stopigBase

\startANSWER
\startsplitformula\startmathalignment
\NC E[T(n)] \NC= E\left[\sum_{q=1}^nX_q(T(q-1) + T(n-q) + \Theta(n))\right] \NR
\NC         \NC= \sum_{q=1}^n\frac{1}{n}(E[T(q-1)] + E[T(n-q)] + \Theta(n)) \NR
\NC         \NC= \frac{1}{n}\sum_{q=1}^nE[T(q-1)]
             + \frac{1}{n}\sum_{q=1}^nE[T(n - q)]
             + \frac{1}{n}\sum_{q=1}^n\Theta(n) \NR
\NC         \NC= \frac{1}{n}\sum_{q=0}^{n-1}E[T(q)]
             + \frac{1}{n}\sum_{q=0}^{n-1}E[T(n - q - 1)]
             + \Theta(n) \NR
\NC         \NC= \frac{1}{n}\sum_{q=0}^{n-1}E[T(q)]
             + \frac{1}{n}\sum_{q=0}^{n-1}E[T(q)]
             + \Theta(n) \NR
\NC         \NC= \frac{2}{n}\sum_{q=0}^{n-1}E[T(q)] + \Theta(n) \NR
\NC         \NC= \frac{2}{n}\sum_{q=2}^{n-1}E[T(q)]
             + \frac{2E[T(0)]}{n}
             + \frac{2E[T(1)]}{n}
             + \Theta(n) \NR
\NC         \NC= \frac{2}{n}\sum_{q=2}^{n-1}E[T(q)] + \Theta(n)
\stopmathalignment\stopsplitformula
\stopANSWER

\startigBase[a,continue]\startitem
證明:
\placeformula[formula:7_7]
\startformula
\sum_{k=2}^{n-1}k\lg{k} \le \frac{1}{2}n^2\lg{n} - \frac{1}{8}n^2
\stopformula
(\hint 以將其分成兩部分,一部分是 $k=2,3,\ldots,\lceil n/2\right\rceil-1$,
另一部分是 $k=\lceil n/2 \right\rceil,\ldots,n-1$。)
\stopitem\stopigBase

\startANSWER
\startsplitformula\startmathalignment
\NC \sum_{k=2}^{n-1}k\lg{k}
   \NC=   \sum_{k=2}^{\lceil n/2 \right\rceil - 1}k\lg{k} + \sum_{k=\left\lceil n/2 \right\rceil}^{n - 1}k\lg{k} \NR
\NC\NC\le \sum_{k=2}^{n/2}k\lg{k} + \sum_{k=n/2 + 1}^{n}k\lg{k} \NR
\NC\NC\le \sum_{k=2}^{n/2}k\lg(n/2) + \sum_{k=n/2 + 1}^{n}k\lg{n} \NR
\NC\NC=   \lg(n/2)\sum_{k=2}^{n/2}k\ + \lg{n}\sum_{k=n/2 + 1}^{n}k \NR
\NC\NC=   (\lg{n} - \lg{2})\left(\frac{(n/2)(n/2 + 1)}{2}\right) +
          \lg{n}\left(\frac{n(n+1)}{2} - \frac{(n/2)(n/2 + 1)}{2}\right) \NR
\NC\NC=   \lg{n}\frac{n(n+1)}{2} - \frac{(n/2)(n/2 + 1)}{2} \NR
\NC\NC=   \frac{1}{2}\lg{n}(n^2 + 2n + 1) - \frac{1}{8}(n^2 + 2n + 1/8) \NR
\NC\NC=   \frac{1}{2}n^2\lg{n} - \frac{1}{8}n^2 - \frac{8n\lg{n} + 4\lg{n} - 2n - 1/8}{8} \NR
\NC\NC\le \frac{1}{2}n^2\lg{n} - \frac{1}{8}n^2 \NR
\stopmathalignment\stopsplitformula
\stopANSWER

\startigBase[a,continue]\startitem
利用\refformula{7_7} 給出的界證明:
\refformula{7_6} 中的遞迴式有解 $E[T(n)]=\Theta(n\lg{n})$。
(\hint 用代入法,證明對於某個正常數 $a$ 和足夠大的 $n$,有 $E[T(n)]\le an\lg{n}$。)
\stopitem\stopigBase

\startANSWER
猜測 $E[T(n)] \le an\lg{n}$:
\startsplitformula\startmathalignment
\NC E[T(n)]
   \NC=   \frac{2}{n}\sum_{q=2}^{n-1}E[T(q)] + \Theta(n) \NR
\NC\NC\le \frac{2}{n}\sum_{q=2}^{n-1}aq\lg{q} + \Theta(n) \qquad \text{(根據猜測)} \NR
\NC\NC\le \frac{2a}{n}\left(\frac{1}{2}n^2\lg{n} - \frac{1}{8}n^2\right)
                 + \Theta(n) \qquad \text{(\refformula{7_7})} \NR
\NC\NC=   an\lg{n} - \frac{a}{4}n + \Theta(n) \qquad \text{(根據 $\Theta$ 記號的定義)} \NR
\NC\NC\le an\lg{n} \NR
\stopmathalignment\stopsplitformula
\stopANSWER

\stopPROBLEM

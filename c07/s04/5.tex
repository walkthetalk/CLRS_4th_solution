\startEXERCISE
當輸入數據已經“幾乎有序”時,插入排序速度很快。
在實際應用中,我們可以利用這一特點提升快速排序的速度。
當子數列規模小於 $k$ 時,改用插入排序,而不是遞迴調用快速排序。
證明:這一排序算法的期望時間復雜度爲 $O(nk+n\lg(n/k))$。
分別從理論和實踐的角度說明應如何選擇 $k$。
\stopEXERCISE

\startANSWER
在所推薦的算法中,遞迴在 $\lg(n/k)$ 層處終止,
期望運行時間爲 $O(n\lg(n/k))$。
但是還有 $n/k$ 個(最大)長度爲 $k$,且沒有排序的子數列。
由插入排序的性質可知,他會將一個個子數列按順序排序。
即所有子數列的時間復雜度是相同的: $\frac{n}{k}O(k^2)=O(nk)$。

理論上,求解 $k$ 時我們可以忽略常數因子:
\startsplitformula\startmathalignment[n=1]
\NC n\lg{n} \ge nk + n\lg{n/k} \NR
\NC \Downarrow \NR
\NC \lg{n} \ge k + \lg{n} - \lg{k} \NR
\NC \Downarrow \NR
\NC \lg{k} \ge k \NR
\stopmathalignment\stopsplitformula
但這是不可能的。

而如果加上常數因子,則:
\startsplitformula\startmathalignment[n=1]
\NC c_qn\lg{n} \ge c_ink + c_qn\lg(n/k) \NR
\NC \Downarrow \NR
\NC c_q\lg{n} \ge c_ik + c_q\lg{n} - c_q\lg{k} \NR
\NC \Downarrow \NR
\NC \lg{k} \ge \frac{c_i}{c_q}k \NR
\stopmathalignment\stopsplitformula
這意味着有解。進一步,可能還要考慮低階項。

在實踐中,應當通過實驗來確定 $k$。
\stopANSWER

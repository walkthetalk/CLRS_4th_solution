%p7-3
\startPROBLEM[problem:alt_quicksort_analysis]
(Alternative quicksort analysis)
在分析隨機化快速排序的性能時,還有一種方法
關注的是每次遞迴調用的期望運行時間,而不是比較的次數。
% a
\startigBase[a]
\item 證明:有規模爲 $n$ 的數列,任一元素被選爲主元的概率均爲 $1/n$。
利用這一點來定義指示器隨機變量 $X_i=I\{\text{第 $i$ 小的元素被選爲主元}\}$,
那麼 $E[X_i]$ 是什麼?
\stopigBase

\startANSWER
$E[X_i] = 1/n$。
\stopANSWER

% b
\startigBase[a,continue]\startitem
令 $T(n)$ 爲指示器隨機變量,
表示快速排序在大小爲 $n$ 的數列上的運行時間,證明:
\setnumber[formula][1]
\placeformula[formula:7_2]
\startformula
E[T(n)] = E\left[\sum_{q=1}^n X_q \left(T(q-1) + T(n-q) + \Theta(n)\right)\right]
\stopformula
\stopitem\stopigBase

\startANSWER
令第 $q$ 小元素爲主元,則有 $n$ 種可能,每種概率爲 $X_q$。
每種都會將數列劃分成 $(q-1):(n-q)$ 兩部分。
\stopANSWER

% c
\startigBase[a,continue]\startitem
證明\informula[7_2]可以重寫爲:
\placeformula[formula:7_3]
\startformula
E[T(n)] = \frac{2}{n}\sum_{q=2}^{n-1}E[T(q)] + \Theta(n)
\stopformula
\stopitem\stopigBase

\startANSWER
\startsplitformula\startmathalignment
\NC E[T(n)] \NC= E\left[\sum_{q=1}^nX_q(T(q-1) + T(n-q) + \Theta(n))\right] \NR
\NC         \NC= \sum_{q=1}^n\frac{1}{n}(E[T(q-1)] + E[T(n-q)] + \Theta(n)) \NR
\NC         \NC= \frac{1}{n}\sum_{q=1}^nE[T(q-1)]
             + \frac{1}{n}\sum_{q=1}^nE[T(n - q)]
             + \frac{1}{n}\sum_{q=1}^n\Theta(n) \NR
\NC         \NC= \frac{1}{n}\sum_{q=0}^{n-1}E[T(q)]
             + \frac{1}{n}\sum_{q=0}^{n-1}E[T(n - q - 1)]
             + \Theta(n) \NR
\NC         \NC= \frac{1}{n}\sum_{q=0}^{n-1}E[T(q)]
             + \frac{1}{n}\sum_{q=0}^{n-1}E[T(q)]
             + \Theta(n) \NR
\NC         \NC= \frac{2}{n}\sum_{q=0}^{n-1}E[T(q)] + \Theta(n) \NR
\NC         \NC= \frac{2}{n}\sum_{q=2}^{n-1}E[T(q)]
             + \frac{2E[T(0)]}{n}
             + \frac{2E[T(1)]}{n}
             + \Theta(n) \NR
\NC         \NC= \frac{2}{n}\sum_{q=2}^{n-1}E[T(q)] + \Theta(n)
\stopmathalignment\stopsplitformula
\stopANSWER

% d
\startigBase[a,continue]\startitem
證明下式對於 $n\ge2$ 成立:
\placeformula[formula:7_4]
\startformula
\sum_{q=1}^{n-1}q\lg{q} \le \frac{n^2}{2}\lg{n} - \frac{n^2}{8}
\stopformula
(\hint 將其一分爲二,一爲 $q=1,2,\ldots,\lceil n/2\right\rceil-1$,
一爲 $q = \lceil n/2 \right\rceil, \ldots, n-1$。)
\stopitem\stopigBase

\startANSWER
\startsplitformula\startmathalignment
\NC \sum_{q=1}^{n-1}q\lg{q}
   \NC=   \sum_{q=1}^{\lceil n/2 \right\rceil - 1}q\lg{q} + \sum_{q=\left\lceil n/2 \right\rceil}^{n - 1}q\lg{q} \NR
\NC\NC=   \sum_{q=1}^{\lceil n/2 \right\rceil - 1}q\lg\frac{n}{2} + \sum_{q=\left\lceil n/2 \right\rceil}^{n - 1}q\lg{q} \NR
\NC\NC\le \sum_{q=1}^{\lceil n/2 \right\rceil - 1}q(\lg{n}-1) + \sum_{q=\left\lceil n/2 \right\rceil}^{n - 1}q\lg{n} \NR
\NC\NC=   \sum_{q=1}^{\lceil n/2 \right\rceil - 1}q\lg{n} + \sum_{q=\left\lceil n/2 \right\rceil}^{n - 1}q\lg{n}
          - \sum_{q=1}^{\lceil n/2 \right\rceil - 1}q \NR
\NC\NC=   \sum_{q=1}^{n - 1}q\lg{n} - \sum_{q=1}^{\lceil n/2 \right\rceil - 1}q \NR
\NC\NC\le \frac{n^2}{2}\lg{n} - \frac{n^2}{8} \NR
\stopmathalignment\stopsplitformula
\stopANSWER

% e
\startigBase[a,continue]\startitem
利用\informula[7_4] 給出的界證明:
\informula[7_3] 中的遞迴式解爲 $E[T(n)]=\Theta(n\lg{n})$。
(\hint 用代入法,證明對於某個正常數 $a$ 和足夠大的 $n$,
有 $E[T(n)]\le an\lg{n}$。)
\stopitem\stopigBase

\startANSWER
猜測 $E[T(n)] \le an\lg{n}$:
\startsplitformula\startmathalignment
\NC E[T(n)]
   \NC=   \frac{2}{n}\sum_{q=2}^{n-1}E[T(q)] + \Theta(n) \NR
\NC\NC\le \frac{2}{n}\sum_{q=2}^{n-1}aq\lg{q} + \Theta(n) \mathcomment{(根據猜測)} \NR
\NC\NC\le \frac{2a}{n}\left(\frac{1}{2}n^2\lg{n} - \frac{1}{8}n^2\right)
                 + \Theta(n) \mathcomment{(\informula[7_4])} \NR
\NC\NC=   an\lg{n} - \frac{a}{4}n + \Theta(n) \mathcomment{(根據 $\Theta$ 記號的定義)} \NR
\NC\NC\le an\lg{n} \NR
\stopmathalignment\stopsplitformula
\stopANSWER

\stopPROBLEM

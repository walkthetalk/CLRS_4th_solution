\startsection[
  reference=section:desc_quicksort,
  title={Description of quicksort},
]

%e7.1-1
\startEXERCISE
參照圖 7-1 的方法,說明 \ALGO{PARTITION} 在如下數列上的操作過程。
\startformula
A = \langle 13, 19, 9, 5, 12, 8, 7, 4, 21, 2, 6, 11 \rangle
\stopformula
\stopEXERCISE

\startANSWER
{\externalfigure[output/e7_1_1-1]}
{\externalfigure[output/e7_1_1-2]}
{\externalfigure[output/e7_1_1-3]}
{\externalfigure[output/e7_1_1-4]}
{\externalfigure[output/e7_1_1-5]}
{\externalfigure[output/e7_1_1-6]}
{\externalfigure[output/e7_1_1-7]}
{\externalfigure[output/e7_1_1-8]}
{\externalfigure[output/e7_1_1-9]}
{\externalfigure[output/e7_1_1-10]}
{\externalfigure[output/e7_1_1-11]}
{\externalfigure[output/e7_1_1-12]}
{\externalfigure[output/e7_1_1-13]}
\stopANSWER

%e7.1-2
\startEXERCISE[exercise:same_partition]
當數列 $A[p..r]$ 中的元素都相同時, \ALGO{PARTITION} 返回的 $q$ 值是多少?
修改 \ALGO{PARTITION},使得當數列 $A[p..r]$ 中所有元素的值都相同時,
 $q=\lfloor (p+r)/2\rfloor$。
\stopEXERCISE

\startANSWER
元素都相同時返回的值爲 $r$。

\CLRSH{PARTITION'(A, p, r)}
\startCLRSCODE
x = A[r]
i = p - 1
for j = p to r - 1
	if A[j] <= x
		i = i + 1
		\ALGO{SWAP(A[i],A[j])}
i = i + 1
\ALGO{SWAP(A[i],A[r])}

if i = r
	return ⌊(p + r) / 2⌋
return i
\stopCLRSCODE
\stopANSWER

%e7.1-3
\startEXERCISE
證明:在規模爲 $n$ 的子數列上, \ALGO{PARTITION} 的時間復雜度爲 $\Theta(n)$。
\stopEXERCISE

\startANSWER
\emph{for} 循環的次數爲 $r - 1 - p = \Theta(n)$。
最壞情況下,每次循環都會執行 \emph{if} 塊,
需要常數時間;循環外的語句也需要常數隨時間。
因此時間復雜度爲 $\Theta(n)$。
\stopANSWER

%e7.1-4
\startEXERCISE
修改 \ALGO{QUICKSORT},使其能以非遞增方式配需。
\stopEXERCISE

\startANSWER
只需修改 \ALGO{PARTITION} 中第 4 行的比較條件。
\stopANSWER

\stopsection

\startEXERCISE
令 $f(n)$ 和 $g(n)$ 爲漸進非負函數。
利用記號 $\Theta$ 的基本定義,
證明 \m{\max(f(n),g(n)) = \Theta(f(n)+g(n))}。
\stopEXERCISE
\startANSWER
由於單調遞增,則:
\startsplitformula\startalign[n=3]
\NC \exists n_1, n_2: \NC f(n) \geq 0 \NC \quad\text{若 \m{n > n_1};} \NR
\NC                   \NC g(n) \geq 0 \NC \quad\text{若 \m{n > n_2}。} \NR
\stopalign\stopsplitformula

設 \m{n_0 = \max(n_1,n_2)},對於 \m{n > n_0}:
\startsplitformula\startalign
\NC f(n) \NC \leq \max(f(n), g(n)) \NR
\NC g(n) \NC \leq \max(f(n), g(n)) \NR
\NC (f(n) + g(n))/2 \NC \leq \max(f(n),g(n)) \NR
\NC \max(f(n), g(n)) \NC \leq f(n) + g(n) \NR
\stopalign\stopsplitformula

對於最後兩個不等式,有:
\startformula
0 \leq \frac{1}{2}(f(n)+g(n)) \leq \max(f(n),g(n)) \leq f(n) + g(n)\text{,若 \m{n > n_0}。}
\stopformula

這與 \m{\Theta(f(n)+g(n))} 的定義一致,其中 \m{c_1 = 1/2}, \m{c_2=1}。
\stopANSWER

\startPROBLEM
(多項式的漸近行爲)
令 $p(n) = \sum_{i = 0}^{d} {a_i n^i}$ 是 $n$ 的 $d$ 次多項式,
其中 $a_d > 0$, $k$是一個常量。
用漸進記號的定義證明下列性質。
\startigBase[a]
\item 如果 $k \ge d$,那麼 $p(n) = O(n^k)$。
\item 如果 $k \le d$,那麼$p(n) = \Omega(n^k)$。
\item 如果 $k = d$,那麼 $p(n) = \Theta(n^k)$。
\item 如果 $k > d$,那麼 $p(n) = o(n^k)$。
\item 如果 $k < d$,那麼 $p(n) = \omega(n^k)$。
\stopigBase
\stopPROBLEM

\startANSWER
取 $c = a_d + b$,滿足下列不等式:
\startformula
p(n) = \sum_{i = 0}^{d}a_i n^i = a_d n^d + a_{d-1}n^{d-1} + \ldots + a_1 n + a_0 \leq cn^d
\stopformula
兩邊同除以 $n^d$,有:
\startformula
c = a_d + b \geq a_d + \frac{a_{d-1}}n + \frac{a_{d-2}}{n^2} + \ldots + \frac{a_0}{n^d}
\stopformula
即
\startformula
b \geq \frac{a_{d-1}}n + \frac{a_{d-2}}{n^2} + \ldots + \frac{a_0}{n^d}
\stopformula
如果使 $b = 1$,則選取 $n_0$,使得:
\startformula
n_0 = \max(da_{d-1}, d\sqrt{a_{d-2}}, \ldots, d\sqrt[d]{a_0})
\stopformula
對於所有 $n\ge n_0$,有 $p(n) \le c n^d$,即 $O(n^d)$;
如果選 $b = -1$,則可得 $\Omega(n^d)$;
綜合可得 $\Theta(n^d)$。
另外兩個的證明類似。
\stopANSWER

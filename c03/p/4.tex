\startPROBLEM
(漸進記號的性質)
令 $f(n)$ 和 $g(n)$ 是漸進正函數,證明或反駁下面的猜測。
\startigBase[a]
\item $f(n) = O(g(n))$ 蘊含 $g(n) = O(f(n))$。

\startANSWER
錯誤。 $n = O(n^2)$,但是 $n^2 \ne O(n)$。
\stopANSWER

\item $f(n) + g(n) = \Theta(\min(f(n), g(n)))$。

\startANSWER
錯誤。 $n^2 + n \ne \Theta(min(n^2, n)) = \Theta(n)$。
\stopANSWER

\item 如果對於所有足夠大的 $n$,
有 $\lg(g(n))\ge 1$ 且 $f(n)\ge 1$,
那麼 $f(n) = O(g(n))$ 蘊含 $\lg(f(n)) = O(lg(g(n)))$。

\startANSWER
正確。 因爲對於給定 $n \ge n_0$, $f(n) \ge 1$:
\startformula
\exists c, n_0 : \forall n \ge n_0 : 0 \le f(n) \le cg(n)
\stopformula
\startformula
   \Downarrow
\stopformula
\startformula
   0 \le \lg{f(n)} \le \lg(cg(n)) = \lg{c} + \lg{g(n)}
\stopformula
需要證明:
\startformula
\lg{f(n)} \le d\lg{g(n)}
\stopformula
很容易找到 $d$:
\startsplitformula\startmathalignment
\NC d \NC = \frac{\lg{c} + \lg{g(n)}}{\lg{g(n)}} \NR
\NC \NC = \frac{\lg{c}}{\lg{g(n)}} + 1 \NR
\NC \NC \le \lg{c} + 1 \NR
\stopmathalignment\stopsplitformula
最後一步顯然成立,因爲 \m{\lg{g(n)} \geq 1}。
\stopANSWER

\item \m{f(n) = O(g(n))} 蘊含 \m{2^{f(n)} = O(2^{g(n)})}。

\startANSWER
錯誤。 $2n = O(n)$,但是 $2^{2n} = 4^n \ne O(2^n)$。
\stopANSWER

\item $f(n) = O((f(n))^2)$。

\startANSWER
正確。只要 $f(n) \ge 1$, $0 \le f(n) \le c(f(n))^2$ 是很自然的。
當然如果對於所有 $n$, $f(n) < 1$,則錯誤,但我們通常不考慮這種函數。
\stopANSWER

\item $f(n) = O(g(n))$ 蘊含 $g(n) = \Omega(f(n))$。

\startANSWER
正確。如果 $f(n) = O(g(n))$,
則 $0 \le f(n) \le cg(n)$,我們只需證明:
\startformula
0 \le d f(n) \le g(n)
\stopformula
對於 $d = 1/c$,上式顯然成立。
\stopANSWER

\item $f(n) = \Theta(f(n/2))$。

\startANSWER
錯誤。取 \m{f(n) = 2^n},我們需要證明:
\startformula
\exists c_1, c_2, n: \forall n \geq n_0 : 0 \leq c_1 \cdot 2^{n/2} \leq 2^n
   \leq c_2 \cdot 2^{n/2}
\stopformula
顯然不成立。
\stopANSWER

\item \m{f(n) + o(f(n)) = \Theta(f(n))}。

\startANSWER
正確。令 $g(n) = o(f(n))$,
我們需要證明 $c_1 f(n) \le f(n) + g(n) \le c_2 f(n)$,
我們知道:
\startformula
\forall c \exists n_0 \forall n \geq n_0 : cg(n) < f(n)
\stopformula
因此,只需 $c_1 = 1$, $c_2 = 2$ 即可。
\stopANSWER

\stopigBase
\stopPROBLEM

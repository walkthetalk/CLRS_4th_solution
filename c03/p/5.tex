\startPROBLEM
(掌握漸進記號)
令 $f(n)$ 和 $g(n)$ 爲漸進正函式。
證明:

% a
\startigBase[a]\startitem
$\Theta(\Theta(f(n))) = \Theta(f(n))$
\stopitem\stopigBase

\startANSWER
\startsplitformula\startmathalignment[n=3]
\NC c_1 f(n) \le \NC \Theta(f(n)) \NC \le c_2 f(n) \NR
\NC c_3 \Theta(f(n)) \le \NC \Theta(\Theta(f(n))) \NC \le c_4 \Theta(f(n)) \NR
\NC c_1 c_3 f(n) \le \NC \Theta(\Theta(f(n))) \NC \le c_2 c_4 f(n) \NR
\stopmathalignment\stopsplitformula
\stopANSWER

% b
\startigBase[continue]\startitem
$\Theta(f(n)) + O(f(n)) = \Theta(f(n))$
\stopitem\stopigBase

\startANSWER
\startsplitformula\startmathalignment[n=3]
\NC c_1 f(n) \le \NC \Theta(f(n)) \NC \le c_2 f(n) \NR
\NC \NC O(f(n)) \NC \le c_3 f(n) \NR
\NC c_1 \Theta(f(n)) \le \NC \Theta(f(n)) + O(f(n)) \NC \le (c_2 + c_3) f(n) \NR
\stopmathalignment\stopsplitformula
\stopANSWER

% c
\startigBase[continue]\startitem
$\Theta(f(n)) + \Theta(g(n)) = \Theta(f(n)+g(n))$
\stopitem\stopigBase

\startANSWER
\startsplitformula\startmathalignment[n=3]
\NC c_1 f(n) \le \NC \Theta(f(n)) \NC \le c_2 f(n) \NR
\NC c_3 g(n) \le \NC \Theta(g(n)) \NC \le c_4 g(n) \NR
\NC \min(c_1,c_3) (f(n)+g(n)) \le
	 \NC \Theta(f(n)) + \Theta(g(n))
	 \NC \le \max(c_2 + c_4) (f(n)+g(n)) \NR
\stopmathalignment\stopsplitformula
\stopANSWER

% d
\startigBase[continue]\startitem
$\Theta(f(n)) \cdot \Theta(g(n)) = \Theta(f(n) \cdot g(n))$
\stopitem\stopigBase

\startANSWER
\startsplitformula\startmathalignment[n=3]
\NC c_1 f(n) \le \NC \Theta(f(n)) \NC \le c_2 f(n) \NR
\NC c_3 g(n) \le \NC \Theta(g(n)) \NC \le c_4 g(n) \NR
\NC c_1 c_3 (f(n)\cdot g(n)) \le
	 \NC \Theta(f(n)) \cdot \Theta(g(n))
	 \NC \le c_2 c_4 (f(n)\cdot g(n)) \NR
\stopmathalignment\stopsplitformula
\stopANSWER

% e
\startigBase[continue]\startitem
分析對於任意常實數 $a_1,a_2 > 0$ 和常整數 $k_1,k_2$,
下列漸進界是否都能成立:
\startformula
(a_1 n)^{k_1} \lg^{k_2}(a_2 n) = \Theta(n^{k_1} \lg^{k_2} n)
\stopformula
\stopitem\stopigBase

\startANSWER
令 $n>n_0=1/a_2$,則有 $\lg(a_2 n)\ge 0$;
令 $0 < c_1 < 1 < c_2$,對於任意 $a_2 > 0$,要使下式成立:
\startsplitformula\startmathalignment[n=3,align={right,middle,left}]
\NC n^{c_1 - 1} \le \NC a_2 \NC \le n^{c_2 - 1} \NR
\stopmathalignment\stopsplitformula
,要求 $n\ge {a_2}^{1/(c_1-1)}$,且 $n\ge {a_2}^{1/(c_2-1)}$,
即 $n \ge \max\left({a_2}^{1/(c_1-1)}, {a_2}^{1/(c_2-1)}\right)$,

\startsplitformula\startmathalignment[n=3,align={right,middle,left}]
\NC n^{c_1 - 1} \le \NC a_2 \NC \le n^{c_2 - 1} \NR
\NC n^{c_1} \le \NC a_2 n \NC \le n^c_2 \NR
\NC c_1 \lg n \le \NC \lg (a_2 n) \NC \le c_2 \lg n \NR
\NC c_1^{k_2} \lg^{k_2} n \le \NC \lg^{k_2}(a_2 n) \NC \le c_2^{k_2} \lg^{k_2}n \NR
\NC c_1^{k_2} (a_1 n)^{k_1} \lg^{k_2} n \le
    \NC (a_1 n)^{k_1} \lg^{k_2}(a_2 n)
	\NC \le c_2^{k_2} (a_1 n)^{k_1} \lg^{k_2}n \NR
\NC (c_1^{k_2} a_1^{k_1}) n^{k_1} \lg^{k_2} n \le
    \NC (a_1 n)^{k_1} \lg^{k_2}(a_2 n)
	\NC \le (c_2^{k_2} a_1^{k_1}) n^{k_1} \lg^{k_2}n \NR
\stopmathalignment\stopsplitformula
\stopANSWER

% f
\startigBase[continue]\startitem
證明對於所有 $S\subseteq Z$,下式成立:
\startformula
\sum_{k\in S}\Theta(f(k)) = \Theta(\sum_{k\in S}f(k))
\stopformula
假設兩邊求和均收斂。
\stopitem\stopigBase

\startANSWER
令 $g(n)=\Theta(f(n))$,則:
\startformula
\exists c_1 >0, c_2 >0, n_0 > 0:
  \forall n\ge n_0,
  0\le c_1 f(n) \le g(n)\le c_2 f(n)
\stopformula

令 $g(k_i)=\Theta(f(k_i))$,則:
\startformula
\exists c_1 >0, c_2 >0, n_0 > 0:
  \forall k_i\ge n_0,
  0\le c_1 f(k_i) \le g(k_i)\le c_2 f(k_i)
\stopformula
其中 $i=1,2,3,\cdots$,
且 $\cup k_i = S$。

因此:
\startformula\startmathalignment[n=1]
\NC \exists c_1 >0, c_2 >0, n_0 > 0:
  \forall k_i\ge n_0, \hfill\NR
\NC \qquad\qquad 0\le \sum c_1 f(k_i) \le \sum g(k_i)\le \sum c_2 f(k_i) \NR
\stopmathalignment\stopformula
即:
\startformula\startmathalignment[n=1]
\NC \exists c_1 >0, c_2 >0, n_0 > 0: \forall k_i\ge n_0,\hfill\NR
\NC \qquad\qquad
  0\le c_1\sum f(k_i) \le \sum \Theta(f(k_i))\le c_2\sum f(k_i) \NR
\stopmathalignment\stopformula
所以:
\startformula
\sum_{k\in S}\Theta(f(k)) = \Theta(\sum_{k\in S}f(k))
\stopformula
\stopANSWER

% g
\startigBase[continue]\startitem
對於 $S\subseteq Z$,
即使下式兩側求積均收斂,
此漸進界也不一定成立,
請給出反例。
\startformula
\prod_{k\in S}\Theta(f(k)) = \Theta(\prod_{k\in S}f(k))
\stopformula
\stopitem\stopigBase

\startANSWER
主要是係數的問題,求和時係數並不會發生變化,
但求積時係數會變, $S$ 中元素數目越大,係數變化就越大。

根據上一項的答案,如果 $S$ 中元素數目爲 $n$,
右側係數爲 $c_1,c_2$,
那麼左側係數會變爲 $c_1^n, c_2^n$。
\stopANSWER

\stopPROBLEM

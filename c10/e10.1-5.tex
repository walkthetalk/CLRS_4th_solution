\startEXERCISE
重寫 \ALGO{ENQUEUE} 和 \ALGO{DEQUEUE} 以處理上溢和下溢的情況。
\stopEXERCISE

\startANSWER
輔助過程:

\CLRSH{NEXT(Q, i)}
\startCLRSCODE
if i == Q.length
	return 1
else
	return i + 1
\stopCLRSCODE

\CLRSH{EMPTY(Q)}
\startCLRSCODE
if Q.head == Q.tail
	return true
else
	return false
\stopCLRSCODE

\CLRSH{FULL(Q)}
\startCLRSCODE
if \ALGO{NEXT(Q, Q.tail)} == Q.head
	return true
else
	return false
\stopCLRSCODE

實現:

\CLRSH{ENQUEUE(Q, x)}
\startCLRSCODE
if \ALGO{FULL(Q)}
	error "Queue overflow"
Q[Q.tail] = x
Q.tail = \ALGO{NEXT(Q, Q.tail)}
\stopCLRSCODE

\CLRSH{DEQUEUE(Q)}
\startCLRSCODE
if \ALGO{EMPTY(Q)}
	error "Queue underflow"
x = Q[Q.head]
Q.head = \ALGO{NEXT(Q, Q.head)}
return x
\stopCLRSCODE
\stopANSWER

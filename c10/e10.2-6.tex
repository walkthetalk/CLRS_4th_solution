\startEXERCISE\DIFFICULT
實現雙向鏈表,要求每個元素只用一個指針 $x.np$,而不是兩個($next$ 和 $prev$)。
假設所有指針都可視爲 $k$ 位整數,
同時令 $x.np = x.next \mathbin{XOR} x.prev$,即 $x.prev$ 和 $x.next$ 的 $k$ 位“異或”。
(0 表示 NIL。)
要訪問鏈表頭需要哪些信息?
如何實現 \ALGO{SEARCH}、 \ALGO{INSERT} 和 \ALGO{DELETE}?
如何在 $O(1)$ 時間內反序?
\stopEXERCISE

\startANSWER
如果是通過前驅節點訪問的當前節點,
則已經知道了 $prev$ 的值,
這時通過 $prev \mathbin{XOR} np$ 即可得 $next$ 的值。

如果是通過後繼節點訪問的當前節點,
則已經知道了 $next$ 的值,
這時通過 $next \mathbin{XOR} np$ 即可得 $prev$ 的值。

\ALGO{SEARCH}、 \ALGO{INSERT} 和 \ALGO{DELETE} 的實現與普通雙向鏈表沒什麼區別,
只是增加了計算 $prev$、 $next$ 和 $np$ 的步驟。

要訪問鏈表只需知道第一個元素或者最後一個元素的指針即可。

反序只需要調換以下 $head$ 和 $tail$ 的值即可。
\stopANSWER

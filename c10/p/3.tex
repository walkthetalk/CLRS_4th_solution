\startPROBLEM
(Searching a sorted compact list)
我們可以用兩個數列 key 和 next 來表示單向鏈表。
給定某個元素的索引 $i$,
 $key[i]$ 中存儲的是他的值,
 $next[i]$ 則是其後繼的索引,
如果沒有後繼, $next[i]$ 就是 $NIL$。
除此之外,我們還需要鏈表第一個元素的索引 head。
如果這個鏈表是已排序的,
我們就他是\emph{緊湊的}(\emph{compact})。

我們假設所有 key 各不相同,且緊湊鏈表已排序,即,
 $key[i] < key[next[i]]$,
其中 $i=1,2,\ldots,n$,
且 $next[i] \ne NIL$。
這些條件下,
隨機化算法 \ALGO{COMPACT-LIST-SEARCH} 搜索
關鍵字 $k$ 所需時間爲 $O(\sqrt{n})$。

\CLRSH{COMPACT-LIST-SEARCH(key,next,head,n,k)}
\startCLRSCODE
i = head
while i \ne NIL and key[i] < k
	j = \ALGO{RANDOM(1, n)}
	if key[i] < key[j] and key[j] \le k
		i = j
		if key[i] == k
			return i
	i = next[i]
if i == NIL or key[i] > k
	return NIL
else
	return i
\stopCLRSCODE

如果忽略 3~7 行,
那麼他就退化成一個普通算法,
用於搜索已排序鏈表,
其中下標 $i$ 依次指向鏈表的各個位置。
一旦下標 $i$ 落到鏈表以外,
或者 $key[i]\ge k$,算法終止。
在 $key[i]\ge k$ 這種情況下,
如果 $key[i] = k$,
顯然找到了值爲 $k$ 的關鍵字;
而如果 $key[i] > k$,
那麼就無法找到值爲 $k$ 的關鍵字了;
因此算法終止。

3~7 行試圖跳到一個隨機選擇的位置 $j$。
如果 $key[j]$ 大於 $key[i]$,
且不大於 $k$,那麼這樣處理是有好處的;
這種情況下, $j$ 所指位置是普通搜索時 $i$ 的必經之地。
由於鏈表是緊湊的,
下標爲 1 到 n 之間的任意值所指位置的元素都屬於這個鏈表。

我們不直接分析 \ALGO{COMPACT-LIST-SEARCH} 的性能,
而是先分析一個相關的算法, \ALGO{COMPACT-LIST-SEARCH'},
他會執行兩個獨立的循環。
這個算法有一個額外的參數 $t$,
此參數決定了第一個循環迭代次數的上界。

\CLRSH{COMPACT-LIST-SEARCH‘(key,next,head,n,k,t)}
\startCLRSCODE
i = head
for q = 1 to t
	j = \ALGO{RANDOM(1, n)}
	if key[i] < key[j] and key[j] \le k
		i = j
		if key[i] == k
			return i
while i \ne NIL and key[i] < k
	i = next[i]
if i == NIL or key[i] > k
	return NIL
else
	return i
\stopCLRSCODE

爲了比較這兩個算法,
我們假設兩個算法中調用 \ALGO{RANDOM(1, n)} 所返回的序列是相同的。
\startigBase[a]\startitem
證明:無論 $t$ 爲何值,
 \ALGO{COMPACT-LIST-SEARCH(key,next,head,n,k)} 和
 \ALGO{COMPACT-LIST-SEARCH'(key,next,head,n,k,t)} 返回的結果都一樣,
且 \ALGO{COMPACT-LIST-SEARCH} 中 2~8 行 while 循環的迭代次數
與 \ALGO{COMPACT-LIST-SEARCH'} 中 for、 while 循環迭代的總次數幾乎相同。
\stopitem\stopigBase

\startANSWER
如果原始算法只迭代了 $t$ 次,
則說明最多只需 $t$ 次隨機跳轉即可找到期望的值。
\stopANSWER

在調用 \ALGO{COMPACT-LIST-SEARCH'(key,next,head,n,k,t)} 的時候,
令隨機變量 $X_t$ 代表 2~7 行的 for 循環迭代 $t$ 次後鏈表中
從位置 i 到目標關鍵字 k 之間的距離(即通過 $next$ 指針)。
% b
\startigBase[continue]\startitem
論證 \ALGO{COMPACT-LIST-SEARCH'(key,next,head,n,k,t)} 的
期望運行時間為 $O(t+E[X_t])$。
\stopitem\stopigBase

\startANSWER
要不就是迭代 $t$ 次後找到了 $k$,
否則 while 循環還要迭代 $X_t$ 次。
\stopANSWER

% c
\startigBase[continue]\startitem
證明: $E[X_t]\le \sum_{r=1}^{n}(1-r/n)^t$。
(\hint 用等式 C.28)附等式 C.28:
\startsplitformula\startmathalignment
\NC E[X] \NC = \sum_{i=0}^{\infty}i\cdot \Pr\{X=i\} \NR
\NC \NC = \sum_{i=0}^{\infty}i\cdot (\Pr\{X\ge i\}-\Pr\{X\ge i+1\}) \NR
\NC \NC = \sum_{i=1}^{\infty}\Pr\{X\ge i\} \NR
\stopmathalignment\stopsplitformula
\stopitem\stopigBase

\startANSWER
當 $t=1$ 時,距離大於等於 r 的概率为 $(n-r)/n$。
\startformula
\Pr\{X_t\ge r\} = (\frac{n-r}{n})^t = (1-\frac{r}{n})^t
\stopformula

即,有一次調用 \ALGO{RANDOM} 會得到期望的距離,
而其他的則會不斷接近期望距離。
利用 C.28:
\startformula
E[X_t] = \sum_{r=1}^{\infty} \Pr\{X_t \ge r\}
        = \sum_{r=1}^n \Pr\{X_t \ge r\}
        = \sum_{r=1}^n \left(1 - \frac{r}{n}\right)^t
\stopformula
\stopANSWER

% d
\startigBase[continue]\startitem
證明: $\sum_{r=0}^{n-1}r^t\le n^{t+1}/(t+1)$。
(\hint 利用等式 A.18)附等式 A.18:
\startformula
\int_{m-1}^{n}f(x) dx \le \sum_{k=m}^{n} f(k)\le \int_{m}^{n+1} f(x) dx
\stopformula
\stopitem\stopigBase

\startANSWER
\startformula
\sum_{r=0}^{n-1}r^t
\le \int_{0}^{n} x^{t} dx
\le \frac{x^{t+1}}{t+1}\vert_{x=0}^{n}
= \frac{n^{t+1}}{t+1}
\stopformula
\stopANSWER

% e
\startigBase[continue]\startitem
證明: $E[X_t]\le n/(t+1)$。
\stopitem\stopigBase

\startANSWER
由上一項可知:
\startformula
\sum_{r=0}^{n-1} r^t \le \frac{n^{t+1}}{t+1}
\stopformula
因此:
\startsplitformula\startmathalignment
\NC E[X_t]
    \NC= \sum_{r=1}^n \bigg(1 - \frac{r}{n}\bigg)^t \NR
\NC \NC= \sum_{r=0}^{n-1} \bigg(\frac{r}{n}\bigg)^t \NR
\NC \NC= \frac{1}{n^t} \sum_{r=0}^{n-1} r^t \NR
\NC \NC\le \frac{1}{n^t} \cdot \frac{n^{t+1}}{t + 1} \NR
\NC \NC= \frac{n}{t+1} \NR
\stopmathalignment\stopsplitformula
\stopANSWER

% f
\startigBase[continue]\startitem
證明: \ALGO{COMPACT-LIST-SEARCH'(key,next,head,n,k,t)} 的
期望運行時間爲 $O(t+n/t)$。
\stopitem\stopigBase

\startANSWER
根據上一項:
\startformula
O(t + E[X_t]) = O(t + \frac{n}{t+1}) = O(t + \frac{n}{t})
\stopformula
\stopANSWER

% g
\startigBase[continue]\startitem
論證 \ALGO{COMPACT-LIST-SEARCH} 的期望運行時間爲 $O(\sqrt{n})$。
\stopitem\stopigBase

\startANSWER
我們只需找到 $t+n/t$ 的最小值。
針對 $t$ 求導爲 $1-n/t^2$, $t=\sqrt{n}$ 時導數爲零。
因此期望運行時間爲 $O(\sqrt{n})$。
\stopANSWER

% h
\startigBase[continue]\startitem
爲什麼我們假設 \ALGO{COMPACT-LIST-SEARCH} 中的所有關鍵字互不相同?
論證如果包含相同關鍵字,則隨機跳躍不一定能改進漸進時間。
\stopitem\stopigBase

\startANSWER
如果有相同關鍵字,則無法得到 $E[X_t]\le n/(t+1)$。
只有當 \ALGO{RANDOM} 找到的值大於當前值時才會發生躍變。
例如,如果鏈表中全都是 0,而我們要找 1,
算法仍然需要一直迭代,直到鏈表尾部。
\stopANSWER

\stopPROBLEM

\startEXERCISE
假設每個節點 x 不是用屬性 $x.p$ 來指向其父節點,
而是用屬性 $x.succ$ 指向其後繼節點,
給出用這種方法表示的二叉搜索樹 $T$ 上的
 \ALGO{SEARCH}、 \ALGO{INSERT} 和 \ALGO{DELETE} 操作的僞碼。
這些過程應在 $O(h)$ 時間內完成,
其中 $h$ 是 $T$ 的高度。
(\hint 你可能需要實現一個子過程來返回父節點)
\stopEXERCISE

\startANSWER
\ALGO{SEARCH} 不用變。

\CLRSH{TREE-INSERT(T,z)}
\startCLRSCODE
y = NIL
x = T.root
while x \ne NIL
	y = x
	if z.key < x.key
		x = x.left
	else
		x = x.right
if y == NIL
	T.root = z	// $T$ 爲空
	z.succ = NIL
elseif z.key < y.key
	y.left = z
	z.succ = y
else
	y.right = z
	z.succ = y.succ
	y.succ = z
\stopCLRSCODE

\CLRSH{TREE-PARENT(T, z)}
\startCLRSCODE
x = z
while x.right \ne NIL
	x = x.right
x = x.succ
if x == NIL
	y = T.root
	while y.right \ne z
		y = y.right
	return y
else
	if x.left == z
		return x	// $x$ 是其父節點的左孩子
	else
		y = x.left
		while y.right \ne z
			y = y.right
		return y
\stopCLRSCODE

\CLRSH{TREE-PREDECESSOR(T, z)}
\startCLRSCODE
if z.left \ne NIL
	y = z.left
	while y.right \ne NIL
		y = y.right
	return y
else
	p = \ALGO{TREE-PARENT(T, z)}
	if p.left == z
		return NIL
	esle
		return p
\stopCLRSCODE

\CLRSH{TREE-TRANSPLANT(T, u, v)}
\startCLRSCODE
y = \ALGO{TREE-PARENT(T, u)}
if y == NIL
	T.root = v
else if u == y.left
	y.left = v
else
	y.right = v
\stopCLRSCODE

\CLRSH{TREE-DELETE(T, z)}
\startCLRSCODE
if z.left == NIL
	\ALGO{TREE-TRANSPLANT(T, z, z.right)}
else if z.right == NIL
	\ALGO{TREE-TRANSPLANT(T, z, z.left)}
else
	y = \ALGO{TREE-MINIMUM(z.right)}
	if y \ne z.right
		\ALGO{TREE-TRANSPLANT(T, y, y.right)}
		y.right = z.right
	\ALGO{TREE-TRANSPLANT(T, z, y)}
	y.left = z.left
pd = \ALGO{TREE-PREDECESSOR(T, z)}
if pd \ne NIL
	pd.succ = z.succ
\stopCLRSCODE
\stopANSWER

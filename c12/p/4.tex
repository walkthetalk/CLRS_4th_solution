\startPROBLEM
(Number of different binary trees)
令 $b_n$ 表示由 $n$ 個節點所構成的二叉樹的數目。
本題會給出 $b_n$ 的公式以及漸進估計。
\startigBase[a]\startitem
證明 $b_0=1$,且當 $n\ge 1$ 時,
\startformula
b_n = \sum_{k=0}^{n-1}b_k b_{n-1-k}
\stopformula
\stopitem\stopigBase

\startANSWER
主要考慮根節點的左右兩棵子樹中的節點數目。
\stopANSWER

\startigBase[continue]\startitem
參考\refproblem{generating_function} 中生成函數的定義,
令 $B(x)$ 爲生成函數:
\startformula
B(x) = \sum_{n=0}^{\infty}b_n x^n
\stopformula
試證明 $B(x) = x B(x)^2 + 1$,也可用如下形式表示 $B(x)$:
\startformula
B(x) = \frac{1}{2x}(1 - \sqrt{1-4x})
\stopformula
\stopitem\stopigBase

\startANSWER
\startformula\startmathalignment
\NC B(x) \NC = \sum_{n=0}^{\infty}b_n x^n \NR
\NC \NC = 1 + \sum_{n=1}^{\infty}b_n x^n \NR
\NC \NC = 1 + \sum_{n=1}^{\infty}\sum_{k=0}^{n-1}b_k b_{n-k-1} x^n \NR
\NC \NC = 1 + x \sum_{n=1}^{\infty}\sum_{k=0}^{n-1}b_k x^k b_{n-k-1} x^{n-k-1} \NR
\NC \NC = 1 + x \sum_{n=0}^{\infty}\sum_{k=0}^{n}b_k x^k b_{n-k} x^{n-k} \NR
\NC \NC = 1 + x B(x)^2\NR
\stopmathalignment\stopformula

$xB(x)^2 + 1 = x \cdot \frac{1}{4x^2}(1+1-4x-2\sqrt{1-4x}) + 1 = B(x)$。
\stopANSWER

$f(x)$ 在 $x=a$ 處\emph{泰勒展開式(Taylor expansion)}爲:
\startformula
f(x) = \sum_{k=0}^{\infty}\frac{f^{(k)}(a)}{k!}(x-a)^k
\stopformula
其中 $f^{(k)}(x)$ 是 $f$ 對 $x$ 的 $k$ 階導數。

% c
\startigBase[continue]\startitem
用 $\sqrt{1-4x}$ 在 $x=0$ 處的泰勒展開式證明:
\startformula
b_n = \frac{1}{n+1}\binom{2n}{n}
\stopformula
(即第 $n$ 個 \emph{Catalan number})
如果不想用泰勒展開式,
也可以使用二項展開式,
參見\insection[geometric_binom],
將其推廣到非整數的指數 $n$ 上去,
也就是對於所有實數 $n$ 和任一整數 $k$,
當 $k\ge 0$ 時,$\binom{n}{k}$ 可以表示爲 $n(n-1)\cdots(n-k+1)/k!$,否則爲 0。
\stopitem\stopigBase

\startANSWER
令 $f(x) = \sqrt{1-4x}$,其泰勒展開式爲:
\startformula
\sqrt{1-4x} = 1 - \frac{2}{1!}x
             - \frac{4}{2!}x^2
	     - \frac{24}{3!}x^3
	     - \frac{240}{4!}x^4
	     - \cdots
\stopformula
因此:
\startsplitformula\startmathalignment
\NC B(x) \NC = \frac{1}{2x}(1-\sqrt{1-4x}) \NR
\NC      \NC = \frac{2}{1!}x
               + \frac{4}{2!}x^2
	       + \frac{24}{3!}x^3
	       + \frac{240}{4!}x^4
	       + \cdots \NR
\NC      \NC = \frac{1}{1!}
               + \frac{2}{2!}x
	       + \frac{12}{3!}x^2
	       + \frac{120}{4!}x^3
	       + \cdots \NR
\NC      \NC = \sum_{n=1}^{\infty}\frac{(2n-2)!}{n(n-1)!(n-1)!}x^{n-1} \NR
\NC      \NC = \sum_{n=0}^{\infty}\frac{1}{n+1}\binom{2n}{n}x^n \NR
\stopmathalignment\stopsplitformula
又由於 $B(x) \NC = \sum_{n=0}^{\infty}b_n x^n$,所以:
\startformula
b_n = \frac{1}{n+1}\binom{2n}{n}
\stopformula
\stopANSWER

% d
\startigBase[continue]\startitem
證明:
\startformula
b_n = \frac{4^n}{\sqrt{\pi}n^{3/2}}(1+O(1/n))
\stopformula
\stopitem\stopigBase

\startANSWER
根據 Stirling 近似:
\startformula
n! = \sqrt{2\pi n}
     \left(\frac{n}{e}\right)^n
     \left(1 + \Theta\left(\frac{1}{n}\right)\right)
\stopformula
可得:
\startformula
b_n = \frac{1}{n+1}\binom{2n}{n}
    = \frac{4^n}{\sqrt{\pi}n^{3/2}}
      \left(1 + O\left(\frac{1}{n}\right)\right)
\stopformula
\stopANSWER

\stopPROBLEM

\startPROBLEM
(Slot-size bound for chaining)
假設散列表有 $n$ 個槽位,用鏈接法解決衝突。
向表中插入 $n$ 個關鍵字,
每個關鍵字被等可能地散列到各個槽中。
插入所有關鍵字後,令 $M$ 爲各槽中所含關鍵字數目的最大值。
讀者的任務是證明 $E[M]$ ($M$ 的期望值)
的上界爲 $O(\lg{n}/\lg\lg{n})$。
% a
\startigBase[a]\startitem
證明:正好有 k 個關鍵字被散列到某一特定槽中的概率 $Q_k$ 爲:
\startformula
Q_k = \left(\frac{1}{n}\right)^k
      \left(1-\frac{1}{n}\right)^{n-k}
      \binom{n}{k}
\stopformula
\stopitem\stopigBase

\startANSWER
先在 $n$ 個關鍵字中選出 $k$ 個,
這 $k$ 個關鍵字都會散列到特定槽中(每次散列的概率爲 $1/n$),
其他關鍵字都散列到其他槽中(每次散列的概率爲 $1- 1/n$)。
\stopANSWER

% b
\startigBase[continue]\startitem
令 $P_k$ 爲 $M=k$ 的概率,
即包含最多關鍵字的槽中有 $k$ 個關鍵字的概率。
證明: $P_k\le n Q_k$。
\stopitem\stopigBase

\startANSWER
所選特定槽可能是 $n$ 個槽中的任何一個,且概率相等。
\stopANSWER

% c
\startigBase[continue]\startitem
應用 Stirling 近似公式 3.25 證明: $Q_k < e^k / k^k$。
附 Stirling 近似公式:
\startformula
n! = \sqrt{2\pi n}
     \left(\frac{n}{e}\right)^n
     \left(1 + \Theta\left(\frac{1}{n}\right)\right)
\stopformula
\stopitem\stopigBase

\startANSWER
\startsplitformula\startmathalignment
\NC Q_k
    \NC = \left(\frac{1}{n}\right)^k
          \left(1-\frac{1}{n}\right)^{n-k} \binom{n}{k} \NR
\NC \NC < \frac{1}{n^k} \frac{n!}{k!(n-k)!} \NR
\NC \NC < \frac{1}{k!} \NR
\NC \NC < \frac{1}{(\frac{k}{e})^k} \NR
\NC \NC = \frac{e^k}{k^k} \NR
\stopmathalignment\stopsplitformula
\stopANSWER

% d
\startigBase[continue]\startitem
證明:存在常數 $c>1$,
使得 $Q_{k_0} < 1/n^3$ 對 $k_0 = c \lg{n}/\lg\lg{n}$ 成立。
並有結論:對於 $k\ge k_0 = c\lg{n}/\lg\lg{n}$, $P_k < 1/n^2$ 成立。
\stopitem\stopigBase

\startANSWER
\startsplitformula\startmathalignment
\NC \frac{e^{k_0}}{{k_0}^{k_0}} \NC< \frac{1}{n^3} \NR
\NC (\frac{k_0}{e})^{k_0} \NC> n^3 \NR
\intertext{兩邊同取對數:}
\NC k_0 (\lg{k_0} - 1) \NC> 3 \lg{n} \NR
\NC c \frac{\lg{n}}{\lg\lg{n}} (\lg(\frac{c\lg{n}}{\lg\lg{n}}) - 1)
   \NC> 3 \lg{n} \NR
\NC c \frac{\lg{n}}{\lg\lg{n}} (\lg{c} + \lg\lg{n} - \lg\lg\lg{n} - 1)
   \NC> 3 \lg{n} \NR
\NC c (\lg{c} + \lg\lg{n} - \lg\lg\lg{n} - 1)
   \NC> 3 \lg\lg{n} \NR
\NC c (\lg{c} - \lg\lg\lg{n} - 1) + c \lg\lg{n}
   \NC> 3 \lg\lg{n} \NR
\intertext{令 $c>3$:}
\NC c (\lg{c} - \lg\lg\lg{n} - 1) \NC> 0 \NR
\NC \lg{c} - \lg\lg\lg{n} \NC> 1 \NR
\NC c / \lg\lg{n} \NC> e \NR
\NC c \NC> e \lg\lg{n} \NR
\stopmathalignment\stopsplitformula

對於 $k\ge k_0 = c\lg{n}/\lg\lg{n}$,
由 b)可知 $P_k\le n Q_k$,又 $Q_k<1/n^3$,因此 $P_k < 1/n^2$。
\stopANSWER

% e
\startigBase[continue]\startitem
證明:
\startformula
E[M] \le \Pr\{ M>\frac{c\lg{n}}{\lg\lg{n}} \} \cdot n
         + \Pr\{ M\le \frac{c\lg{n}}{\lg\lg{n}} \} \cdot \frac{c\lg{n}}{\lg\lg{n}}
\stopformula
並有結論: $E[M] = O(\lg{n}/\lg\lg{n})$。
\stopitem\stopigBase

\startANSWER
令 $k_0 = \frac{c\lg{n}}{\lg\lg{n}}$,則:
\startsplitformula\startmathalignment
\NC E[M]
    \NC = \sum_{k=1}^{n}k\Pr\{M=k\} \NR
\NC \NC = \sum_{k=1}^{k_0}k\Pr\{M=k\}
          + \sum_{k=k_0+1}^{n}k\Pr\{M=k\} \NR
\NC \NC < k_0 \sum_{k=1}^{k_0}\Pr\{M=k\}
          + n\sum_{k=k_0+1}^{n}\Pr\{M=k\} \NR
\NC \NC = k_0 \Pr\{M\le k_0\} + n\Pr\{M>k_0\} \NR
\intertext{其中 $\Pr\{M\le k_0\}\le 1$, $\Pr\{M>k_0\} = P_k < \frac{1}{n^2}$:}
\NC \NC < k_0 + n \cdot \frac{1}{n^2} \NR
\NC \NC < k_0 + 1 \NR
\NC \NC = O(k_0) \NR
\stopmathalignment\stopsplitformula
\stopANSWER

\stopPROBLEM

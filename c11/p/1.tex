\startPROBLEM
(Longest-probe bound for hashing)
有開放尋址散列表,大小爲 $m$,
用於存儲 $n$ ($n\le m/2$)個數據項。
% a
\startigBase[a]\startitem
假設採用獨立均勻散列,證明:
對於 $i=1,2,\ldots,n$,
第 $i$ 次插入所需探查次數嚴格多於 $k$ 次的概率至多爲 $2^{-k}$。
\stopitem\stopigBase

\startANSWER
\startsplitformula\startmathalignment
\NC \Pr\{x>k\}
    \NC = \Pr\{x\ge k+1\} \NR
\NC \NC = \frac{n}{m} \cdot \frac{n-1}{m-1}
          \cdot \ldots \cdot
	  \frac{n-k+1}{m-k+1} \mathcomment{參見定理 11.6} \NR
\NC \NC < (\frac{n}{m})^k \NR
\NC \NC \le (\frac{1}{2})^k \NR
\NC \NC = 2^{-k} \NR
\stopmathalignment\stopsplitformula
\stopANSWER

% b
\startigBase[continue]\startitem
證明:對於 $i=1,2,\ldots,n$,
第 $i$ 次插入所需探查次數多於 $2\lg{n}$ 的概率爲 $O(1/n^2)$。
\stopitem\stopigBase

\startANSWER
根據上一項的結果:
\startformula
\Pr\{x>k\} \le 2^{-k} = 2^{-2\lg{n}} = \frac{1}{n^2}
\stopformula
\stopANSWER

令隨機變量 $X_i$ 表示第 $i$ 次插入所需的探查次數。
在上面 (b) 中已證明 $\Pr\{X_i>2\lg{n} = O(1/n^2)\}$。
令隨機變量 $X=\max\{X_i: 1\le i\le n\}$ 表示
任意 $n$ 次插入過程中所需探查數的最大值。

% c
\startigBase[continue]\startitem
證明: $\Pr\{X>2\lg{n}\} = O(1/n)$。
\stopitem\stopigBase

\startANSWER
\startformula
\Pr\{X>2\lg{n}\}
  \le \sum_{i=1}^{n}\Pr\{X_i>2\lg{n}\}
  \le \sum_{i=1}^{n}\frac{1}{n^2}
  = n \cdot \frac{1}{n^2}
  = \frac{1}{n}
\stopformula
\stopANSWER

% d
\startigBase[continue]\startitem
證明:最長探查序列的期望長度爲 $E[x]=O(\lg{n})$。
\stopitem\stopigBase

\startANSWER
\startsplitformula\startmathalignment
\NC E[x]
    \NC = \sum_{k=1}^{n}k\Pr\{X=k\} \NR
\NC \NC = \sum_{k=1}^{2\lg{n}}k\Pr\{X=k\} + \sum_{k=2\lg{n}+1}^{n}k\Pr\{X=k\} \NR
\NC \NC < 2\lg{n}\sum_{k=1}^{2\lg{n}}\Pr\{X=k\} + n\sum_{k=2\lg{n}+1}^{n}\Pr\{X=k\} \NR
\NC \NC = 2\lg{n}\Pr\{X\le 2\lg{n}\} + n\Pr\{X>2\lg{n}\} \NR
\intertext{其中 $\Pr\{X\le 2\lg{n}\}\le 1$, $\Pr\{X>2\lg{n}\} \le \frac{1}{n^2}$:}
\NC \NC \le 2\lg{n} + n \cdot \frac{1}{n^2} \NR
\NC \NC < 2\lg{n} + 1 \NR
\NC \NC = O(\lg{n}) \NR
\stopmathalignment\stopsplitformula
\stopANSWER

\stopPROBLEM

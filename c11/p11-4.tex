\startPROBLEM
(Hashing and authentication)
有散列函數族 ${\cal H}$,
其中每個函數 $h\in{\cal H}$ 都可以
將關鍵字的全域 $U$ 映射到 $\{0,1,\ldots,m-1\}$ 上。
如果對於任意 $k$ 個互不相同的關鍵字
構成的的固定序列 $\langle x_1,x_2,\ldots,x_k \rangle$,
以及從 ${\cal H}$ 中隨機選出的 $h$,
序列 $\langle h(x_1),h(x_2),\ldots,h(x_k) \rangle$
 是 $m^k$ 種長度爲 $k$ 的序列
(其元素取自 $\{0,1,\ldots,m-1\}$)中任意一個,其可能性相同,
我們稱 ${\cal H}$ 是 \emph{k-獨立}(k-independent)的。
\startigBase[a]\startitem%a
證明:如果散列函數族 ${\cal H}$ 是 2-獨立的,則他是全域的。
\stopitem\stopigBase

\startANSWER
如果 ${\cal H}$ 是 2-獨立的,則對於兩個特定關鍵字 $x\ne y$,
序列 $\langle x, y\rangle$ 是 $m^2$ 種序列中的任一個,概率相同。
因此,由於 $h$ 是 ${\cal H}$ 中的任意一個,
產生衝突 $h(x) = h(y)$ 的概率爲 $(1/m)|{\cal H}|$,
所以 ${\cal H}$ 是全域的。
\stopANSWER

\startigBase[continue]\startitem%b
設全域 $U$ 爲取自 $\integers_p = \{0,1,\ldots,p-1\}$ 中數值的 $n$ 元組集合,
其中 p 是素數。
考慮元素 $x=\langle x_0,x_1,\ldots,x_{n-1} \rangle \in U$。
對於任意 n 元組 $a=\langle a_0,a_1,\ldots,a_{n-1} \rangle \in U$,
定義散列函數 $h_a$ 爲:
\startformula
h_a(x) = \left( \sum_{j=0}^{n-1}a_j x_j \right) \mod p
\stopformula
令 ${\cal H}=\{h_a: a\in U\}$。
證明: ${\cal H}$ 是全域的,但不是 2-獨立的。
(\hint 找一個關鍵字,
使得 ${\cal H}$ 中所有散列函數得到的值相同。)
\stopitem\stopigBase

\startANSWER
假設在 ${\cal H}$ 中選擇函數 $h$,
如果 $a=\langle 0,0,\ldots,0\rangle$,則對於序列 $\langle x,y\rangle$,
有 $h_a(x)=h_a(y) = 0 \mod p = 0$,
即這種情況下無論關鍵字是什麼,
映射對結果永遠是 $\langle 0,0\rangle$,所以不是 2-獨立的。
\stopANSWER

\startigBase[continue]\startitem%c
假設將 (b) 中的 ${\cal H}$ 略作修改:
對任意 $a\in U$ 以及任意 $b\in\integers_p$,定義:
\startformula
h'_{ab}(x) = \left( \sum_{j=0}^{n-1}a_j x_j + b \right) \mod p
\stopformula
且 ${\cal H}'=\{h'_{ab}: a \in U, b\in\integers_p\}$。
論證 ${\cal H}'$ 是 2-獨立的。
(\hint 考慮固定的 n 元組 $x\in U$ 和 $y\in U$,
對某個 $i$ 有 $x_i\ne y_i$。
當 $a_i$ 和 $b$ 超過 $\integers_p$ 時,
 $h'_{ab}(x)$ 和 $h'_{ab}(y)$ 會如何?)
\stopitem\stopigBase

\startANSWER
對於每一對關鍵字 $\langle x,y\rangle$,其中 $x\ne y$,
我們要證明所有值對 $(h_{ab}(x),h_{ab}(y))$ 的概率相同,
其中 $h_{ab}$ 是從 ${\cal H}$ 中隨機選出,
即 $\langle a_0,a_1,\ldots,a_{n-1}\rangle$ 和 $b$ 是隨機選出的。

既然 $x\ne y$,那麼肯定存在 $i$ 使得 $x_i\ne y_i$,
假設 $i=0$,令:
\startformula
\alpha = h_{ab}(x)
\qquad \beta = h_{ab}(y)
\stopformula
且:
\startformula
X=\sum_{j=1}^{n-1}a_j x_j
\qquad Y=\sum_{j=1}^{n-1}a_j y_j
\stopformula
則:
\startformula
\alpha = (a_0 x_0 + b + X) \mod p
\qquad \beta = (a_0 y_0 + b + Y) \mod p
\stopformula
由於 $x_0 \ne y_0$ 且 $p$ 是素數,
那麼對於 $\alpha$ 和 $\beta$ 而言,
上式有唯一解 $a_0$ 和 $b$。

更明顯一點,考慮我們要生成所有可能的值對 $\alpha,\beta$,
他與值對 $\alpha,\alpha-\beta$ 是雙射的。而:
\startformula
\alpha-\beta=a_0(x_0-y_0) + X - Y (\mod p)
\stopformula
所以:
\startformula
a_0 = (\alpha - \beta -X + Y) (x_0 - y_0)^{(-1)} (\mod p)
\stopformula

由於 $x_0 \ne y_0$ 且 $p$ 是素數,
則 $(x_0 - y_0)^{(-1)}$ 肯定存在且唯一。
對於特定 $a_0$ 而言,我們可以隨意選擇 $\alpha-\beta$ 的值。
對於特定 $b$ 而言,我們可以隨意選擇 $\alpha$ 的值。
即對於特定的 $a_0$ 和 $b$,
我們可以隨意選擇 $\alpha$ 和 $\beta$,
只要保證 $(\alpha-\beta)\mod p$ 的值相同即可。

因此,對於任意給定的 $\langle a_0,a_1,\ldots,a_{n-1}\rangle$,
我們可以找到散列函數 $h_{ab}$,
通過選擇合適的 $a_0$ 和 $b$,
使其可以生成任何可能的 $\langle\alpha,\beta\rangle$。

由於 $a_0$ 和 $b$ 的選擇共有 $p^2$ 種,
並且 $\langle\alpha,\beta\rangle$ 也有 $p^2$ 種,
每個 $\langle\alpha,\beta\rangle$ 都對應唯一的 $\langle a_0,b\rangle$。
這對 $a_1,\ldots,a_{n-1}$ 的 $p^{n-1}$ 種選擇均成立。

因此,有 $p^{n-1}$ 種函數 $h_{ab}$ 可以生成每個值對 $\langle\alpha,\beta\rangle$。
如果 $h_{ab}$ 是隨機選擇的,
則每個值對的概率相同,
因此 ${\cal H}$ 是 2-獨立的。
\stopANSWER

\startigBase[continue]\startitem%d
假設 Alice 和 Bob 悄悄地約定了一個散列函數 $h$,
取自 2-獨立的散列函數族 ${\cal H}$。
每個 $h\in {\cal H}$ 將關鍵字全域 U 映射到 $\integers_p$ 上,此處 $p$ 爲素數。
後來, Alice 通過互聯網向 Bob 發送了一個消息 $m$,其中 $m\in U$。
她同時還通過發送一個認證標記 $t=h(m)$ 來向 Bob 認證這一消息,
而 Bob 則要檢查他所收到的 $(m,t)$ 對是否確實滿足 $t=h(m)$。
假設某一對手半路截獲了 $(m,t)$,
並試圖將該值對替換爲令一值對 $(m',t')$ 來欺騙 Bob。
論證無論該對手的計算機性能多好,
他成功欺騙 Bob 接受 $(m',t')$ 的概率至多爲 $1/p$,
即使他知道所用的散列函數族 ${\cal H}$。
\stopitem\stopigBase

\startANSWER
對於關鍵字對 $\langle m,m'\rangle$,如果 $h$ 是隨機選取的,
則所有散列值對 $\langle h(m),h(m')\rangle$ 的概率相同。
這就是 ${\cal H}$ 爲 2-獨立的意思。

即 p 種值對 $\langle t,t'\rangle$ 的概率相同,
所以即使他知道 ${\cal H}$,根據 m 和 t 也無法得出 $h(m')$。
由於 p 種情況的概率相同,且總和爲 1,
任一種可能的 $h(m')$ 概率均爲 $1/p$,
所以他只能靠猜,沒有更好的辦法。
\stopANSWER

\stopPROBLEM

\startEXERCISE\DIFFICULT
設 $U$ 爲 $d$ 元組集合,
其中所有元素均取自 $\integers_p$,
且 $Q=\integers_p$,
其中 $p$ 爲素數。
對於一個取自 $U$ 的 $d$ 元組 $\langle a_0,a_1,\ldots,a_{d-1}\rangle$,
定義其上的散列函數 $h_b: U\rightarrow B(b\in \integers_p)$ 爲:
\startformula
h_b(\langle a_0, a_1, \ldots, a_{d-1} \rangle) =
   \left(\sum_{j=0}^{d-1} a_j b^j \right) \mod p
\stopformula
令 ${\cal H} = \{h_b : b \in \integers_p\}$。
根據上一個練習中 $\epsilon$ 全域的定義,
證明 ${\cal H}$ 是 $((d-1)/p)$ 全域的。
(\hint 參考\inexercise[model_zero])
\stopEXERCISE

\startANSWER
令 $k=\langle k_0,k_1,\ldots,k_{n-1}\rangle \in U$,
 $l=\langle l_0,l_1,\ldots,l_{n-1}\rangle \in U$,
且 $k_i \ne l_i$, $b\in \mathbb{Z}_p$。
則:
\startsplitformula\startmathalignment
\NC \Pr\{h_b(k)=h_b(l)\}
   = \NC \Pr\{h_b(k)-h_b(l)=0\} \NR
\NC= \NC \Pr\{\sum_{j=0}^{d-1}(k_j-l_j)b^j = 0\} \NR
\stopmathalignment\stopsplitformula
其中 $\sum_{j=0}^{n-1}(k_j-l_j)b^j = 0$ 爲 $d-1$ 次多項式,
最多有 $d-1$ 個解。
因此:
\startformula
\Pr\{h_b(k)=h_b(l)\} \le \frac{d-1}{p} \qquad b \in \integers_p
\stopformula
所以 ${\cal H}$ 是 $((d-1)/p)$ 全域的。
\stopANSWER

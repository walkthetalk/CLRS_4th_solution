\startPROBLEM
(The cost of restructuring red-black tree)
紅黑樹有 4 種基本的\emph{結構性修改}(structural modification)操作:
插入、刪除、旋轉及更改顏色。
我們已經看到 \ALGO{RB-INSERT} 和 \ALGO{RB-DELETE} 操作僅使用 $O(1)$ 次旋轉、
插入和刪除操作來維持紅黑樹的性質,
但可能會多次更改顏色。

% a
\startigBase[a]\startitem
設計一個 $n$ 節點的合法的紅黑樹,
使得調用 \ALGO{RB-INSERT} 添加第 $n+1$ 個節點
會引起 $\Omega(\lg n)$ 次顏色更改。
然後設計一個 $n$ 節點的合法紅黑樹,
使得調用 \ALGO{RB-DELETE} 刪除一個特定節點
會引起 $\Omega(\lg n)$ 次顏色更改。
\stopitem\stopigBase

\startANSWER
如果完全二叉樹最底層節點全部爲紅色,則插入新節點時,
需要 $\Omega(\lg n)$ 個 case 1 來改變節點的顏色。
如果完全二叉樹中節點全部爲黑色,則刪除節點所需時間爲 $\Omega(\lg n)$,
因爲 while 循環的每次迭代都會觸發 case 2。
\stopANSWER

雖然最壞情況下,每個操作更改顏色的次數可能是對數的,
但我們可以證明,在一個空紅黑樹上執行任意 $m$ 個操作序列
 \ALGO{RB-INSERT} 和 \ALGO{RB-DELETE},
最壞情況也只會引起 $O(m)$ 次結構性修改。
注意:每次更改顏色都算作一次結構性修改。

% b
\startigBase[continue]\startitem
\ALGO{RB-INSERT-FIXUP} 和 \ALGO{RB-DELETE-FIXUP} 的主循環
都處理一些\emph{終止}條件:
一旦條件得到滿足,會導致循環在常數次操作後終止。
對於 \ALGO{RB-INSERT-FIXUP} 和 \ALGO{RB-DELETE-FIXUP} 中處理的各種情況,
指出其中哪些會導致循環終止,哪些不會。
(\hint 參見圖 13-5、圖 13-6 和圖 13-7)
\stopitem\stopigBase

\startANSWER
插入操作中, case 2 和 case 3 會終止。
刪除操作中, case 1 和 case 3 會終止。
\stopANSWER

我們首先分析僅僅執行插入操作時引起的結構性修改。
另 $T$ 爲一棵紅黑樹,
定義 $\Phi(T)$ 是 $T$ 中紅節點數量。
假定一個單位的勢可以支付 \ALGO{RB-INSERT-FIXUP} 的三種情況
的任意一種所引起的結構性修改的代價。

% c
\startigBase[continue]\startitem
令 $T'$ 表示對 $T$ 應用 \ALGO{RB-INSERT-FIXUP} 時 case 1 所得結果。
證明: $\Phi(T')=\Phi(T) - 1$。
\stopitem\stopigBase

\startANSWER
在應用 case 1 後, $z$ 和父節點和叔父節點均變成黑色,
而 $z$ 的祖父節點變爲紅色。少了一個紅節點,
因此 $\Phi(T')=\Phi(T) - 1$。
\stopANSWER

% d
\startigBase[continue]\startitem
當使用 \ALGO{RB-INSERT} 向一棵紅黑樹中插入一個節點時,
我們可以將操作分解爲三部分。
列出 \ALGO{RB-INSERT} 的第 1 ~ 16 行引起的結構性改變和勢的變化,
以及 \ALGO{RB-INSERT-FIXUP} 中各種情況引起的變化。
\stopitem\stopigBase

\startANSWER
對於 case 1,紅色節點會減少一個,因此勢函數也會減一。
然而,調用一次 \ALGO{RB-INSERT-FIXUP} 會
導致 $\Omega(\lg n)$ 個 case 1。

對於 case 2 和 3,顏色保持不變,都會執行一次旋轉。
\stopANSWER

% e
\startigBase[continue]\startitem
用(d)證明:
任意一次 \ALGO{RB-INSERT} 所導致的結構性修改的攤還次數爲 $O(1)$。
\stopitem\stopigBase

\startANSWER
每個 case 1 都要求特定節點是紅色的,
紅色節點減少的個數不會超過 $\Phi(T)$。
因此勢函數永遠非負。
每次插入最多使紅色節點數目增一,
每個單位勢均能償還任一 case 中中任一結構性修改。
最壞情況下,調用 \ALGO{RB-INSERT} 會執行 $k$ 次 case 1,
其中 $k$ 等於 $\Phi(T_i)-\Phi(T_{i-1})$。
因此,總攤還代價有上界 $2(\Phi(T_n)-\Phi(T_0))\le n$,
所以,每次插入的攤還代價爲 $O(1)$。
\stopANSWER

我們現在希望證明既有插入又有刪除時,所引起的結構性修改次數仍爲 $O(m)$。
對每個節點 $x$,我們定義:
\startformula
\omega(x) = \startcases
\NC 0 \NC \text{若 $x$ 是紅節點} \NR
\NC 1 \NC \text{若 $x$ 是黑節點且沒有紅孩子} \NR
\NC 0 \NC \text{若 $x$ 是黑節點且有一個紅孩子} \NR
\NC 2 \NC \text{若 $x$ 是黑節點且由兩個紅孩子} \NR
\stopcases
\stopformula
現在定義紅黑樹 $T$ 的勢函數爲:
\startformula
\Phi(T) = \sum_{x\in T}\omega(x)
\stopformula
且令 $T'$ 爲對 $T$ 應用 \ALGO{RB-INSERT-FIXUP} 或
 \ALGO{RB-DELETE-FIXUP} 的任意非終結性情況後的結果。

% f
\startigBase[continue]\startitem
證明:
對 \ALGO{RB-INSERT-FIXUP} 的任意非終結性情況,
有 $\Phi(T')\le \Phi(T) - 1$。
證明: \ALGO{RB-INSERT-FIXUP} 的任意一次調用
所引起的結構性修改的攤還次數爲 $O(1)$。
\stopitem\stopigBase

\startANSWER
\ALGO{RB-ISNERT} 時,如果是 case 1,
有兩個紅孩子的黑節點會少一個,
沒有紅孩子的黑節點最多多一個,
從而勢函數最多減一。
新的勢函數, $\Phi(T_n)-\Phi(T_0)\le n$。
由於每個操作需要一個單位的勢,
且終止條件會引發常數次結構性修改,
而總攤還代價爲 $O(n)$,
因此每個 \ALGO{RB-INSERT-FIXUP} 的攤還代價爲 $O(1)$。
\stopANSWER

% g
\startigBase[continue]\startitem
證明:
對 \ALGO{RB-DELETE-FIXUP} 的任意非終結性情況,
有 $\Phi(T')\le \Phi(T) - 1$。
證明: \ALGO{RB-DELETE-FIXUP} 的任意一次調用
所引起的結構性修改的攤還次數爲 $O(1)$。
\stopitem\stopigBase

\startANSWER
\ALGO{RB-DELETE} 時,如果是 case 2,
有兩個紅孩子的黑節點會少一個,
從而勢函數減 2。
由於勢至少減 1,所以就償還了結構性修改的代價。
其他 case 會引發常數次結構性修改,
而總攤還代價又是 $O(n)$,
因此每個 \ALGO{RB-DELETE-FIXUP} 的攤還代價爲 $O(1)$。
\stopANSWER

% h
\startigBase[continue]\startitem
證明:
由任意 $m$ 個 \ALGO{RB-INSERT-FIXUP} 和
 \ALGO{RB-DELETE-FIXUP} 構成的序列最壞情況下執行 $O(m)$ 次結構性修改。
\stopitem\stopigBase

\startANSWER
根據上面所述,無論是插入還是刪除,也無論哪種情況,
只要是非終止,勢函數都會償還所引發的變化。
如果是終止,則他們所用時間本身就是常數。
所以如果操作序列中含有 $m$ 個插入和刪除操作,
其中任一操作的攤還代價都是 $O(1)$,
從而總攤還代價爲 $O(m)$。
\stopANSWER

\stopPROBLEM

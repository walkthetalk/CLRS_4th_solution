\startPROBLEM
(making binary search dynamic)
有序數列上的二分查找花費對數時間,
但插入一個新元素的時間與數列規模呈線性關係。
我們可以通過維護多個有序數列來提高插入性能。

具體地,假定我們希望支持 $n$ 元集合上的 \ALGO{SEARCH} 和 \ALGO{INSERT} 操作。
令 $k=\lceil \lg{(n+1)}\rceil$,
令 $n$ 的二進制表示爲 $\langle n_{k-1},n_{k-2},\ldots,n_0\rangle$。
我們維護 $k$ 個有序數列 $A_0,A_1,\ldots,A_{k-1}$,
對 $i=0,1,\ldots,{k-1}$,數列 $A_i$ 的長度爲 $2^i$。
每個數列或滿或空,
取決於 $n_i=1$ 還是 $n_i=0$。
因此,所有 $k$ 個數列中保存的元素總數爲 $\sum_{i=0}^{k-1}n_i 2^i = n$。
雖然單獨每個數列都是有序的,
但不同數列中的元素之間不存在特定的大小關係。

% a
\startigBase[a]\startitem
設計算法,實現這種數據結構上的 \ALGO{SEARCH} 操作,分析其最壞情況運行時間。
\stopitem\stopigBase

\startANSWER
對於每個數列都用二分查找進行搜索,直到找到想要的元素。
最壞情況下需要搜索所有數列:
\startformula\startmathalignment
\NC T(n) \NC = \Theta(\lg 1 + \lg 2 + \lg 2^2 + \ldots + \lg 2^{k-1}) \NR
\NC \NC = \Theta(0+1+\ldots+(k-1)) \NR
\NC \NC = \Theta(\frac{1}{2}k(k-1)) \NR
\NC \NC = \Theta(\lg^2 n) \NR
\stopmathalignment\stopformula

從按數列大小遞減順序搜索,在某些情況下性能會更好一些,但不影響最壞情況的結果。
\stopANSWER

% b
\startigBase[a,continue]\startitem
設計 \ALGO{INSERT} 算法。分析最壞情況運行時間和攤還時間。
\stopitem\stopigBase

\startANSWER
找到最小的空數列 \m{A_i},將新元素插入 \m{A_i},
並將 \m{A_0,A_1,\ldots,A_{i-1}} 中所有元素全部轉移到 \m{A_i} 中。
最壞情況下,需要移動所有 \m{n+1} 個元素。

用聚合法分析攤還代價。
對於 \m{n} 次插入操作,數列 \m{A_i} 會改變 \m{n/2^i} 次,
此處所謂改變,指數列由空變滿,或由滿變空。
因此:
\startformula
\sum_{i=0}^{k-1}\frac{n}{2^i}2^i = \sum_{i=0}^{k-1}n = nk \in O(n\lg n)
\stopformula
即攤還代價爲 $O(\lg n)$。
\stopANSWER

% c
\startigBase[a,continue]\startitem
討論如何實現 \ALGO{DELETE}。
\stopitem\stopigBase

\startANSWER
\ALGO{DELETE} 花的時間不可能比 \ALGO{SEARCH} 更少。
實際上, \ALGO{DELETE} 最快也需要線性時間,除非我們以其他方式描述元素間的關係。
比如,刪除最長數列 $A_{k-1}$ 正中間的元素。
刪除後,至少得有一個元素 $e$ 從其他數列中轉移到 $A_{k-1}$ 中。
但由於我們不知道 $e$ 與 $A_{k-1}$ 中其他元素的關係,
重排 $A_{k-1}$ 需要時間爲 $2^{k-1}\in O(n)$。
由於每次刪除都需要這麼多時間,攤還代價也是 $O(n)$。

找到最小非空數列 \m{A_i},並找到要刪除的元素。
假定要刪除的元素在數列 \m{A_j} 中,將其刪除。
從 \m{A_i} 中轉移一個元素到 \m{A_j} 中(如果 \m{i=j},則保持不變)。
然後將 \m{A_i} 中剩餘元素轉移到 \m{A_0,A_1,\ldots,A_{i-1}} 中。
最後,重排 \m{A_j}。
\stopANSWER

\stopPROBLEM

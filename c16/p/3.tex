\startPROBLEM
(amortized weight-balanced tree)
考慮擴充普通二叉搜索樹,
爲每個節點 $x$ 增加屬性 $x.size$,
此屬性給出了根爲 $x$ 的子樹中關鍵字的數量。
另 $\alpha$ 是 $1/2\le \alpha < 1$ 之間的一個常數。
當 $x.left.size\le \alpha\cdot x.size$ 且 $x.right.size\le \alpha\cdot x.size$ 時,
我們稱節點 $\alpha$ 是 \emph{$\alpha$ 平衡的}。
如果樹中每個節點都是 $\alpha$ 平衡的,
則稱樹整體是 \emph{$\alpha$ 平衡的}。
 G.Varghese 提出了如下攤還方法來維護加權平衡樹。

% a
\startigBase[a]\startitem
在某種意義上,一棵 $1/2$ 平衡樹達到了極限的平衡。
給定任意一棵二叉搜索樹中的一個節點 $x$,
證明:如何重建以 $x$ 爲根的子樹,
使得他變爲 $1/2$ 平衡的。
你的算法運行時間應爲 $\Theta(x.size)$,
可以使用 $O(x.size)$ 的輔助空間。
\stopitem\stopigBase

\startANSWER
首先中序遍歷子樹,並按順序將所有元素存儲在輔助空間中。
然後選擇數列中中間元素作爲子樹的根,並依此遞迴構建其左、右兩棵子樹。
所用時間以及所用輔助空間均爲 $O(x.size)$。
\stopANSWER

% b
\startigBase[continue]\startitem
證明:在一棵 $n$ 個節點的 $\alpha$ 平衡二叉搜索樹
中執行一次搜索操作的最壞情況時間爲 $O(\lg n)$。
\stopitem\stopigBase

\startANSWER
令 $f(n)$ 爲所需時間,則 $f(n)\le f(\alpha n) + O(1)$。
求解得:
\startformula
f(n)\in O(\lg_{1/\alpha}n) = O(\lg n)
\stopformula
\stopANSWER

對於本問題的剩餘部分,假定常數 $\alpha$ 嚴格大於 $1/2$。
假定你實現的 \ALGO{INSERT} 和 \ALGO{DELETE} 算法
與普通 $n$ 節點二叉搜索樹的算法是一樣的,
差別僅在於,如果發現樹中任何節點不再是 $\alpha$ 平衡的,
則在最高的不平衡節點,
對以他爲根的子樹執行“重建”,
使其變爲 $1/2$ 平衡的。

我們用勢能法分析此重建方法。
對於二叉搜索樹 $T$ 中的一個節點 $x$,定義:
\startformula
\Delta(x) = |x.left.size - x.right.size|
\stopformula
定義 $T$ 的勢函數爲:
\startformula
\Phi(T) = c \sum_{x\in T: \Delta(x)\ge 2}\Delta(x)
\stopformula
其中 $c$ 是一個足夠大的常數,他依賴於 $\alpha$。

% c
\startigBase[continue]\startitem
證明:任意二叉搜索樹的勢都是非負的, $1/2$ 平衡樹的勢爲 $0$。
\stopitem\stopigBase

\startANSWER
略。
\stopANSWER

% d
\startigBase[continue]\startitem
假定 $m$ 個單位的勢夠支付重建 $m$ 節點子樹的代價。
相對於 $\alpha$ 來說, $c$ 應該多大才能使得
重建一棵非 $\alpha$ 平衡的子樹攤還時間爲 $O(1)$。
\stopitem\stopigBase

\startANSWER
重建前:
\startsplitformula\startmathalignment
\NC \Delta(x)
   \NC = |x.left.size - x.right.size| \NR
\NC\NC \ge \alpha \cdot x.size - (1-\alpha)\cdot x.size \NR
\NC\NC = (2\alpha - 1)\cdot x.size \NR
\stopmathalignment\stopsplitformula
即重建前, $\Phi(T) \ge c \cdot \Delta(x) \ge c\cdot (2\alpha - 1)\cdot x.size$。

而重建後 $\Phi(T) = 0$。
因此必須滿足 $\Phi(T) \ge x.size$,其中等式右側爲重建代價。

如果可以保證 $c\dot(2\alpha-1) \cdot x.size \ge x.size$,
則就可以滿足要求。
求解得:
\startformula
c\ge \frac{1}{2\alpha -1}
\stopformula
\stopANSWER

% e
\startigBase[continue]\startitem
證明:在一棵 $n$ 節點的 $\alpha$ 平衡樹中插入一個節點
或刪除一個節點的攤還時間爲 $O(\lg n)$。
\stopitem\stopigBase

\startANSWER
令攤還代價爲 $\hat{c_i}$。
如果沒有觸發重構,則:
\startformula
\hat{c_i} = c_i + \Phi(D_i) - \Phi(D_{i-1})
\stopformula
由 (b) 可知, $c_i \in O(\lg n)$。
類似,樹高也是 $O(\lg n)$。
無論是插入還是刪除節點 $v$ 都會改變從他到根節點路徑上所有節點的 $\Delta(x)$。
因此:
\startformula
\Phi(D_i) - \Phi(D_{i-1}) \le c\cdot O(\lg n) \in O(\lg n)
\stopformula

如果觸發了重構,令 $c_i = c_i' + c_i''$,
其中 $c_i'$ 是插入或刪除的代價,而 $c_i''$ 是重構的代價。
令 $D_i'$ 是插入或刪除之後、重構之前的狀態。那麼:
\startformula
\hat{c_i}=c_i' + c_i'' + (\Phi(D_i) - \Phi(D_i')) + (\Phi(D_i') - \Phi(D_{i-1}))
\stopformula

$\Phi(D_i')$ 是重構之前的勢, $\Phi(D_i)$ 是刪除之後的勢,
 $\Phi(D_i')-\Phi(D_i)$ 是重構後減少的勢。根據上一個問題:
\startformula
c_i'' \le \Phi(D_i')-\Phi(D_i)
\stopformula
因此:
\startsplitformula\startmathalignment
\NC \hat{c_i} \NC = c_i' + c_i'' + (\Phi(D_i) - \Phi(D_i')) + (\Phi(D_i') - \Phi(D_{i-1})) \NR
\NC \NC \le O(\lg n) + c_i'' - (\Phi(D_i')) - \Phi(D_i)) + O(\lg n) \NR
\NC \NC \le O(\lg n) \NR
\stopmathalignment\stopsplitformula

\stopANSWER

\stopPROBLEM

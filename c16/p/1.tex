\startPROBLEM
(Binary reflected Gray code)
\emph{二進制格雷碼}(binary Gray code)即用二進制表示的一組非負整數,
其中任一整數與下一個相比僅一位不同。
\emph{二進制反射格雷碼}(binary reflected Gray code)則是這樣一個整數序列,
其值從从 0 到 $2^k - 1$,對於正整數 $k$,遵循下列遞迴方法:

\startigBase
\item 若 $k=1$,則二進制反射格雷碼爲 $\langle 0, 1\rangle$。

\item 若 $k\ge 2$,則先用 $0$ 到 $2^{k-1}-1$ 共 $2^{k-1}$ 個數
給出 $k-1$ 的二進制反射格雷碼。接着對這個序列進行反射,即逆序。
也就是說,序列中第 $j$ 個整數反射後是第 $2^{k-1}-j-1$ 個整數。
反射後的序列中有 $2^{k-1}$ 個數,爲每個數加上 $2^{k-1}$。
最後,連接 $k-1$ 的二進制反射格雷碼和新序列。
\stopigBase

例如,對於 $k=2$,首先 $k=1$ 的二進制反射格雷碼是 $\langle 0,1\rangle$。
反射後的序列是 $\langle 1,0\rangle$,
加上 $2^{k-1}=2$ 後變成 $\langle 3,2\rangle$.
連接兩個序列後就成了 $\langle 0,1,3,2\rangle$,
二進制形式爲 $\langle 00,01,11,10\rangle$。
其中每個整數與前一個僅有一位不同。
對於 $k=3$,將 $k=2$ 的二進制反射格雷碼反射後變成 $\langle 2,3,1,0\rangle$,
加上 $2^{k-1}=4$ 後變成 $\langle 6,7,5,4\rangle$.
連接兩個序列後就成了 $\langle 0,1,3,2,6,7,5,4\rangle$,
二進制形式爲 $\langle 000,001,011,010,110,111,101,100\rangle$。
即使整個序列是循環的,相鄰兩個整數也只有一位翻轉。

\startigBase[a]\startitem
給二進制反射格雷碼中的整數進行編號,從 $0$ 到 $2^k-1$,
並考慮其中第 $i$ 個整數。
從其中第 $i-1$ 個整數到第 $i$ 個整數,只有一位翻轉了。
如何確定到底是哪一位翻轉了,假設是第 $i$ 個整數。
\stopitem\stopigBase

\startANSWER
首先,遞迴時,加整數的步驟,僅使得兩個序列連接部分的最高位發生翻轉,
即 $i=2^{k-1}$ 時,第 $k-1$ 位發生翻轉。
由於加整數僅改變了最高位,不會影響其他位,因此所連接的兩個序列(除最高位)可以認爲是對稱的。
令 $j=2^{k-1}-1$,即左半個序列的最後一個數,
令 $k=2^{k-1}+1$,即右半個序列的第二個數,
這兩個數相對各自的前一個數,翻轉的是同一位。
以此類推,第 $2^{k}-1$ 和數和第二個數所翻轉的是同一位。

最終我們可以用二進制表示 $i$,
第 $i$ 個數所翻轉的位就是從低位往高位數,第一個爲 $1$ 的位。
也可以是前一個數第一個爲 $0$ 的位(也是從低位往高位數)。
\stopANSWER

\startigBase[a,continue]\startitem
假設可以在常數時間內翻轉第 $j$ 位,如何在 $\Theta(2^k)$ 時間內
算出所有 $2^k$ 個數的二進制反射格雷碼。
\stopitem\stopigBase

\startANSWER
要達到目的,必須在常數時間內爲第 $i$ 個數計算出要翻轉哪一位。

\startCLRSCODE
s = 2^k
g = \ALGO{NEW-ARRAY(s)}
g[0] = 0
for i = 0 to s - 2
	v = (i\oplus(i+1) + 1) / 2
	g[i+1] = g[i] \oplus v
\stopCLRSCODE
\stopANSWER

\stopPROBLEM

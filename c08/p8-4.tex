\startPROBLEM
(Water jugs)
假設有 $n$ 個紅色水壺和 $n$ 個藍色水壺。
他們的形狀和大小各不相同。
所有的紅色水壺的容積都不一樣,
藍色水壺亦是如此。
壺的容積與壺的大小沒有關係。
而且,對於任一個水壺來說,
肯定有一個跟他顏色不同的水壺,容積一樣。

你的任務是將紅水壺和藍水壺配對,要求所盛水一樣多。
爲此,可以這樣做:
挑出一對水壺,其中一個是紅色的,一個是藍色的,
將紅色水壺中倒滿水,再將水倒入藍色的水壺中。
通過這一操作,可以判斷出這兩個水壺容積是否相同。
假設這樣的比較需要花費一個單位時間。
你的目標是找出一個算法,用最少的比較次數進行配對。
注意,你不能直接比較兩個紅色或兩個藍色的水壺。
% a
\startigBase[a]\startitem
設計一個確定性算法,能用 $\Theta(n^2)$ 次比較來完成所有水壺的配對。
\stopitem\stopigBase

\startANSWER
任取一紅色水壺,將其與所有藍色水壺比較,直至找到與之配對的藍色水壺。
一旦爲所有紅色水壺都找到了與之配對的藍色水壺,算法終止。
比較次數最多爲:
\startformula
\sum_{i=1}^{n-1}i = \frac{n(n-1)}{2} = O(n^2)
\stopformula
\stopANSWER

% b
\startigBase[a,continue]\startitem
證明:解決該問題所需比較次數的下界爲 $\Omega(n\lg{n})$。
\stopitem\stopigBase

\startANSWER
参考\insection[lower_bound_sorting]。
先將紅藍水壺各自排序。
则一种配对方案相當於一个 $n$ 项排列,
第 $i$ 個藍水壺與第 $i$ 個紅水壺進行配對。
利用決策樹對紅藍水壺進行比較。
每個內部節點都代表對某個紅水壺和某個藍水壺進行比較,
葉子節點表示根據比較結果對藍水壺的一種排列。
比較結果可能是小於、等於、大於三種情況,
即决策树上的每个內部节点會有三个子节点,即\emph{三叉树}。

由于有 $n!$ 种结果,每种结果都是决策树中的一个叶子节点。
假如树中共有 $l$ 个节点,樹高 $h$:
\startformula
n! \le l \le 3^h
\stopformula
则:
\startformula
h \ge \lg_{3}(n!) = \Omega(n\lg{n})
\stopformula
\stopANSWER

% c
\startigBase[a,continue]\startitem
設計一個隨機算法,比較次數的期望值為 $O(n\lg{n})$,
並證明這個界的正確性。
此算法在最壞情況下的比較次數是多少?
\stopitem\stopigBase

\startANSWER
可以將快速排序隨機化。以如下步驟進行劃分:
\startigBase[n]
\item 隨機選擇一個紅色水壺作為紅色主元; $O(1)$
\item 以紅色主元劃分藍色水壺; $O(n)$
\item 在上一步找到與紅色主元配對的藍色主元並將其歸位-; $O(1)$
\item 以藍色主元劃分紅色水壺,並將紅色主元歸位。 $O(n)$
\stopigBase

最壞情況下比較次數爲 $\sum_{i=2}^{n}(i+i-1) = n^2$。
\stopANSWER

\stopPROBLEM

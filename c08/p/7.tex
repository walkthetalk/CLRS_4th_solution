\startPROBLEM
(The 0-1 sorting lemma and columnsort)
對兩個數列元素 $A[i]$ 和 $A[j]$ (其中 $i < j$)進行\emph{比較交換}
(\emph{compare-exchange}),如下所示:

\CLRSH{COMPARE-EXCHANGE(A, i, j)}
\startCLRSCODE
if A[i] > A[j]
	\ALGO{SWAP(A[i],A[j])}
\stopCLRSCODE
比較交換後,滿足 $A[i] \le A[j]$。

\emph{遺忘比較交換算法(oblivious compare-exchange algorithm)}
指算法只按照事先定義好的比較交換操作執行。
所比較元素的下標必須事先確定好,
這些下標可以依賴待排序元素個數,
卻不能依賴待排序元素的值,
也不能依賴之前比較交換操作的結果。
下面就是一個經過遺忘比較交換改造的插入排序
(此算法在所有情況下運行時間都是 $\Theta(n^2)$,與插入排序不同):

\CLRSH{COMPARE-EXCHANGE-INSERTION-SORT(A,n)}
\startCLRSCODE
for i = 2 to n
	for j = i - 1 downto 1
		\ALGO{COMPARE-EXCHANGE(A, j, j + 1)}
\stopCLRSCODE

\emph{0-1 排序引理(0-1 sorting lemma)}可以有力的證明
遺忘比較交換算法的排序結果是正確的。
該引理表明,如果遺忘比較交換算法能夠對所有
只包含 $0$ 和 $1$ 的序列正確排序,
那麼他就可以對包含任意值的序列正確排序。

你可以用反例來證明 0-1 排序引理:
如果遺忘比較交換算法不能對包含任意值的序列進行排序,
那麼他也不能對某個 0-1 序列進行排序。
不妨假設遺忘比較交換算法 $X$ 未能對數列 $A[1..n]$ 排序。
設 $A[p]$ 是算法 $X$ 將其放到錯誤位置上的最小元素,
而 $X$ 將 $A[q]$ 放在了 $A[p]$ 應該所在的位置上。
定義一個只包含 $0$ 和 $1$ 的數列 $B[1..n]$ 如下:
\startformula
B[i] = \startmathcases
\NC 0 \NC \text{若 $A[i]\le A[p]$,} \NR
\NC 1 \NC \text{若 $A[i] > A[p]$。} \NR
\stopmathcases
\stopformula

% a
\startigBase[a]\startitem
討論:$A[q]>A[p]$,且 $B[p]=0$, $B[q]=1$。
\stopitem\stopigBase

\startANSWER
$A[p]$ 和 $A[q]$ 所放位置均不正確,
且 $A[p]$ 是位置不正確的最小元素,
因此 $A[q] \ge A[p]$。
而如果 $A[q]=A[p]$,就不能說 $A[p]$ 放錯位置了。
所以 $A[q] > A[p]$。

將 $i=p,q$ 代入即得 $B[p]=0$ 和 $B[q]=1$。
\stopANSWER

% b
\startigBase[a,continue]\startitem
爲完成 0-1 排序引理的證明,
請先證明算法 $X$ 不能對數列 $B$ 正確排序。
\stopitem\stopigBase

\startANSWER
經過 $i$ 步操作, $A$ 變爲 $A_i$, $B$ 變爲 $B_i$。
因此 $A=A_0$,且 $B=B_0$。
對於所有 $i$ 和 $j$,我們都可以維持如下不變式:
\startformula
A_i[j] >A[p] \Leftrightarrow B_i[j] > 0 = B[p]
\stopformula

初始化:($i=0$)根據定義,
對於所有比 $A[p]$ 大的元素, $B$ 中對應位置上都是 $1$,
相應的,所有不比 $A[p]$ 大的元素, $B$ 中對應位置上都是 $0$。
假設在步驟 $i+1$ 中,我們對位置 $j_1$ 和 $j_2$ ($j_1<j_2$)上的元素執行了比較交換。
也就是說 $B_i[j_1]>B_i[j_2]$,
根據假設,不可能有 $A_i[j_1]\le A_i[j_2]$,
因爲 $B$ 中的元素關係就意味着 $A_i[j_1]>A[p]\ge A_i[j_2]$。
分情況討論:

1. $A_i[j_1]\le A_i[j_2]$,且 $B_i[j_1]\le B_i[j_2]$。
兩個數列均不會發生交換,即 $A_{i+1}=A_i$ 且 $B_{i+1}=B_i$。
顯然,所期望的屬性仍然成立。

2. $A_i[j_1]> A_i[j_2]$,且 $B_i[j_1]\le B_i[j_2]$。
 $A$ 會發生交換但 $B$ 不會。
 $A_i[j_1]$ 和 $A_i[j_2]$ 肯定都大於 $A[p]$ 或者都小於等於 $A[p]$。
否則 $B_i[j_1]\ne B_i[j_2]$,
即 $0=B_i[j_1]<B_i[j_2]=1$,
即 $A_i[j_1]\le A[p]<A_i[j_2]$,與假設矛盾。
如果 $A_i[j_1]$ 和 $A_i[j_2]$ 都大於 $A[p]$,
則 $B_i[j_1]=B_i[j_2]=1$;
而如果 $A_i[j_1]$ 和 $A_i[j_2]$ 都小於等於 $A[p]$,
則 $B_i[j_1]=B_i[j_2]=0$。
即 $A_{i+1}[j_2]=A_i[j_1] > A[p]$,
且 $A_{i+1}[j_1]=A_i[j_2] > A[p]$。

3. $A_i[j_1]> A_i[j_2]$,且 $B_i[j_1]> B_i[j_2]$。
 $A$ 和 $B$ 都會發生交換。
如果 $j$ 既不是 $j_1$ 也不是 $j_2$,則無需移動,不用考慮。
對於 $j_2$,有:
\startsplitformula\startmathalignment[n=3,align={right,middle,left}]
\NC A_i[j_1] > A[p] \NC \Leftrightarrow \NC B_i[j_1] > 0 \NR
\NC A_{i+1}[j_2] > A[p] \NC \Leftrightarrow \NC B_{i+1}[j_2] > 0 \NR
\stopmathalignment\stopsplitformula
對於 $j_1$ 也是一樣。不變式證明完畢。

綜上所述,令 $i$ 是操作的總數目,
令 $j$ 爲 $A[p]$ 的最終位置,
令 $k$ 爲 $A[p]$ 的正確位置。
因爲 $A[p]$ 是放置錯誤的最小元素,所以 $k<j$,
否則他所佔位置屬於一個更小的元素。
如果他所佔位置屬於一個與他相等的元素,
那麼不能說他的位置不對。
根據定義, $A[q]=A_i[i]$, $A[p]=A_i[j]$。
根據上一項,可知 $A_i[j]=A[p]$。
根據不變式,我們證明了 $B_i[j]>0$ 且 $B[k]=0$。
因此 $B$ 沒有正確排序。
\stopANSWER

現在,需要用 0-1 排序引理來證明一個特殊的排序算法的正確性。
\emph{列排序}算法用於對矩形數列進行排序。
這一矩形數列有 $r$ 行 $s$ 列(因此 $n=rs$),
滿足下列三個限制條件:
\startigNum[2]
\item $r$ 必須是偶數;
\item $s$ 必須是 $r$ 的因子;
\item $r\ge 2s^2$;
\stopigNum
當排序完成時,矩形數列是\emph{列優先有序}的:
按照列從上到下,從左到右,都是單調遞增的。

無論 $n$ 的值如何,列排序需要 8 步。
所有奇數步都一樣:對每一列單獨進行排序。
每一個偶數步是一個特定的排列。具體如下:
\startigNum[n]
\item 對每一列進行排序;
\item 將矩形數列轉置,並重整爲 $r$ 行 $s$ 列的形式。
也就是說,首先將最左邊的一列按順序放入前 $r/s$ 行,
然後將下一列放在第二個 $r/s$ 行,以此類推;
\item 對每一列進行排序;
\item 執行第 2 步排列操作的逆操作;
\item 對每一列進行排序;
\item 將每一列的上半部分移到同一列的下半部分位置,
將每一列的下半部分移到下一列的上半部分,
並將最左邊一列的上半部分置空。
此時,最後一列的下半部分成爲新的最右列的上半部分,新的最右列的下半部分爲空;
\item 對每一列進行排序;
\item 執行第 6 步排列操作的逆操作。
\stopigNum

可以將第 6~8 步視爲一步,
對每列的下半部及下一列的上半部進行排序。
下圖是對 $r=6,s=3$ 矩形數列的排序過程
(即使違背了 $r\ge 2s^2$ 的條件,算法仍然有效)。

\startplacefloat[][location={none}]
\startcombination[nx=5,ny=2, distance=3em, align=middle]
{
\startmatrix
\NC 10 \NC 14 \NC  5 \NR
\NC  8 \NC  7 \NC 17 \NR
\NC 12 \NC  1 \NC  6 \NR
\NC 16 \NC  9 \NC 11 \NR
\NC  4 \NC 15 \NC  2 \NR
\NC 18 \NC  3 \NC 13 \NR
\stopmatrix
}{(a)}{
\startmatrix
\NC  4 \NC  1 \NC  2 \NR
\NC  8 \NC  3 \NC  5 \NR
\NC 10 \NC  7 \NC  6 \NR
\NC 12 \NC  9 \NC 11 \NR
\NC 16 \NC 14 \NC 13 \NR
\NC 18 \NC 15 \NC 17 \NR
\stopmatrix
}{(b)}{
\startmatrix
\NC  4 \NC  8 \NC 10 \NR
\NC 12 \NC 16 \NC 18 \NR
\NC  1 \NC  3 \NC  7 \NR
\NC  9 \NC 14 \NC 15 \NR
\NC  2 \NC  5 \NC  6 \NR
\NC 11 \NC 13 \NC 17 \NR
\stopmatrix
}{(c)}{
\startmatrix
\NC  1 \NC  3 \NC  6 \NR
\NC  2 \NC  5 \NC  7 \NR
\NC  4 \NC  8 \NC 10 \NR
\NC  9 \NC 13 \NC 15 \NR
\NC 11 \NC 14 \NC 17 \NR
\NC 12 \NC 16 \NC 18 \NR
\stopmatrix
}{(d)}{
\startmatrix
\NC  1 \NC  4 \NC 11 \NR
\NC  3 \NC  8 \NC 14 \NR
\NC  6 \NC 10 \NC 17 \NR
\NC  2 \NC  9 \NC 12 \NR
\NC  5 \NC 13 \NC 16 \NR
\NC  7 \NC 15 \NC 18 \NR
\stopmatrix
}{(e)}{
\startmatrix
\NC  1 \NC  4 \NC 11 \NR
\NC  2 \NC  8 \NC 12 \NR
\NC  3 \NC  9 \NC 14 \NR
\NC  5 \NC 10 \NC 16 \NR
\NC  6 \NC 13 \NC 17 \NR
\NC  7 \NC 15 \NC 18 \NR
\stopmatrix
}{(f)}{
\startmatrix
\NC    \NC  5 \NC 10 \NC 16 \NR
\NC    \NC  6 \NC 13 \NC 17 \NR
\NC    \NC  7 \NC 15 \NC 18 \NR
\NC  1 \NC  4 \NC 11 \NC    \NR
\NC  2 \NC  8 \NC 12 \NC    \NR
\NC  3 \NC  9 \NC 14 \NC    \NR
\stopmatrix
}{(g)}{
\startmatrix
\NC    \NC  4 \NC 10 \NC 16 \NR
\NC    \NC  5 \NC 11 \NC 17 \NR
\NC    \NC  6 \NC 12 \NC 18 \NR
\NC  1 \NC  7 \NC 13 \NC    \NR
\NC  2 \NC  8 \NC 14 \NC    \NR
\NC  3 \NC  9 \NC 15 \NC    \NR
\stopmatrix
}{(h)}{
\startmatrix
\NC  1 \NC  7 \NC 13 \NR
\NC  2 \NC  8 \NC 14 \NR
\NC  3 \NC  9 \NC 15 \NR
\NC  4 \NC 10 \NC 16 \NR
\NC  5 \NC 11 \NC 17 \NR
\NC  6 \NC 12 \NC 18 \NR
\stopmatrix
}{(i)}
\stopcombination
\stopplacefloat

% c
\startigBase[a,continue]\startitem
討論:即使不知道奇數步採用了什麼排序算法,
我們也可以把列排序看作一種遺忘比較交換算法。
\stopitem\stopigBase

\startANSWER
偶數步根本就不關心元素的值。
\stopANSWER

似乎很難讓人相信列排序也能實現排序,
不過你可以利用 0-1 排序引理來證明這一點。
之所以可以使用 0-1 排序引理,
是因爲列排序可以看作是一種遺忘比較交換算法。
下面一些定義有助於你試用這一引理。
如果數列仲的某個區域只包含全 0 或者全 1,
或者這個區域中沒有任何元素,
我們定義這個區域是\emph{乾淨}的;
否則稱這個區域是\emph{髒}的。
這裏,假設輸入數據只包含 0 和 1,
且能轉換成 $r$ 行 $s$ 列。
% d
\startigBase[continue]\startitem
證明:經過第 1~3 步,數列由三部分組成:
頂部一些由全 $0$ 組成的乾淨行,
底部一些由全 1 組成的乾淨行,
以及中間最多 $s$ 行髒的行。
\stopitem\stopigBase

\startANSWER
第一步後,每一列均爲連續 0 後緊跟連續 1,
只有一處 0 到 1 的跳變。
由於 $s$ 整除 $r$,
第二步會將第一步的每一列轉換爲 $r/s$ 行。
在這 $r/s$ 行中最多只有一行有 0 到 1 的跳變,
其他行都是 0 或者都是 1。
即每一列所轉換的行中最多有一行是髒的。

第三步會將所有 0 行上移,所有 1 行下移,即最多只會剩下 $s$ 行髒的行。
\stopANSWER

% e
\startigBase[continue]\startitem
證明:經過第 4 步後,如果按列優先讀取數列,
先讀到的是全 0 的乾淨區域,最後是全 1 的乾淨區域,
中間是由最多 $s^2$ 個元素組成的髒區域。
\stopitem\stopigBase

\startANSWER
第 4 步後按列優先讀數據,
相當於在第 3 步後按行優先讀數據,
參考上一項的證明,結論是顯然的。
\stopANSWER

% f
\startigBase[continue]\startitem
證明:第 5~8 步產生一個全排序的 0-1 輸出,
並得到結論:列排序可以正確地對任意輸入值排序。
\stopitem\stopigBase

\startANSWER
第四步後,髒序列中最多有 $s^2$ 個元素,爲一列元素的一半,
可能在一列中,也可能分散於相鄰兩列中。
左邊的所有列爲 0,右邊的所有列爲 1。
如果髒序列在同一列中,則第 5 步即可保證每列有序,後續步驟不受影響。
如果髒序列分散於相鄰兩列中,則第 6 步會將髒序列轉移到同一列中。
\stopANSWER

% g
\startigBase[continue]\startitem
現在假設 $s$ 不能被 $r$ 整除。
證明:經過第 1~3 步,數列的頂部有一些全 0 的乾淨行,
底部有一些全 1 的乾淨行,中間最多 $2s-1$ 行髒行。
那麼與 $s$ 相比,在 $s$ 不能被 $r$ 整除時,
 $r$ 至少要多大才能保證配序的正確性?
\stopitem\stopigBase

\startANSWER
如果 $s$ 不能整除 $r$,
一行可能不僅包含 0 到 1 的跳變,也包含 1 到 0 的跳變。
最多有 $s - 1$  行,即最多有 $2s-1$ 個髒區域。
 $r$ 至少要是 $2(2s-1)s$。
\stopANSWER

% h
\startigBase[continue]\startitem
對第 1 步做一個簡單修改,
使得我們可以在 $s$ 不能被 $r$ 整除時,也保證 $r\ge 2s^2$,
並證明修改後,列排序仍然正確。
\stopitem\stopigBase

\startANSWER
第一步中,可以用 $+\infty$ 補齊數列使得 $s$ 可以整除 $r$,
或者可以將數列截斷一小部分,對其獨立排序。後者會更有效,因爲無需移動數列。
其實這些均無必要,列排序對是否整除沒有嚴格要求。
\stopANSWER

\stopPROBLEM

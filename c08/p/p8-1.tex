\startPROBLEM
(Probabilistic lower bounds on comparison sorting)
在這一問題中,我們將證明:
對於給定的 $n$ 個互異的輸入元素,
任何確定或隨機的比較排序算法,
其運行時間都有基於概率的下界 $\Omega(n\lg{n})$。
首先分析一個確定的比較排序算法對 $A$ 進行排序,
其決策樹爲 $T_A$。
假設 $A$ 的所有排列機會均等。
% a
\startigBase[a]\startitem
假設 $T_A$ 的每個葉子節點都標有在隨機輸入情況下到達該節點的概率。
證明:恰有 $n!$ 個葉子節點標有 $1/n!$,其他的葉子節點標記爲 $0$。
\stopitem\stopigBase

\startANSWER
有 $n!$ 種排序結果,每一種都對應 $n!$ 種輸入中的一種。
每種排序結果都對應一個葉子節點,由於各種輸入的可能性相同,
到達具體排序結果的概率均爲 $1/n!$。
除了這 $n!$ 種結果所對應的葉子節點,
即使決策樹有其他有更多葉子節點,也無法到達。

當然以上僅憑直覺,正式的證明卻有點麻煩。
\stopANSWER

% b
\startigBase[a,continue]\startitem
令 $D(T)$ 表示決策樹 $T$ 的外部路徑長度,
即 $T$ 的所有葉子節點深度的和。
令 $T$ 的葉子節點數目 $k>1$,
 $LT$ 和 $RT$ 分別是 $T$ 的左子樹和右子樹。
證明:
\startformula
D(T) = D(LT) + D(RT) + k
\stopformula
\stopitem\stopigBase

\startANSWER
子樹上的葉子節點在原樹上也是葉子節點。
原樹上的葉子節點也是子樹上的葉子節點
(不是左子樹的葉子,就是右子樹的葉子)。
因此 $LEAVES(LT) + LEAVES(RT) = k$。
對於子樹上的每個葉子節點而言,在子樹上的深度比在原樹上的深度小 1。
因此:
\startsplitformula\startmathalignment
\NC D(T) \NC = D(LT) + LEAVES(LT) + D(RT) + LEAVES(RT) \NR
\NC      \NC = D(LT) + D(RT) + k \NR
\stopmathalignment\stopsplitformula
\stopANSWER

% c
\startigBase[a,continue]\startitem
對於葉子節點數目爲 $k>1$ 的所有決策樹 $T$,
令 $D(T)$ 的最小值爲 $d(k)$。
證明:
\startformula
d(k) = \min_{1\le i\le {k-1}}\left\{d(i)+d(k-i)+k\right\}
\stopformula
(\hint 找一棵能使 $d(k)$ 取得最小值的、有 $k$ 個葉子節點的決策樹 $T$。
設 $i_0$ 是 $LT$ 中的葉子節點數, $k-i_0$ 是 $RT$ 中的葉子節點數。)
\stopitem\stopigBase

\startANSWER
對於一棵具有 $k$ 個葉子節點的決策樹,
由上一項可知:
\startformula
D(T) = D(LT) + D(RT) + k
\stopformula
而左子樹上葉子節點數目有從 $1$ 到 $k-1$ 共 $k-1$ 種可能。
由 $d(k)$ 的定義即得所要證明的結論。
\stopANSWER

% d
\startigBase[a,continue]\startitem
證明:對於給定的 $k$ 和 $i$ ($k>1$ 且 $1\le i\le {k-1}$),
函數 $i\lg{i}+(k-i)\lg{k-i}$ 在 $i=k/2$ 處取得最小值,
並有結論 $d(k)=\Omega(k\lg{k})$。
\stopitem\stopigBase

\startANSWER
\startsplitformula\startmathalignment
\NC f(i) \NC= i\lg{i} + (k-i)\lg(k-i) \NR
\NC f'(i) \NC= \lg{i} + 1 - \lg(k-i) - 1 = \lg\frac{i}{k-i} \NR
\NC f'(i) = 0 \NC \Leftrightarrow \lg\frac{i}{k-i} = 0 \Rightarrow i/(k-i) = 1 \Rightarrow i = \frac k 2 \NR
\stopmathalignment\stopsplitformula

由於 $f'(i)$ 單調遞增,因此 $f(i)$ 在 $i=k/2$ 處取得局部最小值。

就直覺而言,應該是左右兩棵子樹大小相同時取得最小值。
\stopANSWER

% e
\startigBase[a,continue]\startitem
證明: $D(T_A)=\Omega(n!\lg(n!))$,
且對 $n$ 個元素進行排序平均所需時間爲 $\Omega(n\lg{n})$。
\stopitem\stopigBase

\startANSWER
在 $T_A$ 上有 $n!$ 個葉子節點,
由上一項可知 $D(n)>d(k)=\Omega(n!\lg(n!))$。
每種排列的概率都是 $1/n!$,
因此排序的期望時間爲:
\startformula
\frac{\Omega(n!\lg(n!))}{n!} = \Omega(\lg(n!)) = \Omega(n\lg{n})
\stopformula
\stopANSWER

現在考慮一個隨機化的比較排序 $B$。
通過引入兩種節點,
我們可以擴展決策樹模型來處理隨機化的情況。
這兩種節點是:
普通的比較節點和“隨機化”節點。
隨機化節點使用 $B$ 中的 \ALGO{RANDOM(1, r)} 進行隨機挑選。
該類節點有 $r$ 個子節點,
在算法執行過程中,選擇任一子節點的概率相等。
% f
\startigBase[a,continue]\startitem
證明:對任何隨機化比較排序算法 $B$,
總存在一個確定的比較排序算法 $A$,
其比較次數的期望值不多於 $B$ 的比較次數。
\stopitem\stopigBase

\startANSWER
隨機算法 $B$ 所對應的確定算法 $A$ 可以認爲只是進行了預先的“隨機”選擇。
我們可以通過將隨機節點替換爲我們所選的一個子節點,來構造確定算法的決策樹。
隨機節點的所有子樹的期望值要大於等於其中最小的那棵子樹。
由於確定算法時間代價爲 $\Omega(n\lg{n})$,
因此隨機算法的時間代價亦爲 $\Omega(n\lg{n})$。
\stopANSWER

\stopPROBLEM

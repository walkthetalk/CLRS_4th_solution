\startPROBLEM
(Average sorting)
有時我們並不需要一個數列完全有序,
而是要求數列部分元素的平均數是增大的。
更準確地說,如果對所有的 $i=1,2,\ldots,n-k$ 有下式成立,
我們就稱包含 $n$ 個元素的數列 $A$ 是 \emph{$k$-有序}($k$-sorted)的:
\startformula
\frac{\sum_{j=i}^{i+k-1}A[j]}{k}
 \le \frac{\sum_{j=i + 1}^{i+k}A[j]}{k}
\stopformula

% a
\startigBase[a]\startitem
一個數列是 $1$-有序的,表示什麼含義?
\stopitem\stopigBase

\startANSWER
將 $k=1$ 代入可得: $A[j] \le A[j+1]$,
其中 $1 \le j \le n-1$。即整個數列是完全升序的。
\stopANSWER

% b
\startigBase[a,continue]\startitem
給出對 $1,2,\ldots,10$ 的一個排列,
 $2$-有序但不是完全有序。
\stopitem\stopigBase

\startANSWER
2, 1, 4, 3, 6, 5, 8, 7, 10, 9。
\stopANSWER

% c
\startigBase[a,continue]\startitem
證明:一個包含 $n$ 個元素的數列是 $k$-有序的,
當且僅當所有 $i=1,2,\ldots,n-k$ 均滿足 $A[i]\le A[i+k]$。
\stopitem\stopigBase

\startANSWER
\startsplitformula\startmathalignment[n=1]
\NC \frac{\sum_{j=i}^{i+k-1}A[j]}{k} \le \frac{\sum_{j=i + 1}^{i+k}A[j]}{k} \NR
\NC \Updownarrow \NR
\NC \frac{A[i] + \sum_{j=i+1}^{i+k-1}A[j]}{k} \le
     \frac{\sum_{j=i+1}^{i+k-1}A[j] + A[i+k]}{k} \NR
\NC \Updownarrow \NR
\NC \frac{A[i]}{k} \le \frac{A[i+k]}{k} \NR
\NC \Updownarrow \NR
\NC A[i] \le A[i+k] \NR
\stopmathalignment\stopsplitformula
\stopANSWER

% d
\startigBase[a,continue]\startitem
設計一個算法,對一個包含 $n$ 個元素的數列進行 $k$ 排序,
要求運行時間爲 $O(n\lg(n/k))$。
\stopitem\stopigBase

\startANSWER
將數列劃分成 $k$ 個子數列,記爲 $A_s$,其中 $s=1,2,\ldots,k$。

而 $A_s[i] = A[s + i \times k]$,其中 $i=0,1,2,\ldots,n/k-1$。

只需將這 $k$ 個子數列排序即可。每個子數列元素個數爲 $n/k$,
總時間爲
\startformula
O((n/k)\lg(n/k)) \times k = O(n\lg(n/k))
\stopformula
\stopANSWER

當 $k$ 是一個常數時,也可以給出 $k$ 排序算法的下界。
% e
\startigBase[a,continue]\startitem
證明:我們可以在 $O(n\lg{k})$ 時間內對一個長度爲 $n$ 的 $k$-有序數列進行全排序。
(\hint 利用\inexercise[k_merge] 的結果。)
\stopitem\stopigBase

\startANSWER
即總元素個數爲 $n$ 的 $k$ 個有序數列,進行歸並排序,
參見\inexercise[k_merge]。
\stopANSWER

% f
\startigBase[a,continue]\startitem
證明:當 $k$ 是一個常數時,對包含 $n$ 個元素的數列進行 $k$ 排序需要 $\Omega(n\lg{n})$ 的時間。
(\hint 根據比較配需的下界並利用上一項的結論。)
\stopitem\stopigBase

\startANSWER
還是解決 $k$ 個規模爲 $n/k$ 的子問題,
每個子問題最少比較次數爲 $\Omega((n/k)\lg(n/k))$,
總共 $\Omega(n\lg(n/k))$。
如果 $k$ 是常數,則結果爲 $\Omega(n\lg n)$。
\stopANSWER

\stopPROBLEM

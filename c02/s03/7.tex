\startEXERCISE[exercise:bin_search]
回顧\inexercise[linear_earch] 中的搜索問題,我們發現:
如果待搜索的子數列已經排好序,
則搜索算法可以檢查子數列的中間數值,而不是檢查 $v$,
在下一步搜索時可以直接跳過子數列中的一半數據。
二分查找算法就是重複這個過程,
每次都可以使得帶搜索的子數列減半。
試寫出二分查找算法的僞碼,遍歷、遞迴均可。
並證明在最壞情況下,此算法的運行時間為 $\Theta(\lg n)$。
\stopEXERCISE
\startANSWER
\CLRSH{BINARY-SEARCH(A, v)}
\startCLRSCODE
low = 1
high = A.length
while low \le high
	m = (low + high) / 2
	if A[m] == v
		return m
	if A[m] < v
		low = m + 1
	else
		high = m - 1
return NIL
\stopCLRSCODE

\startformula
T(n) = T(n/2) + c = \Theta(\lg n)
\stopformula
\stopANSWER

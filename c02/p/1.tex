\startPROBLEM
在歸並排序中對小數列使用插入排序。

雖然最壞情況下,歸并排序運行時間為 $\Theta(n\lg n)$,
而插入排序為 $\Theta(n^2)$,
但在許多機器上對於小規模問題,
由於插入排序的常係數很小,
反而比歸并排序還要快一些。
因此在歸并排序中,儅子問題規模足夠小時,
可以改用插入排序來處理。
嘗試改進歸并排序,
對於 $n/k$ 個子數列分別進行插入排序,
然後再用標準的歸并算法進行合并。
其中 $k$ 待定。
\startigBase[a]

\item 證明對 $n/k$ 個長度爲 $k$ 的子列進行插入排序,所需時間最長為 $\Theta(nk)$。

\startANSWER
設對長度爲 $k$ 的數列進行插入排序所用時間爲 \m{a k^2 + b k + c},則總時間爲:
\startformula
\frac{n}{k}(a k^2 + b k + c) = a n k + b n + \frac{c n}{k} = \Theta(nk)
\stopformula
\stopANSWER

\item 證明歸並這些子列所需時間最多為 $\Theta(n \lg(n/k))$。

\startANSWER
對 $a$ 個長度爲 $k$ 的子列進行歸並排序所需時間爲:
\startformula
T(a) = \startcases
\NC 0	\NC \text{若 $a = 1$,} \NR
\NC 2 T(a/2) + a k \NC \text{若 $a = 2^p$, $p > 0$。} \NR
\stopcases
\stopformula

設 $G(a) = \frac{T(a)}{a}$,則:
\startsplitformula\startalign
\NC G(a)	\NC = \frac{T(a)}{a} \NR
\NC 		\NC = \frac{2 T(a/2) + a k}{a} \NR
\NC		\NC = \frac{T(a/2)}{a/2} + k \NR
\NC		\NC = G(a/2) + k \NR
\stopalign\stopsplitformula
即
\startformula
G(a) = \startcases
\NC 0	\NC \text{若 $a = 1$,}\NR
\NC G(a/2) + k \NC \text{若 $a = 2^p$, $p > 0$。} \NR
\stopcases
\stopformula

設 $H(p) = G(2^p)$,則:
\startsplitformula\startalign
\NC H(p)	\NC = G(2^p) \NR
\NC		\NC = G(2^p/2) + k \NR
\NC		\NC = G(2^{p-1}) + k \NR
\NC		\NC = H(p-1) + k \NR
\stopalign\stopsplitformula
即
\startformula
H(p) = \startcases
\NC 0	\NC \text{若 $p = 0$,} \NR
\NC H(p-1) + k \NC \text{若 $p > 0$。} \NR
\stopcases
\stopformula

\startsplitformula\startalign
\NC H(p) \NC = k p \NR
\NC G(2^p) \NC = kp \NR
\NC G(a) \NC = k \lg(a) \NR
\NC \frac{T(a)}{a} \NC = k \lg(a) \NR
\NC T(a) \NC = k a \lg(a) \NR
\NC T(\frac{n}{k}) \NC = k \frac{n}{k} \lg(\frac{n}{k}) \NR
\NC		\NC = n \lg(\frac{n}{k}) \NR
\NC     \NC \Theta(n \lg(n/k)) \NR
\stopalign\stopsplitformula
\stopANSWER

\item 假定修改後的算法運行時間爲 $\Theta(nk+n \lg(n/k))$,
$k$ 最大爲多少時(作爲 $n$ 的函式),新算法與原歸並排序算法所用時間相同?

\startANSWER
\startsplitformula\startalign
\NC k \NC = \lg(n) \NR
\NC \Theta(nk + n \lg(n/k)) \NC = \Theta(n \lg(n) + n \lg(\frac{n}{\lg(n)}) \NR
\NC			\NC = \Theta(n \lg(n)) \NR
\stopalign\stopsplitformula
\stopANSWER

\stopigBase
\stopPROBLEM

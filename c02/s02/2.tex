\startEXERCISE[exercise:selection_sort]
對數列 $A[1:n]$ 中的 $n$ 個數進行排序。
先找到 $A[1:n]$ 中的最小元素,並將其與 $A[1]$ 進行交換;
然後找到 $A[2:n]$ 中的最小元素,並將其與 $A[2]$ 進行交換;
再找到 $A[3:n]$ 中的最小元素,並將其與 $A[3]$ 進行交換;
依次类推,對 $A$ 中前 $n-1$ 個元素都用這種方式進行處理。
這種排序方式稱爲\emph{選擇排序},試寫出其僞碼。
此算法滿足循環不變式嗎?
爲什麽僅需處理前 $n-1$ 個元素,而不是所有元素?
給出其最壞情況下的運行時間,用 $\Theta$ 表示。
最好情況下的運行時間有改善嗎?
\stopEXERCISE

\startANSWER
\startCLRSCODE
n = A.length
for j = 1 to n - 1
	s = j	// $s$ 为最小元素的索引
	for i = j + 1 to n
		if A[i] < A[ss]
			s = i
	\ALGO{SWAP(A[j], A[s])}
\stopCLRSCODE

\startsplitformula
\startmathalignment[n=2]
\NC \NC\sum_{i=1}^{n-1}n-i \NR
\NC = \NC n(n-1) - \sum_{i=1}^{n-1}i \NR
\NC = \NC n^2-n-\frac{n^2-n}{2} \NR
\NC = \NC \frac{n^2-n}{2} \NR
\NC = \NC \Theta(n^2) \NR
\stopmathalignment
\stopsplitformula

所用時間爲 $\Theta(n^2)$。
\stopANSWER

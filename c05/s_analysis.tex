\startsection[
  title={Probabilistic analysis and further uses of indicator random variables},
]

%e5.4-1
\startEXERCISE
屋子裏至少要有多少人,才能保證有人與你生日相同的概率不小於 $1/2$?
至少要有多少人,才能保證至少兩人生日爲 7 月 4 日的概率大於 $1/2$?
\stopEXERCISE

\startANSWER
令縂人數為 $n$,
任一人生日與己不同的概率爲 $364/365$,記爲 $p$,
則 $k$ 個人生日都與己不同的概率爲 $p^k$。
利用互補事件求 $n$:
\startsplitformula\startmathalignment
\NC 1 - p^n \NC \ge \frac{1}{2} \NR
\NC p^n     \NC \le \frac{1}{2} \NR
\NC n \lg p  \NC \ge \lg\frac{1}{2} \NR
\NC n = \lceil \log_p \frac{1}{2}\rceil \NC = 254 \NR
\stopmathalignment\stopsplitformula

另外一個問題,令生日為 7 月 4 日的人數為 $k$:
\startsplitformula\startmathalignment
\NC \Pr\{k\ge 2\} \NC=
        1 - \Pr\{k=1\} - Pr\{k=0\} \NR
\NC \NC= 1 - \binom{n}{k}(1-p)^k p^{n-k}|_{k=1} - p^{n-k}|_{k=0} \NR
\NC \NC= 1 - (n-np+p)p^{n-1} \NR
\stopmathalignment\stopsplitformula
計算可得 $n$ 至少爲 $613$。
\stopANSWER

%e5.4-2
\startEXERCISE
屋子裏至少需要多少人,才能保證有兩人生日相同的概率至少為 $0.99$?
在此人數下,記共有 $k$ 對人生日相同, $k$ 的期望值是多少?
\stopEXERCISE

\startANSWER
令人數為 $m$,$n$ 爲 $365$,則兩人生日相同的概率為:
\startsplitformula\startmathalignment
\NC 1 - \frac{n}{n}\cdot\frac{n-1}{n}\cdot\frac{n-2}{n}\cdots\frac{n-m+1}{n} \NC \ge 0.99 \NR
\NC \frac{n!}{(n-m)! n^m} \NC \le 0.01 \NR
\NC m \NC \ge 57 \NR 
\stopmathalignment\stopsplitformula

% @todo
\stopANSWER

%e5.4-3
\startEXERCISE
假設我們將球投到 $b$ 個箱子裏,直到某個箱子中有兩個球。
每一次投擲都相互獨立,並且每個球落入任一箱子的機會均等。
請問投擲次數的期望值是多少?
\stopEXERCISE

\startANSWER
本質還是生日問題,更多討論參見 \simpleurl{http://en.wikipedia.org/wiki/Birthday_problem#Average_number_of_people}。

令投擲次數爲 $X_b$,則 $X_b$ 大於 $k\ge 0$ 的概率爲:
\startformula
\NC \Pr\{X_b>k\} \NC = \frac{b!}{(b-k)!b^k} = 1 - \sum_{i=0}^{i=k}\Pr\{X_b=i\}\NR
\stopformula
$X_b$ 的期望值爲:
\startsplitformula\startmathalignment[n=3]
\NC \E[X_b] \NC = \NC \sum_{k=0}^{b+1}k\Pr\{X_b=k\} \NR
\NC \NC = \NC \Pr\{X_b=1\} + 2\Pr\{X_b=2\} + \cdots + (b+1)\Pr\{X_b=b+1\} \NR
\NC \NC = \NC \Pr\{X_b=1\} + \Pr\{X_b=2\} + \cdots + \Pr\{X_b=b+1\} \NR
\NC \NC   \NC + \Pr\{X_b=2\} + \Pr\{X_b=3\} + \cdots + \Pr\{X_b=b+1\} \NR
\NC \NC   \NC + \cdots \NR
\NC \NC   \NC + \Pr\{X_b=b+1\} \NR
\NC \NC = \NC \sum_{k=0}^{b}\Pr\{X_b>k\} \NR
\NC \NC = \NC 1 + \sum_{k=1}^{b}\frac{b!}{(b-k)!b^k} \NR
\stopmathalignment\stopsplitformula
\stopANSWER

%e5.4-4
\startEXERCISE \DIFFICULT
生日悖論的分析中,要求各人生日彼此獨立是否很重要?
或者,是否只要兩兩成對獨立就足夠了?
證明你的答案。
\stopEXERCISE
\startANSWER
成對獨立足夠了。對於(5.6)之後的推導,有此即可。
\stopANSWER

%e5.4-5
\startEXERCISE \DIFFICULT
一次聚會需要邀請多少人,才能讓其中 3 人的生日很可能相同?
\stopEXERCISE

\startANSWER
令生日有 $m=365$ 種可能,人數爲 $n$,只有 $i$ 對人生日相同的概率:
\startsplitformula\startmathalignment
\NC \NC
\frac{\left(\binom{n}{2}\frac{m}{m} \frac{1}{m}\right)
      \left(\binom{n-2}{2}\frac{m-1}{m} \frac{1}{m}\right)
      \left(\binom{n-4}{2}\frac{m-2}{m} \frac{1}{m}\right)
      \cdots
      \left(\binom{n-2(i-1)}{2}\frac{m-(i-1)}{m} \frac{1}{m}\right)}
     {i!} \NR
\NC \NC \qquad \frac{m-i}{m} \frac{m-(i+1)}{m} \cdots \frac{m-(n-i-1)}{m} \NR
\NC = \NC \frac{1}{i!}
          \frac{n!}{2^i (n-2i)!}
          \frac{m!}{(m-n+i)!}
          \frac{1}{m^n} \NR
\NC = \NC \frac{m!n!}{i!(n-2i)!(m-n+i)! 2^i m^n} \NR
\stopmathalignment\stopsplitformula
至少有三人生日相同的概率爲:
\startsplitformula\startmathalignment
\NC \NC 1 - \sum_{i=0}^{\lfloor n/2\rfloor}\frac{m!n!}{i!(n-2i)!(m-n+i)! 2^i m^n} \NR
\NC = \NC 1 - \frac{m!n!}{m^n}\sum_{i=0}^{\lfloor n/2\rfloor}
              \frac{1}{i!(n-2i)!(m-n+i)! 2^i} \NR
\stopmathalignment\stopsplitformula
請參考:\simpleurl{http://math.stackexchange.com/questions/25876/probability-of-3-people-in-a-room-of-30-having-the-same-birthday}
\stopANSWER

%e5.4-6
\startEXERCISE \DIFFICULT
一個長度爲 $k$ 的字串,其中所有字符均選自一個元素個數爲 $n$ 的集合,
那麼此字串構成一個 $k$ 排列的概率是多少?
此問題與生日悖論有何關聯?
\stopEXERCISE

\startANSWER
\startsplitformula\startmathalignment
\NC \NC \Pr\{k\text{-perm in }n\} \NR
\NC =\NC  1 \cdot \frac{n-1}{n} \cdot \frac{n-2}{n} \cdots \frac{n-k+1}{n} \NR
\NC =\NC \frac{(n-1)!}{(n-k)!n^k} \NR
\stopmathalignment\stopsplitformula
這是生日問題的互補事件,即 \m{k} 個人生日各不相同。
\stopANSWER

%e5.4-7
\startEXERCISE \DIFFICULT
假設將 $n$ 個球投入 $n$ 個箱子裏,其中每次投球相互獨立,
並且每個球落入任一箱子的機會均等。
空箱子的數目期望值是多少?
正好有一個球的箱子數目期望值是多少?
\stopEXERCISE

\startANSWER
當 $n$ 足夠大時,兩個答案都漸進於 $n/e$。
首先來看空箱子的數目:

令 $X_i$ 代表的事件爲:第 $i$ 個箱子爲空:
\startsplitformula\startmathalignment
\NC \Pr\{X_i\} \NC = \left(\frac{n-1}{n}\right)^n\NR
\NC \NC = \left(1 - \frac{1}{n}\right)^n \NR
\NC \NC \approx \frac{1}{e} \NR
\stopmathalignment\stopsplitformula
其期望值爲:
\startformula
E[X] = \sum_{i=1}^n E[X_i] \approx \frac{n}{e}
\stopformula

箱子裏只有一個球的情況類似,其概率爲:
\startsplitformula\startmathalignment
\NC \Pr\{Y_i\} \NC = n\frac{1}{n}\left(\frac{n-1}{n}\right)^{n-1} \NR
\NC \NC = \left(\frac{n-1}{n}\right)^{n-1} \approx \frac{1}{e} \NR
\stopmathalignment\stopsplitformula
期望值一樣。

參見 \simpleurl{http://math.stackexchange.com/questions/545920/expectation-of-throwing-n-balls-into-n-bins}。
\stopANSWER

%e5.4-8
\startEXERCISE \DIFFICULT
爲使特徵序列長度的下界更精確,
請說明公平拋擲 \m{m} 次硬幣,
如果所有連續正面特徵序列的長度均不大於 \m{\lg{n} - 2\lg\lg{n}},
其概率小於 \m{1/n}。
\stopEXERCISE

\startANSWER
將 $n$ 次結果分爲 $s$ 組,其中 $s=\lg(n)-2\lg(\lg(n))$。
一組中均爲正面的概率爲:
\startformula
\Pr(A_{i,\lg n - 2\lg(\lg n)}) = \left(\frac{1}{2}\right)^{\lg n - 2\lg(\lg n)} = \frac{(\lg n)^2}{n}
\stopformula
各組之間相互獨立,所有組均不是全部正面的概率爲:
\startsplitformula\startmathalignment
\NC  \NC \Pr(\bigwedge\neg A_{i,\lg n - 2\lg(\lg n)}) \NR
\NC =\NC \prod_i\Pr(\neg A_{i,\lg n - 2\lg(\lg n)}) \NR
\NC =\NC \left(1-\frac{(\lg n)^2}{n}\right)^{\frac{n}{\lg n - 2\lg(\lg n)}} \NR
\NC \le\NC e^{-\frac{(\lg n)^2}{\lg n - 2\lg(\lg n)}} \NR
\NC =\NC e^{-\lg n \left(1 + \frac{2\lg(\lg n)}{\lg n - 2\lg(\lg n)}\right)} \NR
\NC <\NC e^{-\lg n} \NR
\NC <\NC n^{-1} \NR
\stopmathalignment\stopsplitformula
\stopANSWER

\stopsection

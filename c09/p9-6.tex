\startPROBLEM
(Select with groups of 3)
\inexercise[9_3_1] 中我們已經證明了按每組 $7$ 個元素進行分組,
 \ALGO{SELECT} 的運行時間仍是線性的。
這個問題我們再來看一下按每組 $3$ 個元素分組如何。
\startigBase[a]\startitem
證明:只要分組時每組元素個數是大於 $3$ 的奇數,
 \ALGO{SELECT} 的運行時間就是線性的。
\stopitem\stopigBase

\startANSWER
請參考\inexercise[9_3_1] 的證明。
\stopANSWER

% b
\startigBase[continue]\startitem
證明:如果每組元素個數是 $3$,
 \ALGO{SELECT} 的運行時間是 $O(n\lg{n})$。
\stopitem\stopigBase

\startANSWER
請參考\inexercise[9_3_1] 的證明。
\stopANSWER

b) 項給出的是一個上界,
我們還不知道按 $3$ 個一組劃分是否能在 $O(n)$ 時間內完成。
但在處理中值所構成的集合時,
我們可以改變選擇主元的方式來保證時間爲 $O(n)$。
 \ALGO{SELECT3} 用於選擇第 $i$ 小的元素,
假設數列元素個數 $n>1$,其所有元素互不相同。

\CLRSH{SELECT3(A,p,r,i)}
\startCLRSCODE
while (r-p+1) \bmod 9 \ne 0
	for j = p+1 to r	// 將最小元素放入 $A[p]$
		if A[p] > A[j]
			\ALGO{SWAP(A[p],A[j])}
	// 如果我們想要的是 $A[p:r]$ 中的最小值,那麼已經找到了
	if i == 1
		return A[p]
	// 否則,我們要的是 $A[p+1:r]$ 中的第 $(i-1)$ 小的元素
	p = p+1
	i = i-1
g=(r-p+1)/3	// 一共有 $g$ 組數據,每組 $3$ 個元素
for j=p to p+g-1	// 對每組數據進行排序
	\ALGO{SORT-IN-PLACE(A[j],A[j+g],A[j+2g])}
// 所有組的中值均位於 $A[p:r]$ 中間的 1/3 內
g' = g/3	// 重新分組,每組 3 個元素
for j=p+g to p+g+g'-1	// 爲子組排序
	\ALGO{SORT-IN-PLACE(A[j],A[j+g'],A[j+2g'])}
// 所有子組的中值均位於 $A[p:r]$ 中間的 1/9 內
// 遞迴查找所有子組中值的中值
x = \ALGO{SELECT3(A,p+4g',p+5g'-1,\lceil g'/2\rceil)}
q = \ALGO{PARTITION-AROUND(A,p,r,x)}	// 以 $x$ 爲主元進行劃分
// 接下來就跟 \ALGO{SELECT} 的第 19-24 行一樣了
k = q-p+1
if i==k
	return A[q]	// 主元即爲所求
elseif i < k
	return \ALGO{SELECT3(A,p,q-1,i)}
else
	return \ALGO{SELECT3(A,q+1,r,i-k)}
\stopCLRSCODE

% c
\startigBase[continue]\startitem
請說明 \ALGO{SELECT3} 是如何工作的。
可以用圖表示。
\stopitem\stopigBase

\startANSWER
略。
\stopANSWER

% d
\startigBase[continue]\startitem
證明 \ALGO{SELECT3} 在最壞情況下運行時間爲 $O(n)$。
\stopitem\stopigBase

\startANSWER
略。
\stopANSWER

\stopPROBLEM

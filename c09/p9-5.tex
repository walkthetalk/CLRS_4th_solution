\startPROBLEM
(Alternative analysis of randomized selection)
在這個問題中,我們用指示器隨機變量來分析 \ALGO{RANDOMIZED-SELECT},
與節7.4.2 中 \ALGO{RANDOMIZED-QUICKSORT} 的分析方法類似。

在快速排序的分析中,我們假設所有元素互不相同,
並且將輸入數列 $A$ 中的元素重命名爲 $z_1,z_2,\ldots,z_n$,
其中 $z_i$ 是第 $i$ 小元素。
因此,調用 \ALGO{RANDOMIZED-SELECT(A, 1, n, i)} 返回 $z_i$。

對於 $1\le j < k \le n$,令
\startformula
X_{ijk}=I\{\text{在查找 $z_i$ 的過程中會比較 $z_j$ 和 $z_k$}\}
\stopformula

\startigBase[a]\startitem
給出 $E[X_{ijk}]$ 的算式。
(\hint 算式可能會因 $i$、 $j$ 和 $k$ 的值而異。)
\stopitem\stopigBase

\startANSWER
與分析快速排序類似,但需要考慮 $i$。
所選主元位於 $\min(z_i,z_j,z_k)$ 和 $\max(z_i,z_j,z_k)$ 之間,
若選中 $z_j$ 或 $z_k$ 就會比較他們,
若選中 $z_j$ 和 $z_k$ 中間的數,就不會比較他們,
若選中 $z_i$ 和 $[z_j,z_k]$ 中間的數,也不會比較他們。
準確的算式依賴於 $i$ 與 $j$、 $k$ 的相對位置:
\startformula
E[X_{ijk}] = \startmathcases
\NC 2 / (k - i + 1) \NC \text{若 $z_i \le z_j < z_k$,} \NR
\NC 2 / (k - j + 1) \NC \text{若 $z_j \le z_i < z_k$,} \NR
\NC 2 / (i - j + 1) \NC \text{若 $z_j < z_k \le z_i$。} \NR
\stopmathcases
\stopformula
\stopANSWER

% b
\startigBase[continue]\startitem
在數列 $A$ 中查找 $z_i$ 時,令 $X_i$ 指示總的比較次數。證明:
\startformula
E[X_i] \le 2 \left(
       \sum_{j=1}^i \sum_{k=i}^n \frac{1}{k - j + 1} +
       \sum_{k=i+1}^n \frac{k - i - 1}{k - i + 1} +
       \sum_{j=1}^{i-2} \frac{i - j - 1}{i - j + 1}
       \right)
\stopformula
\stopitem\stopigBase

\startANSWER
\startsplitformula\startmathalignment
\NC E[X_i]
    \NC = \sum_{j=1}^{n-1}   \sum_{k=j+1}^n E[X_{ijk}] \NR
\NC \NC = \sum_{j=1}^i       \sum_{k=j+1}^n E[X_{ijk}]
            + \sum_{j=i+1}^{n-1} \sum_{k=j+1}^n E[X_{ijk}] \NR
\NC \NC = \sum_{j=1}^i \left(\sum_{k=j+1}^{i-1}E[X_{ijk}]
                                 + \sum_{k=i}^n E[X_{ijk}] \right)
            + \sum_{j=i+1}^{n-1} \sum_{k=j+1}^n E[X_{ijk}] \NR
\NC \NC = \sum_{j=1}^i \sum_{k=j+1}^{i-1}E[X_{ijk}]
            + \sum_{j=1}^i \sum_{k=i}^n E[X_{ijk}]
            + \sum_{j=i+1}^{n-1} \sum_{k=j+1}^n E[X_{ijk}] \NR
\NC \NC = \sum_{j=1}^{i-2} \sum_{k=j+1}^{i-1}E[X_{ijk}]
            + \sum_{j=1}^i \sum_{k=i}^n E[X_{ijk}]
            + \sum_{j=i+1}^{n-1} \sum_{k=j+1}^n E[X_{ijk}] \NR
\NC \NC = \sum_{j=1}^{i-2} \sum_{k=j+1}^{i-1} \frac{2}{i-j+1}
            + \sum_{j=1}^i \sum_{k=i}^n \frac{2}{k-j+1}
            + \sum_{j=i+1}^{n-1} \sum_{k=j+1}^n \frac{2}{k-i+1} \NR
\NC \NC = 2\left(
                \sum_{j=1}^{i-2} \sum_{k=j+1}^{i-1} \frac{1}{i-j+1}
            + \sum_{j=1}^i \sum_{k=i}^n \frac{1}{k-j+1}
            + \sum_{j=i+1}^{n-1} \sum_{k=j+1}^n \frac{1}{k-i+1}
          \right) \NR
\NC \NC = 2\left(
                \sum_{j=1}^{i-2} \frac{i-j-1}{i-j+1}
            + \sum_{j=1}^i \sum_{k=i}^n \frac{1}{k-j+1}
            + \sum_{j=i+1}^{n-1} \sum_{k=j+1}^n \frac{1}{k-i+1}
          \right) \NR
\stopmathalignment\stopsplitformula
接下來只須證明第三項:
\startformula
\sum_{j=i+1}^{n-1} \sum_{k=j+1}^n \frac{1}{k-i+1} \le \sum_{k=i+1}^{n}\frac{k-i-1}{k-i+1}
\stopformula
證明如下:
\startsplitformula\startmathalignment
\NC \NC \sum_{j=i+1}^{n-1} \sum_{k=j+1}^n \frac{1}{k-i+1} \NR
\NC = \NC \sum_{k=i+2}^{n} \sum_{j=i+1}^{k-1} \frac{1}{k-i+1} \NR
\NC = \NC \sum_{k=i+2}^{n} \frac{k-i-1}{k-i+1} \NR
\NC \le \NC \sum_{k=i+1}^{n}\frac{k-i-1}{k-i+1} \NR
\stopmathalignment\stopsplitformula
\stopANSWER

% c
\startigBase[continue]\startitem
證明 $E[X_i]\le 4n$。
\stopitem\stopigBase

\startANSWER
後兩項的和:
\startformula
\sum_{k=i+1}^n\frac{k-i-1}{k-i+1} + \sum_{j=1}^{i-2}\frac{i-j-1}{i-j+1}
   \le \sum_{k=i+1}^n 1 + \sum_{j=1}^{i-2} 1
   = n - i + i - 2
   \le n
\stopformula
對於第一項,則可由下列矩陣來說明:
\startsplitformula\startmatrix
\NC j\backslash k  \NC i \NC i+1 \NC \ldots \NC n-1 \NC n \NR
\NC i  \NC 1 \NC \frac{1}{2} \NC \ldots \NC \frac{1}{n-i} \NC \frac{1}{n-i+1} \NR
\NC i-1\NC \frac{1}{2} \NC \frac{1}{3} \NC \ldots \NC \frac{1}{n-i+1} \NC \frac{1}{n-i+2} \NR
\NC \cdots \NC \cdots \NC \cdots\NC \ddots \NC \cdots \NC \cdots \NR
\NC 2  \NC \frac{1}{i-1} \NC \frac{1}{i}   \NC \ldots \NC \frac{1}{n-2} \NC \frac{1}{n-1} \NR
\NC 1  \NC \frac{1}{i}   \NC \frac{1}{i+1} \NC \ldots \NC \frac{1}{n-1} \NC \frac{1}{n}   \NR
\stopmatrix\stopsplitformula
上面矩陣中,行坐標爲 $j=i,i-1,\ldots,1$,
列坐標爲 $k=i,i+1,\ldots,n$,
矩陣元素即爲 $\frac{1}{k - j + 1}$。
左下到右上斜對角線上的元素相同均爲 $\frac{1}{m}$ 的形式,
其中 $m=1,2,\ldots,n$。
每個 $\frac{1}{m}$ 的個數都小於等於 $m$。因此整個矩陣元素的和小於 $n$。

因此:
\startformula
E[X_i] \le 2 \left(
       \sum_{j=1}^i \sum_{k=i}^n \frac{1}{k - j + 1} +
       \sum_{k=i+1}^n \frac{k - i - 1}{k - i + 1} +
       \sum_{j=1}^{i-2} \frac{i - j - 1}{i - j + 1}
       \right) \le 2(n+n) = 4n
\stopformula
\stopANSWER

% d
\startigBase[continue]\startitem
證明:假設數列 $A$ 中的所有元素互異,
則 \ALGO{RANDOMIZED-SELECT} 的期望運行時間爲 $O(n)$。
\stopitem\stopigBase

\startANSWER
\ALGO{RANDOMIZED-SELECT} 中的運算跟比較次數成線性關系,
而比較次數的期望值又和 $n$ 成線性關系,
因此期望運行時間爲線性的 $O(n)$。
\stopANSWER

\stopPROBLEM

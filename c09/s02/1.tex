\startEXERCISE
證明:在 $RANDOMIZED-SELECT$ 中,不會出現對 0-長度數列的遞迴調用。

\CLRSH{RANDOMIZED-SELECT(A,p,r,i)}
\startCLRSCODE
if p == r
	return A[p]	// $1\le i \le r-p+1$,當 $p==r$ 時, $i=1$。
q = \ALGO{RANDOMIZED-PARTITION(A,p,r)}
k = q - p + 1
if i == k
	return A[q]	// 主元就是答案
elseif i < k
	return \ALGO{RANDOMIZEWD-SELECT(A,p,q-1,i)}
else
	return \ALGO{RANDOMIZED-SELECT(A,q+1,r,i-k)}
\stopCLRSCODE
\stopEXERCISE

\startANSWER
如果要對長度爲 0 的數列進行遞迴調用:
\startigNum
\startitem
第 8 行中,需要 $p=q$,即 $k=1$,但是 $i<k$ 就不可能成立;
\stopitem
\startitem
第 9 行中,需要 $q=r$,則 $k=r-p+1$,但同時需要 $i>k$,不可能成立。
\stopitem
\stopigNum
然後根據歸納法維持不變式得證。
\stopANSWER

\startEXERCISE[exercise:oil_well]
Olay 教授是一家石油公司的顧問。
這家公司正在計劃建造一條從東到西的大型輸油管道,
這一管道所穿越的油田有 $n$ 口油井。
公司希望每口油井都有一條管道支線沿着最短路徑連接到主管道(方向或南或北),如下圖所示。
給定每口油井的 $x$ 和 $y$ 坐標,
教授應該如何選擇主管道的最優位置,使得各支線的總長度最小?
證明:該最優位置可以在線性時間內確定。

\externalfigure[output/e9_3_9-1]
\stopEXERCISE

\startANSWER
只需關心 $y$ 坐標, $x$ 坐標沒有影響。

如果 $n$ 是奇數,則選取所有油井 $y$ 坐標的中位數,作爲主管道的 $y$ 坐標,
即主管道穿過此油井。這樣主管道兩側的油井數目相同。
對於任兩口油井而言,只要主管道在他們中間通過,那麼這兩口油井的支線管道總長度是不變的。

如果 $n$ 是偶數,則需要所有油井 $y$ 坐標的兩個中位數,
主管道的 $y$ 坐標在這兩個 $y$ 坐標中間即可。
\stopANSWER

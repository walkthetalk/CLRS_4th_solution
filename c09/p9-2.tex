\startPROBLEM
(隨機選擇的變種)
 Mendel 教授建議不要檢查 $i$ 和 $k$ 是否相等,
從而簡化 \ALGO{RANDOMIZED-SELECT},
簡化後結果如下:

\CLRSH{SIMPLER-RANDOMIZED-SELECT(A,p,r,i)}
\startCLRSCODE
if p == r
	return A[p]	// $1\le i \le r-p+1$ 意味着 $i=1$
q = \ALGO{RANDOMIZED-PARTITION(A,p,r)}
k = q-p+1
if i \le k
	return \ALGO{SIMPLER-RANDOMIZED-SELECT(A,p,q,i)}
else
	return \ALGO{SIMPLER-RANDOMIZED-SELECT(A,q+1,r,i-k)}
\stopCLRSCODE

\startigBase[a]\startitem
證明:在最壞情況下,\ALGO{SIMPLER-RANDOMIZED-SELECT(A,p,r,i)} 無法終止。
\stopitem\stopigBase

\startANSWER
如果 $q=r$,則會執行第 6 行的分支,算法就不會終止。
\stopANSWER

\startigBase[continue]\startitem
證明: \ALGO{SIMPLER-RANDOMIZED-SELECT(A,p,r,i)} 的期望運行時間仍是 $O(n)$。
\stopitem\stopigBase

\startANSWER
\startformula
E_n = 1 + \frac{2}{n}\sum_{i=0}^{n-1}E_i
\stopformula
解得:
\startformula
E_n = \frac{2n-1}{3} + \frac{n+1}{3}(E_0 + E_1)
\stopformula
\stopANSWER
\stopPROBLEM


\startEXERCISE\DIFFICULT
證明:要在 $n$ 個元素中找到最大值和最小值,
最壞情況下比較次數的下界是 $\lceil 3n/2 \rceil - 2$。
(\hint 想一下有多少個數有可能是最大值或最小值,
然後分析一下每一次比較會如何影響這些計數。)
\stopEXERCISE

\startANSWER
以錦標賽方式,最大元素需要 $n-1$ 次比較,
而最小元素只可能在第一輪比賽中失利的元素中產生,
這樣的元素有 $\lceil n/2 \rceil$ 個
(考慮 $n$ 爲奇數的情況)。
要產生最小元素還須 $\lceil n/2 \rceil - 1$ 次比較。
一共需要 $\lceil 3n/2 \rceil - 2$ 次比較。

也可以這樣考慮:
第一步,先取兩個數進行比較,大的候選最大數,小的候選最小數,需比較 $1$ 次;
第二步,然後再取兩個數進行比較,大的跟最大候選進行比較,小的跟最小候選進行比較,共 $3$ 次。
重複第二步,直到所有數全部取完。

如果 $n$ 是偶數,則共比較 $ (n-2)/2 * 3 + 1 = 3n/2 - 2$ 次。
如果 $n$ 是奇數,最後一步最多須比較 $2$ 次,
總比較次數爲 $(n-3)/2*3 + 1 + 2 = \frac{3n+1}{2}-2$。
綜上,最終比較次數爲 $\lceil 3n/2 \rceil - 2$。

\stopANSWER

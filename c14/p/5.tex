\startPROBLEM
(Edit distance)
爲將一串文本 $x[1..m]$ 轉換爲目標串 $y[1..n]$,
我們可以使用多種變換操作。
我們的目標是:給定 $x$ 和 $y$,
求將 $x$ 轉換爲 $y$ 的一個變換操作序列。
使用 $z$ 保存中間結果,假定他足夠大,
可寸下中間結果的所有字符。
初始時, $z$ 是空的,
結束時,應有 $z[j]=y[j]$,其中 $j=1,2,\ldots,n$。
我們維護兩個下標 $i$ 和 $j$,分別用來索引 $x$ 和 $z$,
變換操作允許改變 $z$ 的內容和這兩個下標。
初始時, $i=j=1$。
在轉換過程中應處理 $x$ 的所有字符,
這意味着在變換操作結束時,應有 $i=m+1$。

我們可以使用如下 6 種變換操作,每種變化的開銷都是常數:
\startigBase
\item \emph{複製}(copy)——從 $x$ 複製一個字符到 $z$,
即賦值操作 $z[j] = x[i]$,並將兩個下標 $i$ 和 $j$ 都增 1。
此操作會處理 $x[i]$,開銷爲 $Q_C$。

\item \emph{替換}(replace)——將 $x$ 中的一個字符
替換爲另一個字符 $c$, $z[j]=c$,
並將兩個下標 $i$ 和 $j$ 都增 1。
此操作會處理 $x[i]$,開銷爲 $Q_R$。

\item \emph{刪除}(delete)——刪除 $x$ 中一個字符,
即將 $i$ 增 1, $j$ 不變。
此操作會處理 $x[i]$,開銷爲 $Q_D$。

\item \emph{插入}(insert)——將字符 $c$ 插入 $z$ 中,
 $z[j]=c$,將 $j$ 增 1, $i$ 不變。
此操作不會處理 $x$,開銷爲 $Q_I$。

\item \emph{旋轉}(twiddle,即交換)——將 $x$ 中下兩個字符複製到 $z$ 中,
但交換順序, $z[j]=x[i+1]$、 $z[j+1]=x[i]$,
將 $i$ 和 $j$ 都增 2。
此操作處理了 $x[i]$ 和 $x[i+1]$,開銷爲 $Q_T$。

\item \emph{終止}(kill)——刪除 $x$ 中剩餘字符,
令 $i=m+1$。
此變換會處理 $x$ 中所有尚未處理的字符。
如果執行此操作,則轉換過程結束。
開銷爲 $Q_K$。
\stopigBase

圖 14.12 給出了一種變換,
可以將源字串 {\tt algorithm} 轉換爲目標字串 {\tt altruistic},
下劃線標出了執行每個變換操作後兩個下標的位置。
還有很多其他變換序列也可以完成這個變換。

假設 $Q_C<Q_D+Q_I$ 且 $Q_R<Q_D+Q_I$,
否則,拷貝和替換就沒有意義了。
整個變換序列的總開銷就是各步驟開銷的和。
圖 14.12 中變換序列的總開銷是 $3Q_C + Q_R + Q_D + 4Q_I + Q_T + Q_K$。

\inputsamedir{fig14.12}

% a
\startigBase[a]\startitem
給定兩個字串 $x[1..m]$ 和 $y[1..n]$ 以及變換操作的開銷,
 $x$ 到 $y$ 的\emph{編輯距離}(edit distance)是
將 $x$ 轉換爲 $y$ 所需變換序列的最小開銷。
設計動態規劃算法,
求 $x[1..m]$ 到 $y[1..n]$ 的編輯距離並打印最優變換序列。
分析算法的時間、空間複雜度。
\stopitem\stopigBase

\startANSWER
子問題:將 $x[1..i]$ 轉換成 $y[1..j]$,
其中 $1\le i\le m$, $1\le j\le n$,
其代價爲 $Q(i,j)$。則:
\startformula\startmathalignment
\NC Q(0,0) \NC = 0 \NR
\NC Q(i,j) \NC = \min\startcases
\NC Q(i-1,j-1) + Q_C \NC \text{如果 $x_i=y_j$;} \NR
\NC Q(i-1,j-1) + Q_R \NC \text{如果 $x_i\ne y_j$;} \NR
\NC Q(i-1,j) + Q_D \NC \NR
\NC Q(i,j-1) + Q_I \NC \NR
\NC Q(i-2,j-2) + Q_S \NC \text{如果$[x_{i-1}x_i]=[y_i y_{i-1}]$;} \NR
\stopcases \NR
\stopmathalignment\stopformula
除 $D$ 外,我們還可以維護一張表用來記錄執行了哪些操作。
最小代價爲
\startformula
Q[m,n] = \min_{i<m}\{Q[i,n]\} + Q_K
\stopformula

時間、空間複雜度均爲 $\Theta(mn)$。
\stopANSWER

編輯距離問題是 DNA 序列對齊問題的推廣
(參考其他文獻,如 Setubal 和 Meidanis [405, 3.2 節])。
已有多種方法可以通過對齊兩個 DNA 序列來衡量他們的相似度。
有一種對齊方法是將空格符插入到兩個序列 $x$ 和 $y$ 中,
可以插入到任何位置(包括兩端),
使得結果序列 $x'$ 和 $y'$ 的長度相同,
但不會在相同的位置出現空格符
(即不存在位置 $j$ 使得 $x'[j]$ 和 $y'[j]$ 都是空格符)。
然後爲每個位置“打分”,位置 $j$ 的分數爲:

+1,如果 $x'[j]=y'[j]$ 且不是空格符;

-1,如果 $x'[j]\ne y'[j]$ 且都不是空格符;

-2, $x'[j]$ 或 $y'[j]$ 是空格符;

對齊方案的分數爲所有位置的分數之和。
例如,給定序列 $x=\text{\tt GATCGGCAT}$ 和 $y=\text{\tt CAATGTGAATC}$,
一種對齊方案爲:

{\tt G ATCG GCAT }

{\tt CAAT GTGAATC}

{\tt -*++*+*+-++*}

其中 + 表示該位置分數爲 +1, - 表示分數爲 -1, * 表示分數爲 -2,
因此此方案的總分數爲 $6\times 1 - 2 \times 1 - 4 \times 2 = -4$。

\startigBase[continue]\startitem
解釋如何將最優對齊問題轉換爲編輯距離問題,
使用的操作爲變換操作復制、替換、刪除、插入、旋轉和終止的子集。
\stopitem\stopigBase

\startANSWER
如果 $x'[j]=y'[j]$ 且不是空格符,則等效爲複製;
如果 $x'[j]$ 爲空格, $y'[j]$ 不是空格,則等效爲插入;
如果 $x'[j]$ 不是空格, $y'[j]$ 是空格,則等效爲刪除;
如果 $x'[j]\ne y'[j]$ 且都不是空格符,則等效爲替換。
\stopANSWER

\stopPROBLEM

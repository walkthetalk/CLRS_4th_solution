\startPROBLEM
(Image compression by seam carving)
給定一副彩色圖像,他由數列 $A[1..m,1..n]$ 構成,
每個元素都是包含紅綠藍(RGB)亮度的三元組。
假如我們希望輕度壓縮這幅圖像。具體而言,
我們希望從 $m$ 行中各刪除一個像素,使得圖像變窄一個像素。
但爲了避免影響視覺效果,我們要求相鄰兩行中刪除的像素必須位於同一列或相鄰列。
也就是說,刪除的像素構成從頂端到底端的一條“接縫”(seam),
相鄰像素均在垂直或對角線方向上相鄰。
\startigBase[a]\startitem
證明:可能的接縫數量是 $m$ 的指數函數,假定 $n>1$。
\stopitem\stopigBase

\startANSWER
窮舉法,第一行有 $n$ 種選擇,
以後每一行都有 3 種選擇(不考慮邊界的特殊性)。
共 $n\times 3^{m-1}$ 種選擇。
即使考慮邊界,下一行也至少有兩種選擇,
即下界爲 $n\times 2^{m-1}$。
命題得證。
\stopANSWER

\startigBase[continue]\startitem
假定現在對每個像素 $A[i,j]$,
我們都已計算出其“破壞度” $d[i,j]$ (實數),
表示刪除像素 $A[i,j]$ 對圖像可視效果的破壞程度。
直觀地,一個像素的破壞度越低,他與相鄰像素的相似度越高。
再假定一條接縫的破壞度定義爲他包含的像素的破壞度之和。
設計算法,尋找破壞度最低的接縫。
分析算法的時間複雜度。
\stopitem\stopigBase

\startANSWER
定義 $a[i,j]$:如果接縫只包含第 1 行到第 $i$ 行,
選擇了 $A[i,j]$ 作爲接縫時的最小破壞度。
則:
\startformula
a[i,j] = d[i,j] + \min(a[i-1,j-1], a[i-1,j], a[i-1,j+1])
\stopformula
注意,對於第一行元素 $a[1,j] = d[1,j]$,
其他行邊緣像素只有兩種選擇。
算法複雜度爲 $O(mn)$。
\stopANSWER

\stopPROBLEM

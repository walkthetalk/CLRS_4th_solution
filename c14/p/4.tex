\startPROBLEM
(Printing neatly)
所謂整齊打印問題,
就是用等寬字符打印一段文本。
輸入文本是 $n$ 個單詞,
單詞長度分別爲 $l_1,l_2,\ldots,l_n$ 個字符。
我們希望將此段文本整齊打印爲多行,
每行最多 $M$ 個字符。
“整齊”的標準是這樣的:
如果某行包含第 $i$ 到第 $j$ 個單詞($i\le j$),
且單詞間隔爲一個空格符,
則行尾的額外空格數量爲 $M-j+i-\sum_{k=i}^{j}l_k$,
這個值不能是負數,否則一行內無法容納這些單詞。
我們希望所有行(最後一行除外)的額外空格數的立方之和最小。
設計一個動態規劃算法,
整齊打印一段有 $n$ 個單詞的文本。
分析算法的時間、空間復雜度。
\stopPROBLEM

\startANSWER
如果某一行包含單詞 $i$ 到 $j$,
令 $s[i,j] = M - j + i - \sum_{k=1}^{j}l_k$ 爲此行行尾空格數目。
需要注意的是, $s$ 可能是負值。

定義此行的代價爲 $c[i,j]$,則:
\startformula
c[i,j] = \startcases
\NC \infty \NC \text{如果 $s[i,j] < 0$ (即單詞 $i,\ldots,j$ 不滿足要求);} \NR
\NC 0 \NC \text{如果 $j=n$ 且 $s\ge 0$ (最後一行代價爲 0);} \NR
\NC (s[i,j])^3 \NC \text{否則。} \NR
\stopcases
\stopformula

目標就是使得所有 $c$ 的和最小。

令 $C(j)$ 爲前 $j$ 個單詞的最小代價,其中第 $j$ 個單詞位於一行的末尾,則:
\startformula\startmathalignment
\NC C(0) \NC = 0 \NR
\NC C(j) \NC = \min_{i = 1}^{j}\{C(i-1) + c(i, j)\} \NR
\stopmathalignment\stopformula
$C(n)$ 即爲最終結果。

\CLRSH{PRINT_NEATLY(M, l, n)}
\startCLRSCODE
C[0] = 0
for j = 1 to n
	C[j] = ∞
	for i = 1 to j
		if s(i,j) < 0
			break;
		tmp = C[i-1] + s(i,j)
		if tmp < C[j]
			C[j] = tmp
			L[j] = i
return L
\stopCLRSCODE

$L[n]$ 的值爲最後一行第一個單詞的序號,
即,令 $k=L[n]$,則最後一行的單詞爲 $l_k,\ldots, l_n$。
而 $L[k]$ 的值爲倒數第二行第一個單詞的序號。
以此類推,我們可以找到每一行的第一個單詞,即可打印出所有單詞。

$C$ 和 $L$ 的長度均爲 $n+1$,
因此空間複雜度爲 $O(n)$。

另外 $s(i,j)$ 也可以用數組記錄下來,所用空間爲 $O(n)$。

運行時間爲填充 $C$ 和 $L$ 的時間,
加上遍歷 $L$ 打印所有單詞的時間。
其中遍歷 $L$ 的時間爲 $O(n)$。
填充 $C[j]$ 時,我們做了 $j$ 次比較,
因此循環所需時間爲 $O(n^2)$。
但是考慮到每一行的單詞數目肯定小於 $M/2$ (一旦打印長度超過了 $M$,
即可提前退出循環),因此比較所用時間爲 $O(M)$。
從而整體時間爲 $O(nM)$。
\stopANSWER

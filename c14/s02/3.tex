\startEXERCISE
用代入法證明遞迴公式(14.6)的結果爲 $\Omega(2^n)$。附公式(14.6):
\startformula
P(n) = \startmathcases
\NC 1 \NC \text{如果 $n = 1$,} \NR
\NC \sum_{k=1}^{n-1} P(k) P(n-k) \NC \text{如果 $n\ge 2$。} \NR
\stopmathcases
\stopformula
\stopEXERCISE

\startANSWER
猜測 $P(k) \ge c 2^k$,則:
\startsplitformula\startmathalignment
\NC P(n+1)
    \NC \ge \sum_{k=1}^{n} c 2^k \cdot c 2^{n-k} \NR
\NC \NC = \sum_{k=1}^{n-1} c^2 2^n \NR
\NC \NC = c^2 (n-1) 2^n \NR
\NC \NC \ge c^2 2^n \mathcomment{$(n\ge 2)$}\NR
\NC \NC \ge c 2^n \mathcomment{$(c\ge 1)$}\NR
\NC \NC = \Omega(2^n) \NR
\stopmathalignment\stopsplitformula
\stopANSWER

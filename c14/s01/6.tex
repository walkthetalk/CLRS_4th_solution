\startEXERCISE
Fibonacci 數列可以用遞迴式(3.31)定義。
設計一個運行時間爲 $O(n)$ 的動態規劃算法計算第 $n$ 個 Fibonacci 數。
畫出子問題圖。圖中有多少頂點和邊?
附遞迴式(3.31):

\startformula
F_i = \startmathcases
\NC 0 \NC \text{如果 $i=0$,} \NR
\NC 1 \NC \text{如果 $i=1$,} \NR
\NC F_{i-1} + F_{i-2} \NC \text{如果 $i\ge 2$。} \NR
\stopmathcases
\stopformula
\stopEXERCISE

\startANSWER
\CLRSH{Fibonacci(n)}
\startCLRSCODE
F = \ALGO{NEW-ARRAY(0,n)}
F[0] = 1
F[1] = 1
for i = 2 to n
	F[i] = F[i - 1] + F[i - 2]
return F[n]
\stopCLRSCODE

圖中有 n 個節點,數列中的每個元素是一個節點。
所有邊都是由稍大的子問題指向稍小的子問題:
如由 $F_2$ 起始的兩條邊分別指向 $F_1$ 和 $F_0$。
除了 $0$ 和 $1$ 兩個節點外,
其他節點均會發出兩條邊指向前置節點,
即總邊數爲 $2\times(n-1)=2n-2$。
節點數爲 $n+1$。

\externalfigure[e14_1_6-1]
\stopANSWER

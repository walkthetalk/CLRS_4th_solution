\startEXERCISE
假設 \ALGO{OPTIMAL-BST} 不維護表 $\omega[i,j]$,
而是在第 9 行利用公式(14.12)直接計算 $\omega(i,j)$,
並在第 11 行使用。
如此改動對漸進時間複雜度會有何影響?
附公式(5-5):
\startformula
\omega(i,j) = \sum_{l=i}^{j}p_l + \sum_{l=i-1}^{j}q_l
\stopformula
\stopEXERCISE

\startANSWER
此改動不會影響算法的漸進時間複雜度。
就改動本身而言,時間由 $\Theta(1)$ 變爲 $\Theta(j-i)$。
但是,後面的循環本來就是 $\Theta(j-i)$,
所以再加一個 $\Theta(j-i)$,
結果還是 $\Theta(j-i)$,從而對整個算法而言,
時間複雜度沒什麼變化,還是 $\Theta(n^3)$。
\stopANSWER

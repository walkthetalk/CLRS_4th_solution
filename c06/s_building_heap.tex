\startsection[
  title={Building heap},
]

%e6.3-1
\startEXERCISE
參照圖 6-3 的方法,
說明 \ALGO{BUILD-MAX-HEAP} 在數列 $A = \langle 5, 3, 17, 10, 84, 19, 6, 22, 9 \rangle$ 上的操作過程。
\stopEXERCISE

\startANSWER
\startcombination[3*2]
{\externalfigure[e6_3_1-1]}{}
{\externalfigure[e6_3_1-2]}{}
{\externalfigure[e6_3_1-3]}{}
{\externalfigure[e6_3_1-4]}{}
{\externalfigure[e6_3_1-5]}{}
{}{}
\stopcombination
\stopANSWER

%e6.3-2
\startEXERCISE
證明 $n/2^{h+1} \ge 1/2$ 對於所有 $0\le h\le \lfloor \lg n\rfloor$。
\stopEXERCISE

\startANSWER
由 $0\le h\le \lfloor \lg n\rfloor$ 可知 $2^h \le n < 2^{h+1}$。
同時除以 $2^{h+1}$,得到 $1/2 \le n/2^{h+1} < 1$。
\stopANSWER

%e6.3-3
\startEXERCISE
對於 \ALGO{BUILD-MAX-HEAP} 中第 2 行的循環控制變量 $i$ 而言,
爲什麼要求他從 $\lfloor n/2 \rfloor$ 到 $1$ 遞減,
而不是從 $1$ 到 $\lfloor n/2 \rfloor$ 遞增?
附:

\CLRSH{BUILD-MAX-HEAP(A)}
\startCLRSCODE
for i = \lfloor A.size/2\rfloor downto 1
  \ALGO{MAX-HEAPIFY(A, i)}
\stopCLRSCODE
\stopEXERCISE

\startANSWER
如果是遞增的話,就不能調用 \ALGO{MAX-HEAPIFY} 了,
因爲無法保證子樹是最大堆。
即如果從 $1$ 開始,無法保證 $A[2]$ 和 $A[3]$ 是最大堆的根節點。
\stopANSWER

%e6.3-4
\startEXERCISE
證明:對於任一含有 $n$ 個元素的堆,
高度爲 $h$ 的節點數目最多爲 $\lceil n/2^{h+1} \rceil$。
\stopEXERCISE

\startANSWER
首先,堆中葉子節點的個數爲 $\lceil n/2 \rceil$ (參見\inexercise[heap_leave])。
下面歸納證明 $h$:

\emph{初始化:} $h = 0$ 時,
葉子節點數目 $\lceil n/2 \rceil = \left\lceil n/2^{0+1} \right\rceil$;

\emph{保持:}假設對於 $h - 1$ 結論成立,
移除所有葉子節點得到的新堆含有 $n-\lceil n/2 \rceil = \lfloor n/2 \rfloor$ 個元素,
原堆中高度爲 $h$ 的節點在新樹中高度爲 $h-1$;
令新樹中高度爲 $h-1$ 的節點個數爲 $T$,則由假設可知:
\startsplitformula\startmathalignment
\NC T \NC = \lceil \lfloor n/2 \rfloor / 2^{h-1+1} \rceil \NR
\NC \NC < \lceil (n/2)/2^h \rceil \NR
\NC \NC = \left\lceil \frac{n}{2^{h+1}} \right\rceil \NR
\stopmathalignment\stopsplitformula
即對於 $h$ 結論依舊成立。
\stopANSWER

\stopsection

\startsection[
  title={The heapsort algorithm},
]

%e6.4-1
\startEXERCISE
參照圖 6-4 的方法,
說明 \ALGO{HEAPSOR} 在如下數列上的操作過程。
\startformula
A = \langle 5, 13, 2, 25, 7, 17, 20, 8, 4 \rangle
\stopformula
\stopEXERCISE

\startANSWER
\startcombination[4*3]
{\externalfigure[output/e6_4_1-1]}{}
{\externalfigure[output/e6_4_1-2]}{}
{\externalfigure[output/e6_4_1-3]}{}
{\externalfigure[output/e6_4_1-4]}{}
{\externalfigure[output/e6_4_1-5]}{}
{\externalfigure[output/e6_4_1-6]}{}
{\externalfigure[output/e6_4_1-7]}{}
{\externalfigure[output/e6_4_1-8]}{}
{\externalfigure[output/e6_4_1-9]}{}
\stopcombination
\stopANSWER

%e6.4-2
\startEXERCISE
試分析在使用下列循環不變式時, \ALGO{HEAPSORT} 的正確性:

在算法的第 2~5 行 \emph{for} 循環每次迭代開始時,
子數列 $A[1..i]$ 是一個最大堆,包含了數列 $A[1..n]$ 中最小的 $i$ 個元素,
而子數列 $A[i+1..n]$ 則包含了數列 $A[1..n]$ 中已排序的 $n-i$ 個最大元素。
\stopEXERCISE

\startANSWER
\emph{初始化:}子數列 $A[i+1..n]$ 爲空,不變式成立;

\emph{保持:} $A[1]$ 是 $A[1..i]$ 中最大的,但是小於 $A[i+1..n]$ 中所有元素。
將 $A[1]$ 和 $A[i]$ 調換後,則 $A[i..n]$ 中的元素是最大的,且是排好序的。
堆的大小減一,並調用 \ALGO{MAX-HEAPIFY} 會將 $A[1..i-1]$ 構造成最大堆。
將 $i$ 減一,繼續下一次迭代;

\emph{終止:}待 $i=1$ 時, $A[2..n]$ 是排好序的,而 $A[1]$ 是最小的,
因此整個數列爲排好序的。
\stopANSWER

%e6.4-3
\startEXERCISE[exercise:heapsort_time]
如果數列 $A$ 含有 $n$ 個元素,
且已經是升序排列, \ALGO{HEAPSORT} 的時間復雜度是多少?
如果已經按降序排列呢?
\stopEXERCISE

\startANSWER
兩者均爲 $\Theta(n\lg{n})$。

如果已經按升序排列,將其轉換成堆需要 $O(n)$,
然後需要調用 \ALGO{MAX-HEAPIFY} 共 $n-1$ 次,
每次調用需要 $\lg{k}$ 次運算。因此:
\startformula
\sum_{i=1}^{n-1}\lg{k} = \lg((n-1)!) = \Theta(n\lg{n})
\stopformula

如果已經按降序排列,則 \ALGO{BUILD-MAX-HEAP} 會快一些(常量因子),
但時間主要花在 \ALGO{HEAPSORT} 中的循環上,時間爲 $\Theta(n\lg{n})$。
\stopANSWER

%e6.4-4
\startEXERCISE
證明:最壞情況下, \ALGO{HEAPSORT} 的時間復雜度是 $\Omega(n\lg{n})$。
\stopEXERCISE

\startANSWER
與\inexercise[heapsort_time]的第一部分相同。
如果數列已經排好序,我們需要線性時間將其轉換成最大堆,然後需要 $n\lg{n}$ 的時間排序。
\stopANSWER

%e6.4-5
\startEXERCISE\DIFFICULT
證明: 在所有元素都不同的情況下, \ALGO{HEAPSORT} 的時間復雜度爲 $\Omega(n\lg{n})$。
\stopEXERCISE

\startANSWER
假設堆是完全二叉樹,元素個數爲 $n=2^k - 1$。
有 $2^{k-1}$ 個葉子節點以及 $2^{k-1} - 1$ 個內部節點。

先來看爲堆中前 $2^{k-1}$ 個元素排序。
給葉子節點着紅色,內部節點着藍色。
已着色的節點是堆的子樹。
由於有 $2^{k-1}$ 個節點已着色,
紅色的最多有 $2^{k-2}$ 個,
藍色的最少 $2^{k-2}-1$ 個。

紅色節點可以直接跳到根節點位置上,而藍色節點則需要先上移。
讓我們統計以下要將藍色節點移到根節點位置上,需要交換多少次。
交換次數最少的情況必須滿足兩點,
有 $2^{k-2} -1$ 個藍色節點,
且他們在樹中的相對位置已經滿足最大堆的要求。
如果有 $d$ 個這樣的藍色節點,則有 $i=\lg{d}$ 層,
第 $i$ 層含有 $2^i$ 個節點。
因此交換次數爲:
\startformula
\sum_{i=0}^{\lg{d}}i2^i = 2 + (\lg{d} - 2)2^{\lg{d}} = \Omega(d\lg{d})
\stopformula

遞迴式:
\startformula
T(n) = T(n/2) + \Omega(n\lg{n})
\stopformula

由主定理可得 $T(n)=\Omega(n\lg{n})$。
\stopANSWER

\stopsection

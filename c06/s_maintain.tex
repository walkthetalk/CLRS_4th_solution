\startsection[
  title={Maintaining the heap property},
]

%e6.2-1
\startEXERCISE
參照圖 6-2 的方法,說明 \ALGO{MAX-HEAPIFY(A, 3)} 在
數列 $A = \langle 27, 17, 3, 16, 13, 10, 1, 5, 7, 12, 4, 8, 9, 0\rangle$ 上的操作過程。
\stopEXERCISE

\startANSWER
\startcombination[2*2]
{\externalfigure[output/e6_2_1-1]}{}
{\externalfigure[output/e6_2_1-2]}{}
{\externalfigure[output/e6_2_1-3]}{}
{}{}
\stopcombination
\stopANSWER

%e6.2-2
\startEXERCISE
如果堆有 $n$ 個節點,那麼以根節點的子節點爲根節點的子樹最多有 $2n/3$ 個節點。
如果每個子樹最多有 $\alpha n$ 個節點,那麼常數 $\alpha$ 最小是多少?
這對遞迴(6.1)及其解有什麼影響?附遞迴式(6.1):
\startformula
T(n)\le T(2n/3) + \Theta(1)
\stopformula
\stopEXERCISE

\startANSWER
第一問,根節點的做子樹滿,右子樹最後一層空。
令層高爲 $l$,則左子樹節點數爲 $2^l - 1$,右子樹爲 $2^{l-1} - 1$,
總數 $n=3 \cdot 2^{l-1} - 1$,
左子樹佔比爲 $2/3$。

$\alpha$ 最小是 $1/2$。

這對遞迴式(6.1)沒什麼影響。
\stopANSWER

%e6.2-3
\startEXERCISE
參考 \ALGO{MAX-HEAPIFY},寫出對應的 \ALGO{MIN-HEAPIFY(A, i)} 的僞碼,
用以維護最小堆。
並比較 \ALGO{MIN-HEAPIFY} 和 \ALGO{MAX-HEAPIFY} 的運行時間。
\stopEXERCISE

\startANSWER
\CLRSH{MIN-HEAPIFY(A, i)}
\startCLRSCODE
l = \ALGO{LEFT(i)}
r = \ALGO{RIGHT(i)}
if l ≤ A.size and A[l] < A[i]
	smallest = l
else
	smallest = i
if r ≤ A.size and A[r] < A[i]
	smallest = r
if smallest ≠ i
	// 交換 $A[i]$ 和 $A[smallest]$
	\ALGO{MIN-HEAPIFY(A, smallest)}
\stopCLRSCODE

\ALGO{MIN-HEAPIFY} 和 \ALGO{MAX-HEAPIFY} 的運行時間相同。
\stopANSWER

%e6.2-4
\startEXERCISE
當元素 $A[i]$ 比其兩個子節點的值都大時,調用 \ALGO{MAX-HEAPIFY(A, i)} 會有什麼結果?
\stopEXERCISE

\startANSWER
沒有變化。執行比較後,發現 $A[i]$ 本身就是最大的,直接返回了。
\stopANSWER

%e6.2-5
\startEXERCISE
當 $i > A.size / 2$ 時,調用 \ALGO{MAX-HEAPIFY(A, i)} 會有什麼結果?
\stopEXERCISE

\startANSWER
沒有變化。因爲 $A[i]$ 是葉子節點。 \ALGO{LEFT} 和 \ALGO{RIGHT} 所返回的值都大於 $A$ 的元素個數;
 $i$ 即 $largest$,然後就返回了。
\stopANSWER

%e6.2-6
\startEXERCISE
就常量因子而言, \ALGO{MAX-HEAPIFY} 的執行效率很不錯,
但第 $10$ 行可能是個例外,有的編譯器可能會產生低效的代碼。
請用循環控制結構取代遞迴,重寫 \ALGO{MAX-HEAPIFY}。
\stopEXERCISE

\startANSWER
\CLRSH{MAX-HEAPIFY-LOOP(A, i)}
\startCLRSCODE
while true
	l = \ALGO{LEFT(i)}
	r = \ALGO{RIGHT(i)}
	if l ≤ A.size and A[l] < A[i]
		largest = l
	else
		largest = i
	if r ≤ A.size and A[r] < A[i]
		largest = r
	if largest == i
		return
	// 交換 $A[i]$ 和 $A[largest]$
	i = largest
\stopCLRSCODE
\stopANSWER

\blank

%e6.2-7
\startEXERCISE
證明:對於一個大小爲 $n$ 的堆, \ALGO{MAX-HEAPIFY} 的最壞情況運行時間爲 $\Omega(\lg{n})$。
(\hint 對於有 $n$ 個節點的堆,可以通過對每個節點設定恰當的值,
使得從根節點到葉節點路徑上的每個節點都會遞迴調用 \ALGO{MAX-HEAPIFY}。)
\stopEXERCISE

\startANSWER
結論是顯然的。
以最左邊的路徑爲例,如果堆中最大元素在此路徑上,且根節點是最小元素,
那麼爲了將最小元素放到葉子節點上,
對此路徑上的每一個節點都會調用一次 \ALGO{MAX-HEAPIFY}。
由於堆的高度爲 $\lfloor \lg{n} \rfloor$ (參考\inexercise[heap_height]),
最壞情況運行時間爲 $\Omega(\lg{n})$。
\stopANSWER

\stopsection

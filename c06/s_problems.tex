\startsubject[
  title={Problems},
]

\startPROBLEM
(用插入的方式建堆 Building a heap using insertion)
我們可以通過反復調用 \ALGO{MAX-HEAP-INSERT} 不斷向堆中插入元素來構建一個堆。
考慮 \ALGO{BUILD-MAX-HEAP} 如下實現方式
(假設所插入對象就是堆元素):

\CLRSH{BUILD-MAX-HEAP'(A, n)}
\startCLRSCODE
A.heap-size = 1
for i = 2 to n
	\ALGO{MAX-HEAP-INSERT(A, A[i], n)}
\stopCLRSCODE

\startigBase[a]
\startitem
當輸入數據相同時, \ALGO{BUILD-MAX-HEAP} 和 \ALGO{BUILD-MAX-HEAP'} 所生成的堆是否相同?
如果是,請證明;否則請舉出反例。
\stopitem

\startANSWER
不同。例如,輸入數據爲 $\langle 1, 2, 3, 4, 5, 6 \rangle$ 時,兩個堆分別如下:

\startcombination[2*1]
{\externalfigure[output/p6_1_a-1]}{}
{\externalfigure[output/p6_1_a-2]}{}
\stopcombination
\stopANSWER

\startitem
證明:調用 \ALGO{BUILD-MAX-HEAP'} 建立包含 $n$ 個元素的堆,
最壞情況下其時間復雜度爲 $\Theta(n\lg{n})$。
\stopitem

\startANSWER
最壞情況下, \ALGO{MAX-HEAP-INSERT} 的運行時間爲 $\Theta(\lg{n})$,會被調用 $n-1$ 次。
最壞情況下, \ALGO{MAX-HEAP-INSERT} 會將所有元素都移到堆的根節點上, 即需要 $\lg{k}$ 次交換,
無論 $k$ 的值是多少。
如果輸入數列已經是排好序的,則就是最壞情況。時間復雜度爲(參見\inexercise[lg_n_fac]):
\startformula
\sum_{i=2}^{n}\lg{i} = \lg(n!) = \Theta(n\lg{n})
\stopformula
\stopANSWER

\stopigBase
\stopPROBLEM

%p6-2
\startPROBLEM[problem:6-2]
(d-叉堆的分析 Analysis of d-ary heaps)
d-叉堆與二叉堆類似,只是每個非葉子節點有 $d$ 個子節點,而不是 2 個子節點。
\startigBase[a]
% a
\startitem
如何在一個數列中表示 $d$ 叉堆?
\stopitem

\startANSWER
需要修改 \ALGO{LEFT}、 \ALGO{RIGHT} 和 \ALGO{PARENT} 的定義。
第 $i$ 個元素的第 $k$ 個子節點的索引爲 $di + k - 1$,
而父節點的索引爲 $\lfloor i/d \rfloor$。(索引從 $1$ 開始)
\stopANSWER

% b
\startitem
包含 $n$ 個元素的 $d$ 叉堆的高度是多少?用 $n$ 和 $d$ 表示。
\stopitem

\startANSWER
$\log_d{n}$。
\stopANSWER

% c
\startitem
請給出 \ALGO{EXTRACT-MAX} 在 d-叉最大堆上的一個有效實現,
並用 $d$ 和 $n$ 表示其時間復雜度。
\stopitem

\startANSWER
\ALGO{EXTRACT-MAX} 的時間復雜度爲 $O(d\log_d{n})$。
\stopANSWER

% d
\startitem
請給出 \ALGO{INCREASE-KEY(A, i, k)} 在 d-叉最大堆上的一個有效實現。
並用 $d$ 和 $n$ 表示其時間復雜度
\stopitem。

\startANSWER
\ALGO{INCREASE-KEY(A, i, k)} 的時間復雜度爲 $O(\log_d{n})$。
\stopANSWER

% e
\startitem
請給出 \ALGO{INSERT} 在 d-叉最大堆上的一個有效實現,
並用 $d$ 和 $n$ 表示其時間復雜度。
\stopitem

\startANSWER
\ALGO{INSERT} 的時間復雜度爲 $O(\log_d{n})$。
\stopANSWER
\stopigBase
\stopPROBLEM

%p6-3
\startPROBLEM
(Young tableaus)
在一個 $m\times n$ 的\emph{楊氏矩陣(Young tableau)}中,
每一行的數據都是從左到右有序的,每一列數據都是從上到下有序的。
其中也會有一些值爲 $\infty$ 的數據項,用來表示不存在的元素。
因此,楊氏矩陣可以用來存儲 $r\le mn$ 個有限的數。

\simpleurl{https://en.wikipedia.org/wiki/Young_tableau}

% a
\startigBase[a]\startitem
畫出一個包含元素爲 $\{9, 16, 3, 2, 4, 8, 5, 14, 12\}$ 的 $4\times 4$ 楊氏矩陣。
\stopitem\stopigBase

\startANSWER
\startformula\startpmatrix%[location=low]
\NC      2 \NC      3 \NC     12 \NC     14 \NR
\NC      4 \NC      8 \NC     16 \NC \infty \NR
\NC      5 \NC      9 \NC \infty \NC \infty \NR
\NC \infty \NC \infty \NC \infty \NC \infty \NR
\stoppmatrix\stopformula
\stopANSWER

% b
\startigBase[continue]\startitem
對於一個 $m\times n$ 的楊氏矩陣 $Y$ 而言,請證明:
如果 $Y[1,1]=\infty$,則 $Y$ 爲空;
如果 $Y[m,n]<\infty$,則 $Y$ 爲滿(即包含 $mn$ 個元素)。
\stopitem\stopigBase

\startANSWER
如果 $Y[1,1]=\infty$,則第一行元素都要大於等於左上角元素,即第一行元素均爲 $\infty$,
而對於任一列而言,所有元素都要大於此列中第一行的元素,即整個矩陣所有元素均爲 $\infty$。
類似,如果 $Y[m,n]<\infty$,則其他元素均小於等於右下角元素,
即所有元素都不是 $\infty$,即矩陣爲滿。
\stopANSWER

% c
\startigBase[continue]\startitem
對於非空 $m\times n$ 楊氏矩陣,
請給出一個時間復雜度爲 $O(m+n)$ 的算法 \ALGO{EXTRACT-MIN}。
可以考慮使用一個遞迴過程,將規模爲 $m\times n$ 的問題分解爲規模爲 $(m-1)\times n$ 或
者 $m\times(n-1)$ 的子問題(\hint 參考 \ALGO{MAX-HEAPIFY})。
並解釋時間複雜度爲 $O(m+n)$ 的原因。
\stopitem\stopigBase

\startANSWER
$A[1,1]$ 是最小的,就是返回值,將其替換爲 $\infty$,這會破壞楊氏矩陣的性質,
用類似 \ALGO{MAX-HEAPIFY} 的過程來維持其性質。
將 $A[i,j]$ 與其鄰居比較,並將鄰居中最小的與其交換位置。
這樣會使得 $A[i,j]$ 遵循楊氏矩陣的性質,
然後將變成 $A[i,j+1]$ 或 $A[i+1,j]$ 的問題。
當 $A[i,j]$ 比所有鄰居都小時,就終止程序。
\stopANSWER

% d
\startigBase[continue]\startitem
試說明如何在 $O(m+n)$ 的時間內,
將一個新元素插入到一個未滿的 $m\times n$ 的楊氏矩陣中。
\stopitem\stopigBase

\startANSWER
與上一題類似,只是改爲從右下角開始,向左向上移動。時間不變。
\stopANSWER

% e
\startigBase[continue]\startitem
在不用其他排序算法的情況下,
試說明如何利用一個 $n\times n$ 的楊氏矩陣在 $O(n^3)$ 時間內
對 $n^2$ 個數進行排序。
\stopitem\stopigBase

\startANSWER
矩陣開始爲空,最終爲滿,插入元素 $n^2$ 個。
每次插入操作都需時間 $O(n+n)=O(n)$。復雜度爲 $n^2 O(n)=O(n^3)$。
然後在矩陣中一個一個的取元素,放入原數列中,時間復雜度一樣。
總共時間爲 $O(n^3)$。

如果允許矩陣中只有左上角一部分元素具有楊氏矩陣的性質,則可以原地排序。
\stopANSWER

% f
\startigBase[continue]\startitem
設計一個時間復雜度爲 $O(m+n)$ 的算法,
用來判斷一個給定的數是否存儲在 $m\times n$ 的楊氏矩陣中。
\stopitem\stopigBase

\startANSWER
從左下角開始,比較 $current$ 和 $key$,
如果 $current > key$,則上移,如果 $current < key$,則右移。
如果 $current = key$,則返回成功,否則直到終止。
\stopANSWER

\stopPROBLEM

\stopsubject

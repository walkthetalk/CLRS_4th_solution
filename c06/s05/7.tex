\startEXERCISE
試分析在使用下列循環不變式時, \ALGO{MAX-HEAP-INCREASE-KEY} 的正確性:

在算法的第 5~7 行 \emph{while} 循環每次迭代開始時:
\startigBase[a]
\item 如果 \ALGO{PARENT(i)} 和 \ALGO{LEFT(i)} 均存在,
則 $A[\ALGO{PARENT(i)}].key \ge A[\ALGO{LEFT(i)}].key$。

\item 如果 \ALGO{PARENT(i)} 和 \ALGO{RIGHT(i)} 均存在,
則 $A[\ALGO{PARENT(i)}].key \ge A[\ALGO{RIGHT(i)}].key$。

\item 子數列 $A[1..A.size]$ 滿足最大堆的性質,只有一個例外:
$A[i].key$ 可能大於 $A[\ALGO{PARENT(i)}].key$。
\stopigBase
可以假定調用 \ALGO{MAX-HEAP-INCREASE-KEY} 時,
子數列 $A[1..A.size]$ 滿足最大堆性質。
\stopEXERCISE

\startANSWER
\emph{初始化:} $A$ 是最大堆,除非 $A[i]$ 比他的父節點大,
因此只有 $A[i]$ 被修改過。 $A[i]$ 比其子節點大,
否則無法通過參數校驗,即不會進入循環(新值大於舊值,
 且舊值大於其父節點);

\emph{保持:} 調換 $A[i]$ 和其父節點時,仍然滿足最大堆的性質,
只有 $A[PARENT(i)]$ 可能比其父節點大。
 $i$ 變爲 \ALGO{PARENT(i)} 後此不變式仍然成立;

\emph{終止:}當到達根節點,或者 $A[i]$ 和其父節點關系滿足最大堆的性質時,
循環終止。循環終止後, $A$ 就是一個最大堆了。
\stopANSWER

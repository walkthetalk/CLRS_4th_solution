\startsection[
  title={Priority queues},
]

%e6.5-1
\startEXERCISE
試說明 \ALGO{HEAP-EXTRACT-MAX} 在如下堆上的操作過程。
\startformula
A = \langle 15, 13, 9, 5, 12, 8, 7, 4, 0, 6, 2, 1 \rangle
\stopformula
\stopEXERCISE

\startANSWER
\startcombination[2*2]
{\externalfigure[output/e6_5_1-1]}{}
{\externalfigure[output/e6_5_1-2]}{}
{\externalfigure[output/e6_5_1-3]}{}
\stopcombination
\stopANSWER

%e6.5-2
\startEXERCISE
試說明 \ALGO{MAX-HEAP-INSERT(A, 10)} 在如下堆上的操作過程。
\startformula
A = \langle 15, 13, 9, 5, 12, 8, 7, 4, 0, 6, 2, 1 \rangle
\stopformula
\stopEXERCISE

\startANSWER
\startcombination[2*2]
{\externalfigure[output/e6_5_2-1]}{}
{\externalfigure[output/e6_5_2-2]}{}
{\externalfigure[output/e6_5_2-3]}{}
\stopcombination
\stopANSWER

%e6.5-3
\startEXERCISE
要求用最小堆實現最小優先隊列,請寫出\ALGO{HEAP-MINIMUM}、
\ALGO{HEAP-EXTRACT-MIN}、\ALGO{HEAP-DECREASE-KEY} 和 \ALGO{MIN-HEAP-INSERT} 的僞碼。
\stopEXERCISE

\startANSWER
\CLRSH{HEAP-MINIMUM(A)}
\startCLRSCODE
return A[1]
\stopCLRSCODE

\CLRSH{HEAP-EXTRAC-MIN(A)}
\startCLRSCODE
if A.size < 1
	error "heap underflow"
min = A[1]
A[1] = A[A.size]
A.size = A.size - 1
\ALGO{MIN-HEAPIFY(A, 1)}
\stopCLRSCODE

\CLRSH{HEAP-DECREASE-KEY(A, i, key)}
\startCLRSCODE
if key > A[i]
	error "new key is bigger than current key"
A[i] = key
j = \ALGO{PARENT(i)}
while i > 1 and A[j] > A[i]
	\ALGO{SWAP(A[i],A[j])}
	i = j
	j = \ALGO{PARENT(i)}
\stopCLRSCODE

\CLRSH{MIN-HEAP-INSERT(A, key)}
\startCLRSCODE
A.size = A.size + 1
A[A.size] = +\infty
\ALGO{HEAP-DECREASE-KEY(A, A.size, key)}
\stopCLRSCODE
\stopANSWER

%e6.5-4
\startEXERCISE
寫出最大堆 \ALGO{MAX-HEAP-DECREASE-KEY(A,x,k)} 的僞碼。
時間複雜度如何?
\stopEXERCISE

\startANSWER
時間複雜度爲 $\Theta(\lg (n/x))$。
\CLRSH{MAX-HEAP-DECREASE-KEY(A, x, k)}
\startCLRSCODE
if k > A[x]
	error "new key is bigger than current key"
A[x] = k
do
	l = \ALGO{LEFT(x)}
	r = \ALGO{RIGHT(x)}
	// $A[b]$ 爲 $A[x],A[l],A[r]$ 三個元素中最大的
	b = x
	if l \le A.size and A[b] < A[l]
		b = l
	if r \le A.size and A[b] < A[r]
		b = r

	// $A[x]$ 是最大的,則終止
	if b = x
		break

	if b = l
		\ALGO{SWAP(A[1],A[x])}
	else if b = r
		\ALGO{SWAP(A[r],A[x])}

	x = b	// 更新 $x$,進行下次迭代
\stopCLRSCODE
\stopANSWER

%e6.5-5
\startEXERCISE
在 \ALGO{MAX-HEAP-INSERT} 的第 5 行,爲什麼要先把所插入對象的值設爲 $-\infty$,
然後在第 8 行又將其設爲目標值?
\stopEXERCISE

\startANSWER
是爲了能通過 \ALGO{HEAP-INCREASE-KEY} 中的校驗 $key < A[i]$。
\stopANSWER

%e6.5-6
\startEXERCISE
Uriah 教授建議將 \ALGO{MAX-HEAP-INCREASE-KEY} 中第 5~7 行的 \emph{while} 循環
改爲調用 \ALGO{MAX-HEAPIFY}。
這樣做有沒有問題?
\stopEXERCISE

\startANSWER
\emph{while} 循環是將數據上移,
而 \ALGO{MAX-HEAPIFY} 是下移,
如果改成調用 \ALGO{MAX-HEAPIFY} 會直接返回。
\stopANSWER

%e6.5-7
\startEXERCISE
試分析在使用下列循環不變式時, \ALGO{MAX-HEAP-INCREASE-KEY} 的正確性:

在算法的第 5~7 行 \emph{while} 循環每次迭代開始時:
\startigBase[a]
\item 如果 \ALGO{PARENT(i)} 和 \ALGO{LEFT(i)} 均存在,
則 $A[\ALGO{PARENT(i)}].key \ge A[\ALGO{LEFT(i)}].key$。

\item 如果 \ALGO{PARENT(i)} 和 \ALGO{RIGHT(i)} 均存在,
則 $A[\ALGO{PARENT(i)}].key \ge A[\ALGO{RIGHT(i)}].key$。

\item 子數列 $A[1..A.size]$ 滿足最大堆的性質,只有一個例外:
$A[i].key$ 可能大於 $A[\ALGO{PARENT(i)}].key$。
\stopigBase
可以假定調用 \ALGO{MAX-HEAP-INCREASE-KEY} 時,
子數列 $A[1..A.size]$ 滿足最大堆性質。
\stopEXERCISE

\startANSWER
\emph{初始化:} $A$ 是最大堆,除非 $A[i]$ 比他的父節點大,
因此只有 $A[i]$ 被修改過。 $A[i]$ 比其子節點大,
否則無法通過參數校驗,即不會進入循環(新值大於舊值,
 且舊值大於其父節點);

\emph{保持:} 調換 $A[i]$ 和其父節點時,仍然滿足最大堆的性質,
只有 $A[PARENT(i)]$ 可能比其父節點大。
 $i$ 變爲 \ALGO{PARENT(i)} 後此不變式仍然成立;

\emph{終止:}當到達根節點,或者 $A[i]$ 和其父節點關系滿足最大堆的性質時,
循環終止。循環終止後, $A$ 就是一個最大堆了。
\stopANSWER

%e6.5-8
\startEXERCISE
在 \ALGO{HEAP-INCREASE-KEY} 的第 6 行,
一般要通過三次賦值才能完成交換操作
(不算將對象映射到數列索引上)。
想一想如何利用 \ALGO{INSERTION-SORT} 內循環部分的思想,
只用一次賦值就完成這一交換操作?
\stopEXERCISE

\startANSWER
\CLRSH{MAX-HEAP-INCREASE-KEY(A, i, key)}
\startCLRSCODE
if key < A[i]
	error "new key is smaller than current key"
while i > 1 and A[PARENT(i)] < key
	A[i] = A[PARENT(i)]
	i = PARENT(i)
A[i] = key
\stopCLRSCODE
\stopANSWER

%e6.5-9
\startEXERCISE
試說明如何使用優先隊列來實現一個先進先出隊列,
以及如何使用優先隊列實現棧
(隊列和棧的定義見\insection[stack_and_queue])。
\stopEXERCISE

\startANSWER
對於棧,新元素具有最高優先級,而對於隊列,新元素具有最小優先級。
對於棧,新元素優先級爲 $HEAP-MAXIMUM(A) + 1$。
而對於隊列,則需要跟蹤新元素,每次增加新元素時,都用更小的優先級。
但是兩者性能均不高。
而且如果優先級會上溢或下溢,我們需要重新指定優先級。
\stopANSWER

%e6.5-10
\startEXERCISE
\ALGO{MAX-HEAP-DELETE(A, x)} 能將對象 $x$ 從堆 $A$ 中刪除。
對於一個包含 $n$ 個元素的最大堆,
請設計一個能在 $O(\lg{n})$ 時間內完成的 \ALGO{MAX-HEAP-DELETE} 算法。
\stopEXERCISE

\startANSWER
\CLRSH{MAX-HEAP-DELETE(A, x)}
\startCLRSCODE
A[x] = A[A.size]
A.size = A.size - 1
MAX-HEAPIFY(A, x)
\stopCLRSCODE
\stopANSWER

%e6.5-11
\startEXERCISE[exercise:k_merge]
請設計一個時間復雜度爲 $O(n\lg{k})$ 的算法,
可以將 $k$ 個有序鏈表合並爲一個有序鏈表,
其中 $n$ 是所有輸入鏈表包含的元素總數。
(\hint 用最小堆來完成 $k$ 路歸並)
\stopEXERCISE

\startANSWER
從各鏈表中取一個元素,放入最小堆中。
對於每個元素,我們需要跟蹤是從哪個鏈表中取出來的。
歸並時,每從堆中取出最小的元素,
則從此元素所屬鏈表中取一個元素放入堆中(除非這個鏈表爲空)。
一直重復此操作直到堆爲空。

一共需 $n$ 步,每一步都包含一個堆的插入操作(時間爲 $\lg{k}$)。
\stopANSWER

\stopsection

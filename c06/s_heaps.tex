\startsection[
  reference=section:heaps,
  title={Heaps},
]

%e6.1-1
\startEXERCISE
高度爲 $h$ 的堆,最多有多少個元素,最少又有多少個元素?
\stopEXERCISE

\startANSWER
令高度爲 $h$ 的堆中元素個數爲 $i$,則$2^h \le i \le (2^{h+1} - 1)$。
\stopANSWER

%e6.1-2
\startEXERCISE[exercise:heap_height]
證明:含 $n$ 個元素的堆的高度爲 $\lfloor \lg{n} \rfloor$。
\stopEXERCISE

\startANSWER
由上個練習可知 $n \in (2^h, 2^{h+1} - 1)$,所以高度爲 $\lfloor \lg{n} \rfloor$。
\stopANSWER

%e6.1-3
\startEXERCISE
證明:在最大堆的任一子樹中,根節點的元素值是最大的。
\stopEXERCISE

\startANSWER
這就是最大堆的性質。

令子樹的根節點爲第 $i$ 個元素,則他的子節點均小於或等於他。
由於其子節點均滿足此性質,且此性質是可傳遞的,子樹中的所有節點均小於或等於其根節點,
因此根節點是最大的。
\stopANSWER

%e6.1-4
\startEXERCISE
假設最大堆的所有元素都不相同,那麼該堆的最小元素在哪?
\stopEXERCISE

\startANSWER
可能在任何一個葉子節點上,即索引爲 $\lfloor n/2 \rfloor + 1$ 的元素
(參見\inexercise[heap_leave]),即堆數列的右半部分中。
\stopANSWER

%e6.1-5
\startEXERCISE
假設最大堆的所有元素都不相同,
對於 $2\le k\le \lfloor n/2\rfloor$,
此堆中第 $k$ 大的元素在哪一層上?
\stopEXERCISE

\startANSWER
最大的在第一層、根節點上;
第二大、第三大的在第二層;
第 $k$ 大的在第 $\lfloor \lg n\rfloor$ 層上。
\stopANSWER

%e6.1-6
\startEXERCISE
已排序的數列是最小堆嗎?
\stopEXERCISE

\startANSWER
是。對任一索引 $i$, \ALGO{LEFT(i)} 和 \ALGO{RIGHT(i)} 均要大於 $i$,
相應的元素均大於或等於 $A[i]$(數列已排序)。
\stopANSWER

%e6.1-7
\startEXERCISE
數列 $\langle 23, 17, 14, 6, 13, 10, 1, 5, 7, 12 \rangle$ 是最大堆嗎?
\stopEXERCISE

\startANSWER
不是。 $7$ 比 $6$ 大。

\externalfigure[output/e6_1_6-1]

\stopANSWER

%e6.1-7
\startEXERCISE[exercise:heap_leave]
證明:用數列表示有 $n$ 個元素的堆時,
葉子節點的索引爲 $\lfloor n/2 \rfloor + 1,\lfloor n/2 \rfloor + 2, \ldots, n$。
\stopEXERCISE

\startANSWER
若索引爲 $i$ 的節點是葉子節點,則沒有子節點,等價於:
\startsplitformula\startmathalignment[n=1]
\NC \mfunction{LEFT}(i) = 2 i > n \NR
\NC i > n/2 \NR
\NC i \ge (\lfloor n/2 \rfloor + 1) \NR
\stopmathalignment\stopsplitformula
而顯然 $i\le n$,因此 $\lfloor n/2 \rfloor + 1 \le i\le n$。
\stopANSWER

\stopsection%Heaps

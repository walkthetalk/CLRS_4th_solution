\startEXERCISE
證明:分數揹包問題具有貪心選擇性質。
\stopEXERCISE

\startANSWER
設商品數量爲 $n$,
序號爲 $i\in \{1,2,\ldots,n\}$ 的商品
價值爲 $v_i$,重量爲 $w_i$。
揹包能承重 $W$。目標是讓揹包中的商品價值最大。
不必將整個商品都放到揹包內。
即如果可以將價值最大化,可以只將商品的一部分放入揹包。
先計算每個商品的價值密度,即 $v_i / w_i$。
按價值密度對所有商品進行排序。
設商品 $j$ 的價值密度最大。

\startigBase[n]
\item Case 1:如果 $W=w_j$,則將商品 $j$ 放入揹包,問題結束;
\item Case 2:如果 $W<w_j$,則將儘量多的 $j$ 放入揹包,問題結束;
\item Case 3:如果 $W>w_j$,則將商品 $j$ 放入揹包後。
然後揹包還可以放入重量 $W-w_j$ 的商品。
後面採用貪心策略繼續解決子問題:揹包能承重 $W-w_j$,
有 $n-1$ 個商品共選擇。
需要注意的是,選擇 $j$ 時不必考慮子問題的解。
\stopigBase

因此 0-1 揹包問題具有貪心選擇性質。
\stopANSWER

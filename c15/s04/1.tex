\startEXERCISE
如何用未來最久策略管理快取?
已知有集合 $C$ 包含快取中的數據塊、
快取最多能容納 $k$ 塊數據,
請求序列 $b_1,b_2,\cdots,b_n$,
還有序列索引 $i$ (請求的是 $b_i$)。
對於每次請求,
需打印出命中還是未命中,
未命中時,
如果有數據被置換出去,也要打印出來。
\stopEXERCISE

\startANSWER
令 $b=\langle b_1,b_2,\cdots,b_n\rangle$。
先預處理,記錄每種請求的下標,採用散列表存儲,
鍵爲請求的數據塊,值是有請求序列的索引構成的隊列,
完成請求 $b_i$ 後,散列表中 $b_i$ 對應的隊列隊頭索引出隊。
快取用大根堆表示,
大根堆的關鍵字爲數據塊所對應的隊列首個索引。
如果快取已滿且需要發生置換,
堆頂的數據塊即爲將來最久的需要請求的數據塊。

\CLRSH{FURTHSET-IN-FUTURE(b, n, k)}
\startCLRSCODE
S = \ALGO{NEW-SET()}
T = \ALGO{NEW-HASH-TABLE()}
for j = i to n
	q = \ALGO{HASH-SEARCH(T, b[j])}
	if q == NIL
		q = \ALGO{NEW-QUEUE()}
		\ALGO{HASH-INSERT(T, (b[j], q))}
	\ALGO{ENQUEUE(q, j)}
// 用散列表中對應數據塊所在隊列的 $queue.front$ 比較大小
H = \ALGO{NEW-HEAP()}
for i = 1 to n
	q = \ALGO{HASH-SEARCH(T, b[i])}
	\ALGO{DEQUEUE(q)}
	if b[i] \in H
		\ALGO{PRINT(b[i], \text{"cache hit"})}
		\ALGO{DELETE(H, b[i])}
	else
		\ALGO{PRINT(b[i], \text{"cache miss"})}
		if H.n == k	// 已滿
			\ALGO{EXTRACT-MIN(H)}
	if q \ne NIL
		\ALGO{INSERT(H, b[i])}
\stopCLRSCODE
\stopANSWER

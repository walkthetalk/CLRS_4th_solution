\startPROBLEM
(Scheduling to minimize average completion time)
給定任務集合 $S=\{a_1,a_2,\ldots,a_n\}$,
其中任務 $a_i$ 需要的處理時間爲 $p_i$。
令 $C_i$ 表示任務 $a_i$ 的\emph{完成時間(completion time)},
即 $a_i$ 處理完成的時間。
目標是使得平均完成時間最小,即 $(1/n)\sum_{i-1}^{n}c_i$ 最小。
例如,有兩個任務 $a_1$ 和 $a_2$, $p_1=3$, $p_2=5$,
如果先執行 $a_2$,則 $c_2=5,c_1=8$,平均完成時間爲 $(5+8)/2=6.5$。
如果先執行 $a_1$,則 $c_1=3,c_2=8$,平均完成時間爲 $(3+8)/2=5.5$。

% a
\startigBase[a]\startitem
設計算法來調度這些任務,使得平均完成時間最小。
執行任務時不允許搶佔,即一旦開始一個任務,必須等其執行完後才能執行另一個任務。
證明算法的正確性,並分析其運行時間。
\stopitem\stopigBase

\startANSWER
輸入:任務集合 $\{a_i\}$, $1\le i\le n$。 $a_i$ 的執行時間爲 $p_i$。
如果任務按 $a_1,a_2,\ldots,a_n$ 的順序執行,則 $a_k$ 的完成時間爲:
\startformula
c_k = \sum_{i=1}^{k}p_i
\stopformula

直覺:如果先執行處理時間長的任務,則之後的所有任務的完成時間都會增加,
最好先執行處理時間短的任務。

貪婪算法:首先將任務按處理時間遞增順序排序,
即使得 $p_1 \le p_2 \le \ldots \le p_n$。
然後按 $\{a_1,a_2,\ldots,a_n\}$ 的順序執行任務。
排序時間爲 $O(n\lg n)$,執行任務時間固定或者爲 $O(n)$(用於選擇任務)。
總用時爲 $O(n\lg n)$。

正確性的證明:假設執行順序爲 $a_1 a_2 \ldots a_m \ldots a_n$,
其中 $a_m$ 的處理時間最少。
則我們可以調換 $a_1$ 和 $a_m$,
將執行順序改爲 $a_m a_2 \ldots a_{m-1} a_1 a_{m+1} \ldots a_n$。
重排後的平均完成時間要少於原來順序的平均完成時間。
\startformula\startmathalignment[n=1, m=3, align={left,left,left}, distance=2em]
\NC n \NC p_1 \NC p_m \NR
\NC n-1 \NC p_2 \NC p_2 \NR
\NC \ldots \NC \ldots \NC \ldots \NR
\NC n-(m-2) \NC p_{m-1} \NC p_{m-1} \NR
\NC n-(m-1) \NC p_m \NC p_1 \NR
\NC n-m \NC p_{m+1} \NC p_{m+1} \NR
\NC \ldots \NC \ldots \NC \ldots \NR
\NC 1 \NC p_n \NC p_n \NR
\stopmathalignment\stopformula

我們可以按每個任務對總完成時間的貢獻來考慮,這個貢獻只與任務的執行順序有關。
假設任務 $i$ 的在執行順序中的位置爲 $j$,則其貢獻爲 $p_i \times (n-(j-1))$。
這樣我們再來比較這兩種執行順序,則只有 $a_1$ 和 $a_m$ 的貢獻發生了變化,
其餘任務對總完成時間的貢獻不變。
由於 $p_m < p_1$,則 $n p_1 + (n-(m-1)) p_m > n p_m + (n-(m-1)) p_1$,
因此重新排序後總完成時間變少了,
這樣一直執行剩餘任務中執行時間最少的,就可以滿足要求。
\stopANSWER

% b
\startigBase[continue]\startitem
現在假定任務不能立刻執行,
每個任務都有一個\emph{釋放時間(release time)} $b_i$,
在此之後才可執行。
此外假定任務是可以\emph{搶佔的(preemption)},
即可以將任務掛起,稍後再繼續。
例如,任務 $a_i$ 的運行時間爲 $p_i=6$,
釋放時間爲 $b_i=1$,
他可能在時刻 1 開始運行,在時刻 4 被搶佔。
然後在時刻 10 恢復運行,
在時刻 11 再次被搶佔,
最後在時刻 13 恢復運行,
在時刻 15 運行完畢。
任務 $a_i$ 共運行了 6 個時間單位,
但運行時間被分割成三部分, $a_i$ 的完成時間爲 15。
設計算法,在新場景下求解平均完成時間最小的調度方案。
證明其正確性,並分析算法的運行時間。
\stopitem\stopigBase

\startANSWER
任意時刻 $t$,在 $b_i \le t$ 的任務中選擇\emph{剩餘}時間最小的運行。
由於可搶佔,我們可以在任意時刻執行這個動作。
實際上,只需要在有新任務準備好,以及某個任務執行完畢時才需要做此動作。
也就是說一旦開始執行任務,如果沒有新任務準備好,
則可以一直執行完畢(此過程中,當前任務的剩餘時間一直是最小)。
即一旦有新任務,則要做插入排序,新任務和當前正在執行且沒有執行完的任務都需要插入。
此算法運行時間爲 $O(n\lg n)$。
\stopANSWER

\stopPROBLEM

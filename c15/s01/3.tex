\startEXERCISE
對於作業選擇問題,並不是所有貪心方法都能得到最大兼容作業子集。
請舉例說明,在剩餘兼容作業中選擇持續時間最短者不能得到最優解。
類似地,在剩餘兼容說動中選擇與其他剩餘作業重疊最少者,
以及選擇最早開始者均不能得到最優解。
\stopEXERCISE

\startANSWER
\startigBase[a]\startitem
選擇持續時間最短作業:
\inputsamedir{tbl15.1-3-1}
此方法會選擇 $\{a_2\}$,但最優解應爲 $\{a_1,a_3\}$。
\stopitem\stopigBase

\startigBase[continue]\startitem
選擇與其他作業重疊最少的作業:
\inputsamedir{tbl15.1-3-2}
此方法會先選擇 $a_6$,然後只能選擇兩個: $a_1,a_2,a_3,a_4$ 中的一個,
以及 $a_8,a_9,a_{10},a_{11}$ 中的一個。
而最優解爲 $a_1,a_5,a_7,a_{11}$。
\stopitem\stopigBase

\startigBase[continue]\startitem
選擇最早開始的作業:只要最早開始的作業持續時間足夠長,
就只能選擇這一個作業,其他的都無法與其兼容。
\stopitem\stopigBase
\stopANSWER

\startEXERCISE
假定右一組作業,我們需要將他們安排到一些教室,
任一作業都可以在任一教室進行。
我們希望使用最少的教室完成所有作業。
設計一個高效的貪心算法求每個作業應該在哪個教室進行。

這個問題稱爲\emph{區間圖着色問題}
(interval-graph color problem)。
我們可以構造一個區間圖,頂點表示給定的作業,
邊連接不兼容的作業。
要求用最少的顏色對頂點進行着色,
使得所有相鄰頂點顏色均不相同——
這與使用最少的教室完成所有作業的問題是對應的。
\stopEXERCISE

\startANSWER
集合 $S$ 包含 $n$ 個作業。
一個顯而易見的方案:使用 \ALGO{GREEDY-ACTIVITY-SELECTOR},
先爲第一個教室找到最大集合 $S_1$ ($S_1\subseteq S$),
再爲第二個教室找到最大集合 $S_1$ ($S_1\subseteq S - S_1$),
……。最壞情況下需要時間 $\Theta(n^2)$。

還有一種更好的算法,然而,
其漸進時間只是將作業按時間排序所需的時間 $O(n\lg n)$。
如果作業的時間是小整數,甚至有可能達到 $O(n)$。

通常我們會按起始時間遍歷所有作業,
並將其指派給任何一個可用的教室。
要達到這個目的,
我們需要對作業的起始時間和結束時間進行排序,
並遍歷所有時間點。
維護兩個教室清單:
一個是當前時間點 $t$ 被佔用的教室,
另一個是當前時間點 $t$ 空閒的教室。
如果 $t$ 是某個作業的結束時間點,
則將此作業所佔教室挪到空閒清單中。
如果 $t$ 是另一個作業的起始時間點,
則可以將這個作業安排到一個空閒的教室中。

爲了儘量減少教室佔用,
作業結束時將教室放入空閒清單的頭部,
爲作業安排教室也從空閒教室頭部選取。

運行時間包括兩部分:
 a)爲作業起始、結束時間排序。
如果某作業的結束時間和另一作業起始時間相同,
則應將結束的作業放在前面。
所需時間爲 $O(n\lg n)$,
而如果作業所用時間是受限的(比如,都是小整數),
則所需時間可能是 $O(n)$。
 b)處理所有起始、結束時間點所需時間爲 $O(n)$:
 掃描 $2n$ 個時間點,
 每個時間點的時間爲 $O(1)$。
\stopANSWER

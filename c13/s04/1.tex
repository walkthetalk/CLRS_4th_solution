\startEXERCISE
證明:在 \ALGO{RB-DELETE} 中,如果節點 $y$ 是紅色,
那麼黑高不會發生變化。

\CLRSH{RB-DELETE}
\startCLRSCODE
y=z
y-original-color=y.color
if z.left == T.nil \hfill // case 1
	x = z.right
	\ALGO{RB-TRANSPLANT(T,z,z.right)} // 用右孩子代替 $z$
elseif z.right == T.nil \hfill // case 2
	x=z.left
	\ALGO{RB-TRANSPLANT(T,z,z.left)} // 用左孩子代替 $z$
else \hfill // case 3
	y=\ALGO{TREE-MINIMUM(z.right)} // $y$ 是 $z$ 的後繼
	y-original-color=y.color
	x=y.right
	if y\ne z.right // $y$ 是否節點下移
		\ALGO{RB-TRANSPLANT(T,y,y.right)} // 用右孩子代替 $y$
		y.right=z.right // $z$ 的右孩子變成 $y$ 的右孩子
		y.right.p=y
	else
		x.p = y
	\ALGO{RB-TRANSPLANT(T,z,y)} // 用後繼 $y$ 代替 $z$
	y.left = z.left // 將 $z$ 的左孩子給 $y$
	y.left.p = y
	y.color = z.color
if y-original-color == BLACK // 如果破壞了紅黑規則就進行修正
	\ALGO{RB-DELETE-FIXUP(T,x)}
\stopCLRSCODE
\stopEXERCISE

\startANSWER
$y$ 初始化成了待刪除的節點 $z$。
如果 $y$ 爲紅色,則無需最後的修正,這時:

前兩種情況顯然不影響黑高。

第三種情況: $z$ 的兩個孩子均爲黑色,
 $y$ 變成的 $z$ 的後繼,因此 $y$ 的左孩子肯定是哨兵;
又因爲 $y$ 爲紅色,他的右孩子也只能是哨兵,
否則違反了性質 5。
所以 $y$ 的兩個孩子均爲哨兵。
根據操作流程,先將哨兵移動到 $y$ 節點的位置,
然後將 $z$ 替換爲 $y$,顏色也與 $z$ 保持一致。
比較刪除前 $z$ 爲根節點的子樹和刪除後 $y$ 爲根節點的子樹,
後者比前者少了一個紅色節點,
所有簡單路徑上的黑色節點數量保持不變,即黑高不變。

綜上,如果 \ALGO{RB-DELETE} 中 $y$ 是紅色的,
那麼黑高不會發生變化。

注意,根據性質 5,左右子樹的黑高應相同,
任何節點也不可能有且只有一個孩子爲內部節點。
\stopANSWER

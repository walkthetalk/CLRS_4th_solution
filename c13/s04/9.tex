\startEXERCISE\DIFFICULT
考慮一下 \ALGO{RB-ENUMERATE(T,r,a,b)},
假設紅黑樹 $T$ 中有 $n$ 個節點,
此算法會在以 $r$ 爲根節點的子樹中找到所有滿足 $a\le k\le b$ 的節點,
並輸出這些節點的關鍵字。
要想時間複雜度達到 $\Theta(m+\lg n)$,
其中 $m$ 是說輸出關鍵字的數目,
我們該如何實現 \ALGO{RB-ENUMERATE}?
假設 $T$ 中的關鍵字沒有重複,且肯定有 $a$ 和 $b$。
如果 $T$ 中沒有 $a$ 和 $b$,
需要如何修改算法?
\stopEXERCISE

\startANSWER
我們先通過 \ALGO{RB-SEARCH(T,T.root,a)} 找到 $a$,
然後調用 $m-1$ 次 $\ALGO{RB-SUCCESSOR}$ 到達 $b$,
依次輸出這 $m-1$ 個節點的關鍵字。

\CLRSH{RB-SEARCH(T,x,k)}
\startCLRSCODE
if x == T.nil or k == x.key
	return x
if x < x.key
	return \ALGO{RB-SEARCH(T,x.left,k)}
else
	return \ALGO{RB-SEARCH(T,x.right,k)}
\stopCLRSCODE

\CLRSH{RB-MINIMUM(T,x)}
\startCLRSCODE
while x.left \ne T.nil
	x = x.left
return x
\stopCLRSCODE

\CLRSH{RB-SUCCESSOR(T,x)}
\startCLRSCODE
if x.right \ne T.nil
	return \ALGO{RB-MINIMUM(T,x.right)}
else
	y = x.p
	while y \ne T.nil and x == y.right
		x = y
		y = y.p
	return y
\stopCLRSCODE

當然也可以先找到節點 $b$,
然後調用 $m-1$ 次 \ALGO{RB-PREDECESSOR} 後到達 $a$,
並依次輸出這 $m$ 個節點的關鍵字。
以 \ALGO{TREE-PREDECESSOR} 爲基礎進行修改:

\CLRSH{RB-MAXIMUM(T,x)}
\startCLRSCODE
while x.right \ne T.nil
	x = x.right
return x
\stopCLRSCODE

\CLRSH{RB-PREDECESSOR(T,x)}
\startCLRSCODE
if x.left \ne T.nil
	return \ALGO{RB-MAXIMUM(T,x.left)}
else
	y = x.p
	while y \ne T.nil and x == y.left
		x = y
		y = y.p
	return y
\stopCLRSCODE

紅黑樹的查找與二叉搜索樹一致,
紅黑樹高度 $h=\Theta(\lg n)$,
運行時間爲 $\Theta(h) + m \times \Theta(1) = \Theta(m + \lg n)$。

如果紅黑樹中沒有 $a$ 或 $b$,
則在開始時找到最接近 $a$ 或 $b$ 的節點即可。
由於未要求按順序輸出,也可以用遞迴的方式處理:

\CLRSH{RB-ENUMERATE(x,a,b)}
\startCLRSCODE
if a\le x.key \le b
	print x.key
if a \le x.key and x.left \ne T.nil
	\ALGO{RB-ENUMERATE(x.left,a,b)}
if x.key \le b and x.right \ne T.nil
	\ALGO{RB-ENUMERATE(x.right,a,b)}
\stopCLRSCODE
\stopANSWER

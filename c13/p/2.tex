\startPROBLEM
(Join operation on red-black trees)
\emph{連接(join)}運算:取兩個動態集合 $S_1$、 $S_2$ 和一個元素 $x$,
使得對於任何 $x_1\in S_1$ 和 $x_2\in S_2$,
均滿足 $x_{1}.key \le x.key \le x_{2}.key$。
該運算返回集合 $S=S_1 \cup \{x\} \cup S_2$。
在這個問題中,討論如何在紅黑樹上實現連接運算。
\startigBase[a]\startitem%a
紅黑樹 $T$ 中,新屬性 $T.bh$ 中存儲其黑高。證明:
在樹中節點不增加額外存儲空間,
同時不增加漸進運行時間的前提下,
 \ALGO{RB-INSERT} 和 \ALGO{RB-DELETE} 可以維護這個屬性。
當沿 $T$ 下降時,如何確定所訪問節點的黑高?
要求每個節點所用時間爲 $O(1)$。
\stopitem\stopigBase
\startANSWER
對於空的紅黑樹,將其 bh 初始化爲 0。
插入只會增加黑高,刪除只會減小黑高。
插入過程中,如果 \ALGO{BALANCING} 到達了根節點,
將根節點由紅色改成了黑色,則 bh 會增 1。
這是往樹中添加黑色節點唯一的地方。
刪除過程中,如果額外的黑節點到達了根節點,則將 bh 減 1;
這是往樹中刪除黑色節點唯一的地方。
因此每次操作過程中, bh 的開銷爲 $O(1)$。
樹中每個節點的黑高,都可以由根節點的 bh 開始,每過一個黑色節點就減 1;
其開銷也爲 $O(1)$。
\stopANSWER

$T_1$ 和 $T_2$ 是兩棵紅黑樹,
有關鍵字 $x$,在 $T_1$ 中任取節點 $x_1$,
在 $T_2$ 中任取節點 $x_2$,
均滿足 $x_1.key\le x.key\le x_2.key$。
要求實現運算 \ALGO{RB-JOIN(T_1, x, T_2)},
用以銷燬 $T_1$ 和 $T_2$ 並返回一棵紅黑樹 $T=T_1\cup \{x\} \cup T_2$。
 $T_1$ 和 $T_2$ 中的節點總數爲 $n$。

\startigBase[a,continue]\startitem%b
假設 $T_{1}.bh\ge T_{2}.bh$。
試給出一個 $O(\lg n)$ 時間的算法,
使之能在 $T_1$ 中所有從黑高爲 $T_{2}.bh$ 的節點中,
選出具有最大關鍵字的黑色節點 $y$。
\stopitem\stopigBase

\startANSWER
我們只需要遍歷樹的最右路徑,即,
如果有右孩子,我們就可以一直往右走,否則就往左走。
在此過程中,每過一個黑色節點,就將黑高減 1。
直到所到達黑色節點的黑高等於 $T_2.bh$,
我們就找到了黑高爲 $T_2.bh$ 的最大關鍵字。
由於紅黑樹的高度爲 $O(\lg n)$,此運算需要時間爲 $O(\lg n)$。
\stopANSWER

\startigBase[a,continue]\startitem%c
令 $T_y$ 爲以 $y$ 爲根節點的子樹。
試說明如何在不破壞二叉搜索樹性質的前提下,
在 $O(1)$ 時間內用 $T_y\cup \{x\} \cup T_2$ 來取代 $T_y$。
\stopitem\stopigBase

\startANSWER
在 $T_1$ 中 $y$ 所在位置插入 $x$。
令 $y$ 爲左孩子, $T_2$ 爲右孩子。
假定連接運算滿足 $x_1.key\le x.key\le x_2.key$,
則會保持二叉搜索樹的性質,所需時間爲 $O(1)$。
\stopANSWER

\startigBase[a,continue]\startitem%d
要保持紅黑性質 1、 3 和 5,應將 $x$ 着成什麼顏色?
試說明如何在 $O(\lg n)$ 時間內維護性質 2 和 4。
\stopitem\stopigBase

\startANSWER
將 $x$ 着爲紅色。

我們可以執行 \ALGO{RB-INSERT-FIXUP(T_1, x)} 來維持性質 2、 4。
 $T_2$ 的黑高等於 $T_y$,
因此 \ALGO{RB-INSERT-FIXUP} 可以工作。
運行時間爲 $O(\lg n)$。
\stopANSWER

\startigBase[a,continue]\startitem%e
論證使用(b)部分的假設是不失一般性的,
並描述當 $T_1.bh\le T_2.bh$ 時所出現的對稱情況。
\stopitem\stopigBase

\startANSWER
如果 $T_1.bh < T_2.bh$,
我們會掃描 $T_2$ 最左側的路徑,找到黑高爲 $T_1.bh$ 的最小的黑色節點。
\stopANSWER

\startigBase[a,continue]\startitem%f
證明: \ALGO{RB-JOIN} 的運行時間是 $O(\lg n)$。
\stopitem\stopigBase

\startANSWER
由之前幾部分來實現 \ALGO{RB-JOIN}:
可以在 $O(1)$ 時間內計算維護黑高。
可以在 $O(\lg n)$ 時間內找到所需黑色節點 $y$。
在 $O(1)$ 時間內完成連接,
最後花 $O(\lg n)$ 時間來保持紅黑樹的性質。
總共需要時間爲 $O(\lg n)$。
\stopANSWER

\stopPROBLEM
